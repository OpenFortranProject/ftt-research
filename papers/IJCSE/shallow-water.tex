\section{Shallow Water Model}
\label{sec:shallow-water}

The numerical code used for this work is from a presentation at the NM
Supercomputing Challenge~\citep{Robey07}.  The algorithm solves the
standard 2D shallow water equations. This algorithm is typical of a
wide range of modeling equations based on conservation laws such as
compressible fluid dynamics (CFD), elastic material waves, acoustics,
electromagnetic waves and even traffic flow~\citep{Leveque02}. For the
shallow water problem there are three equations with one based on
conservation of mass and the other two on conservation of momentum.

\begin{eqnarray*}
h_{t}+(hu)_{x}+(hv)_{y} & = & 0\quad\mbox{(mass)}\\
(hu)_{t}+(h{u}^{2}+\tfrac{1}{2}gh^{2})_{x}+(huv)_{y} & = & 0\mbox{\quad($x$-momentum)}\\
(hv)_{t}+(huv)_{x}+(h{v}^{2}+\tfrac{1}{2}gh^{2})_{y} & = & 0\mbox{\quad($y$-momentum)}
\end{eqnarray*}


%h_{t}+(hu)_{x}+(hv)_{y} & = & 0\qquad\mbox{(conservation of mass)}\\
%(hu)_{t}+(h{u}^{2}+\tfrac{1}{2}gh^{2})_{x}+(huv)_{y} & = & 0\mbox{\qquad(conservation of x-momentum)}\\
%(hv)_{t}+(huv)_{x}+(h{v}^{2}+\tfrac{1}{2}gh^{2})_{y} & = & 0\mbox{\qquad(conservation of y-momentum)}\end{eqnarray*}

\noindent
where h = height of water column (mass), $u$ = x velocity, $v$ =
y velocity, and $g$ = gravity. The height $h$ can be used for mass
because of the simplification of a unit cell size and a uniform water
density. Another simplifying assumption is that the water depth is
small in comparison to length and width and so velocities in the z-direction
can be ignored. A fixed time step is used for simplicity though it
must be less than

\begin{eqnarray*}
dt \leqq \frac{dx}{\left(\left(\sqrt{gh}+|u|\right) + \left(\sqrt{gh}+|v|\right)\right)}
\end{eqnarray*}

\noindent to fulfill the CFL condition.

The numerical method is a two-step Lax-Wendroff scheme. The method
has some numerical oscillations with sharp gradients but is adequate
for simulating smooth shallow-water flows. In the following explanation,
$U$ is the conserved state variable at the center of the cell. This
state variable, $U$ $=\{h,hu,hv\}$ in the first term in the equations
below. $F$ is the flux quantity that crosses the boundary of the cell
and is subtracted from one cell and added to the other. The remaining
terms after the first term are the flux terms in the equations above
with one term for the flux in the x-direction and the next term for
the flux in the y-direction. The first step estimates the values a
half-step advanced in time and space on each face, using loops on
the faces.

\begin{eqnarray*}
U_{i+\frac{1}{2},j}^{n+\frac{1}{2}} & = & (U_{i+1,j}^{n}+U_{i,j}^{n})/2+\frac{\triangle t}{2\triangle x}\left(F_{i+1,j}^{n}-F_{i,j}^{n}\right)\\
U_{i,j+\frac{1}{2}}^{n+\frac{1}{2}} & = & (U_{i,j+1}^{n}+U_{i,j}^{n})/2+\frac{\triangle t}{2\triangle y}\left(F_{i,j+1}^{n}-F_{i,j}^{n}\right)
\end{eqnarray*}


The second step uses the estimated values from step 1 to compute the
values at the next time step in a dimensionally unsplit loop.

\begin{eqnarray*}
U_{i,j}^{n+1} = U_{i,j}^{n} &-&\frac{\triangle t}{\triangle
  x}(F_{i+\frac{1}{2},j}^{n+\frac{1}{2}}-F_{i-\frac{1}{2},j}^{n+\frac{1}{2}})\\
&-& \frac{\triangle
  t}{\triangle
  y}(F_{i,j+\frac{1}{2}}^{n+\frac{1}{2}}-F_{i,j-\frac{1}{2}}^{n+\frac{1}{2}})
\end{eqnarray*}


\subsection{Fortran implementation}

Selected portions of the data-parallel implementation of the shallow water model
are now shown.  This code serves as input to the ForOpenCL transformations
described in the next section.  The interface for the Fortran kernel procedure {\tt
  wave\_advance} is declared as:

{\small
\begin{verbatim}
subroutine wave_advance(dx,dy,dt,H,U,V,oH,oU,oV)
  !$OFP PURE, KERNEL   :: wave_advance
  real, intent(in)     :: dx,dy,dt
  real, dimension(:,:) :: H,U,V,oH,oU,oV
  contiguous  :: H,U,V,oH,oU,oV
  intent(in)  :: H,U,V
  intent(out) :: oH,oU,oV
  target      :: oH,oU,oV
end subroutine
\end{verbatim}
}

\noindent
where {\tt dx, dy, dt} are differential quantities in space $x, y$ and time $t$,
{\tt H, U}, and {\tt V} are state variables for the height and $x$ and $y$
momentum respectively, and {\tt oH, oU, oV} are corresponding output arrays used
in the double buffering scheme.  The \emph{OFP} compiler-directive attributes
{\tt PURE} and {\tt KERNEL} indicate that the procedure {\tt wave\_advance} is
to be transformed as an OpenCL kernel and that it must be pure, other than for any
pointers used to reference interior regions of the output arrays.

Temporary arrays are required for the quantities {\tt Hx, Hy, Ux, Vx}, and {\tt
  Vy,} that are defined on cell faces.  Also, the pointer variables, {\tt pH,
  pU,} and {\tt pV}, are needed to access and update interior regions of the
output arrays.  As these pointers are assigned to the arrays {\tt oH, oU, oV},
these output arrays must have the target attribute, as shown in the interface
above.  The temporary arrays and array pointers are declared as,

{\small
\begin{verbatim}
real, allocatable, dimension(:,:) :: Hx, Hy, Ux
real, allocatable, dimension(:,:) :: Uy, Vx, Vy
real, pointer,     dimension(:,:) :: pH, pU, pV
\end{verbatim}
}

Halo variables for the interior and the cell faces are declared and defined as

{\small
\begin{verbatim}
integer, dimension(4) :: face_lt, face_rt, halo
integer, dimension(4) :: face_up, face_dn

halo    = [1,1,1,1]
face_lt = [0,1,1,1];  face_rt = [1,0,1,1]
face_dn = [1,1,0,1];  face_up = [1,1,1,0]
\end{verbatim}
}

\noindent
Note that the halo definitions for the four faces each have a 0 in the
initialization. Thus the returned array copy will have a size that is larger
than any interior region that uses the full halo {\tt [1,1,1,1]}.  This is
because there is one more cell face quantity than there are cells in a given
direction.

The first Lax-Wendroff step updates state variables on the cell faces.  Assignment
statements like the following,

{\small
\begin{verbatim}
Hx = 0.5*( region_cpy(H,face_lt) +   &
           region_cpy(H,face_rt) )   &
     + (0.5*dt/dx)                   &
     * (region_cpy(U,face_lt) - region_cpy(U,face_rt))
\end{verbatim}
}

\noindent
are used to calculate these quantities.  This equation updates the array for the
height in the $x$-direction.  The second step then uses these face quantities to
update the interior region, for example,

{\small
\begin{verbatim}
face_lt = [0,1,0,0];  face_rt = [1,0,0,0]
face_dn = [0,0,0,1];  face_up = [0,0,1,0]

pH = region_ptr(oH, halo)

pH = region_cpy(H, halo)
       + (dt/dx) * ( region_cpy(Ux, face_lt) -   &
                     region_cpy(Ux, face_rt) )   &
       + (dt/dy) * ( region_cpy(Vy, face_dn) -   &
                     region_cpy(Vy, face_up) )
\end{verbatim}
}

\noindent
Note that face halos have been redefined so that the array copy
returned has the same size as the interior region.

These simple code segments show how the shallow water model is implemented in
standard Fortran using the data-parallel programming model described above.  The
resulting code is simple, concise, and easy to understand.  However it does
\emph{not} necessarily perform well when compiled for a traditional sequential
system because of suboptimal use of temporary array variables,
especially those produced by the function {\tt region\_cpy}.  This is generally true
of algorithms that use Fortran shift functions as well, as some Fortran
compilers (e.g., gfortran) do not generate optimal code for shifts.  We note (as
shown below) that these temporary array copies are replaced by scalars in the
transformed Fortran code so there are no performance penalties for using
data-parallel statements as outlined.  However, there is an increased memory
cost due to the double buffering required by the kernel execution semantics.
