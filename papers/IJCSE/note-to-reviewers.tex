\section{Notes to reviewers for IJCSE submission}

This section was added based on instructions for submission to the SI from
the GPUScA program committee and will NOT appear in the final version of the
paper.

This paper is an extended and updated version of the paper by the same title
that was submitted to the GPUScA 2011 workshop.  The first round of reviews
that were for the workshop resulted in us making a set of changes to add
detail in addressing the scope of work that the paper does and does not
cover.  Specifically, we state that the transformation technique descriped
in the paper is focused on the pure elementatal functions that are
turned into OpenCL kernels, and do not perform loop or other analyses that
are used by other methods to infer what loop bodies should be turned into
kernels.  We also provided updated information to reflect the most
recent Fortran 2008 standard treatment of pure and elemental functions.

The revised version of the GPUScA paper contains the following changes in
addition to those made based on the reviewer comments from the workshop:

- Added a subsection that details the ROSE-based transformation that does
  the actual Fortran to OpenCL conversion of the kernel.  This was not
  present in the workshop paper due to space constraints.  Based on
  comments during the presentation of the paper and discussions with
  readers, we believe that this makes the paper more complete.

- We added an additional figure to illustrate the halo concept to augment
  the discussion in the text.  This was again based on feedback that we
  received from readers in order to provide clarity with the additional
  space in the journal submission.

- Updated reference set to include recent activity in similar efforts,
  specifically the OpenACC work that was presented at Supercomputing 2011.

- Revision of the CFL condition presented in the shallow water model
  to address a missing term that, while unrelated to the transformation,
  is important for correctly describing the shallow water example.

- Numerous tweaks and minor text edits for clarification and presentation.

- Editing to formatting to match the LaTeX style for the journal.



Thank you for your time to review this updated version of the paper.


