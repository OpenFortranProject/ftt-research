%%%%%%%%%%%%%%%%%%%%%%%%%%%%%%%%%%%%%%%%%%%%%%%%
\documentclass{doublecol-new}
%%%%%%%%%%%%%%%%%%%%%%%%%%%%%%%%%%%%%%%%%%%%%%%%

\usepackage{natbib,stfloats}
\def\newblock{\hskip .11em plus.33em minus.07em}

\theoremstyle{TH}{
\newtheorem{lemma}{Lemma}[section]
\newtheorem{theorem}{Theorem}
\newtheorem{corrolary}{Corrolary}
\newtheorem{conjecture}[lemma]{Conjecture}
\newtheorem{proposition}[lemma]{Proposition}
\newtheorem{claim}[lemma]{Claim}
\newtheorem{stheorem}[lemma]{Wrong Theorem}
}

\theoremstyle{THrm}{
\newtheorem{definition}{Definition}[section]
\newtheorem{question}{Question}[section]
\newtheorem{remark}{Remark}[section]
\newtheorem{scheme}{Scheme}
}

\theoremstyle{THhit}{
\newtheorem{case}{Case}[section]
}

\makeatletter
\def\theequation{\arabic{equation}}
\makeatother

%%%%%%%%%%%%%%%%%
\begin{document}%
%%%%%%%%%%%%%%%%%

\thispagestyle{plain}

\setcounter{page}{197}

\LRH{T. Tuithung et~al.}

\RRH{Motion Compensated JPEG2000 based video compression algorithms}

\VOL{1}

\ISSUE{3/4}

\PUBYEAR{2008}

%\BottomCatch

\title{Motion Compensated JPEG2000 based video compression algorithms}

\authorA{T. Tuithung}

\affA{Department of Computer Science and Engineering,\\
 Indian Institute of Technology Kharagpur,\\
 Kharagpur 721302, India\\
Fax: 91-3222-282206/282700\\ E-mail: themri@cse.iitkgp.ernet.in}

\authorB{S.K. Ghosh*}
\affB{School of Information Technology,\\
 Indian Institute of Technology Kharagpur,\\
 Kharagpur 721302, India\\
Fax: 91-3222-282206/282700\\ E-mail: skg@iitkgp.ac.in\\ {*}Corresponding
author}

\authorC{Jayanta Mukherjee}

\affC{Department of Computer Science and Engineering,\\
 Indian Institute of Technology Kharagpur,\\
 Kharagpur 721302, India\\
Fax: 91-3222-282206/282700\\ E-mail: jay@cse.iitkgp.ernet.in}

\begin{abstract}
A new approach of Motion-Compensated JPEG2000 (MCJ2K) video compression is
proposed in this paper. It uses simplified MPEG-2 fundamentals along with
JPEG2000 encoding for compression of the image frames. MCJ2K generates
differential images using motion compensation algorithms and store the motion
vectors in a customised header. The~images are then compressed using JPEG2000
based scheme. MCJ2K have been studied on QCIF and CIF formated videos for
various rate control mechanism. The proposed scheme produced encouraging results
compared to standard MPEG-2 and MJPEG2000 schemes, especially at higher bit
rates.
\end{abstract}

\KEYWORD{video compression; JPEG2000; MJPEG2000; motion compensation; motion
vectors; rate-control.}

\REF{to this paper should be made as follows: Tuithung,~T.,
Ghosh,~S.K. and Mukherjee,~J. (2008) `Motion Compensated JPEG2000
based video compression algorithms', {\it Int. J. Signal and Imaging
Systems Engineering}, Vol.~1, Nos.~3/4, pp.197--212.}

\begin{bio}
T. Tuithung did her BTech in Computer Science and Engineering and MTech in
Information Technology from North Eastern Regional Institute of Science and and
Technology (NERIST), Itanagar, India. Currently she is doing her PhD in Computer
Science and Engineering at Indian Institute of Technology, Kharagpur, India. Her
areas of interest are image processing, video compression, algorithms.\vs{8}

\noindent S.K. Ghosh has done his PhD in the year 2002 from Indian
Institute of Technology (IIT), Kharagpur, India from the Department
of Computer Science and Engineering. He~is~presently working as an
Associate Professor in the School of Information Technology, Indian
Institute of Technology, Kharagpur. Prior to IIT Kharagpur, he~worked
for Indian Space Research Organisation (ISRO), Department of Space,
Government of India, for about eight years in the field of Satellite
Remote Sensing and GIS. His research interest includes image
processing, computer networking, education technology, remote sensing
and~GIS.\vs{8}

\noindent Jayanta Mukherjee received his BTech, MTech, and PhD
Degrees in Electronics and Electrical Communication Engineering from
the Indian Institute of Technology (IIT), Kharagpur in 1985, 1987,
and 1990, respectively. He is presently a Professor in the Department
of Computer Science and Engineering, IIT Kharagpur. He~was a
Humboldt\break
\end{bio}

\maketitle\vfill\pagebreak

\maketitle

\thispagestyle{headings}

\noindent\parbox{42.4pc}{\leftskip 12.75pc\NINE Research Fellow at
the Technical University of Munich in Germany for one year in 2002.
He also has held short term visiting positions at the University of
California, Santa Barbara, University of Southern California, and the
National University of Singapore. His research interests are in image
processing, pattern recognition, computer graphics, multimedia
systems and medical informatics. He~has published over 100~papers in
international journals and conference proceedings in these areas.
He~served in various technical program committees of international
conferences in related areas. He~received the Young Scientist Award
from the Indian National Science Academy in 1992. He~is~a~Senior
Member of the IEEE. He is a fellow of the Indian National Academy of
Engineering~(INAE).\vs{12}}

\noindent\rule{42.5pc}{1.5pt}

\maketitle

%\vspace*{-9pt}
\section{Introduction}

Over the years Discrete Cosine Transform (DCT) based encoding algorithms such
as, MPEG-2 \citep{1,2}, MPEG-4 \citep{3}, H.264 \citep{4,5} etc.
 have been proposed for video compression.
The core compression algorithm used in such algorithms  is the baseline JPEG
\citep{6} encoding algorithm. A~typical block diagram outlining these approaches
are shown in Figure~\ref{fig1}.

\noindent
 All of these approaches have the following
commonalities:

\begin{BL}
\item Use of block DCT.

\item The video stream is segmented into a Group Of Pictures
(GOP), each
       starting with an Intra (I) frame, which is encoded by the usual static
       image compression algorithm such as block DCT based base-line JPEG
       compression algorithm.

\item Block-wise (or Macroblock-wise) motion compensated error
prediction.
       Prediction may be made from previous and future frames. However, in
       our simplified framework we have considered only predictions from the
       previous frame. This type of predicted error frame is termed as a `P' %(P frames)
       frame. For each predictive frame, there is a set of motion vectors and
       corresponding error blocks, which are also further compressed using
  again the block-DCT based compression scheme, almost similar to baseline
       JPEG compression algorithm.

\item In the above schemes, typical size of a DCT block is $8
\times 8$ and that  of a macro-block for motion compensation is $16 \times 16$.
Colour images are represented in $Y-C_r-C_b$ colour space and usually $C_r$ and
$C_b$ components are subsampled at a ratio of $4:1:1$ before the application of
DCT.
\end{BL}
However one should note that there are  variations on the motion compensated
prediction algorithms, block and macro block sizes, intra-block prediction,
object based video streaming, rate control strategies etc. among these
approaches.

\enlargethispage{7pt}

Other than DCT based approaches, wavelet based approaches are also
increasingly being pursued in recent years. Specially, wavelet
\vadjust{\vspace*{-9pt}\pagebreak\vspace*{11.9pc}}based static image
compression schemes were found to be having better rate distortion
performance over the DCT based JPEG \citep{6} compression scheme.
Subsequently, the JPEG2000 \citep{7,8,9,10} standard has incorporated
wavelet based static image compression scheme. In~wavelet based approaches
usually 3-D wavelets \citep{11,12} are applied to~a~GOP. In~some work
\citep{13,14}, motion compensation is used to improve the performance.
However the usage of 3D wavelets over the GOP causes temporal artifacts
(equivalent to jittering etc.) at its boundaries \citep{15}. In~our
proposed algorithm \citep{16}, we~restrict ourselves to 2-D wavelets only.
The scheme is similar to the DCT based schemes, where Intra frames
($I$-frame) and motion compensated Predicted frames ($P$-frame) are
encoded using similar 2-D transform domain techniques. Like MPEG
\citep{17}, in our proposed algorithm we have used JPEG2000 (instead of
JPEG as the core compression technique in MPEG) towards this. It~may be
noted that JPEG2000 has already been used in the MJPEG2000 \citep{18}
video compression standard, where all the frames are intra frames and no
motion compensation is performed. The~MJPEG2000 usually provides superior
performance over MPEG-2 at very high bit rates. Our proposed scheme is a
generalisation over MJPEG2000, as~it allows motion compensation. In~many
cases, the proposed approach \citep{19} is found to be having better rate
distortion performance than the MJPEG2000 at very high bit rates. There
are a few advantages on using JPEG2000 over the JPEG-like schemes for
encoding prediction errors which have been adopted in the proposed
algorithm:\vs{-4}

\begin{NLh}{4}
\item[1] Since the wavelet transform can be applied over the
entire image, a JPEG2000
       image does not exhibit the blocky artifacts common in highly compressed
       traditional JPEG images.

\item[2] Scalability and multi-resolution representation make
JPEG2000 as ideal
       choice for scalable video encoding scheme.

\item[3] Precise frame wise rate control is possible while using
JPEG2000. This is significant in the context of errors being modelled as
{\it Gaussian distribution} \citep{21}. It~may be noted that MPEG-2 adopts
macroblock wise rate control (for~$P$~frames),  which makes error encoding
algorithm little different than the standard
       baseline JPEG encoding scheme used for $I$~frames. In~our proposed
       approach, rate control mechanisms
       for both $I$ and $P$~frames remain same, which makes the
encoder much simpler for
       implementation.

\item[4] The other advantage of JPEG2000 scheme is that the scope
of making the        scheme lossless or semi-lossless by using reversible
wavelets. This makes
        the proposed approach very attractive for the higher
bit rate video transmission,
        specially the medical videos, where physicians demand very high quality
        of the decompressed video.\vs{-2}
\end{NLh}

\noindent
 The proposed algorithm has the following distinct
features:\vs{-2}

\begin{NLh}{3}
\item[1] It uses similar motion compensation algorithm that is
used in MPEG-2. However, in the present work, we have demonstrated results using
only full pixel motion compensation and also using only predictive frames
($P$~frames).

\item[2] To make the scheme semi-lossless, we have used the
feed-back from the  compressed video stream in the prediction of errors. We~have
also used reversible wavelets for encoding error frames.

\item[3] Precise rate control is possible by allocating fixed bits
for $I$ and $P$~frames. For efficient rate distortion performance, we~have
adopted  dynamic rate control and distortion adjustment strategies.\vs{-2}
\end{NLh}
The paper is organised in the following sections. In~Section~\ref{base_algo} we
provide the basic outline of the proposed scheme and its performance on fixed
bit allocation. We~discuss about the rate and distortion control strategies in
Section~\ref{rate_control}, which make the proposed  scheme more efficient.
Section~\ref{discussion} presents a brief overall discussion of the performance
of the scheme. Finally, the conclusion is drawn in Section~\ref{conclusion}.
Experimental results and graphs are provided in the respective sections.
Further, we have considered the videos in YUV colour model for CIF and QCIF
formats throughout this paper.

\vspace*{-0.5pc}
\section{Base compression scheme}\label{base_algo}

In our scheme, we have used JPEG2000 for compressing the intra frames
or $I$~frames as well as the error frames obtained through motion
compensation. The $I$~frame compression is the usual JPEG2000 still
image compression. However, while compressing error frames
($P$~or~$B$), error values are computed for each macroblock of $Y$
($16\times 16$ pixels for each block), $U$ and $V$ ($8\times
 8$ blocks for each of $U$ and~$V$) components by estimating motion
vectors with respect to the previous (or/and next) frame(s). Motion compensation
is made over the decompressed previous frames. Then the whole error frame is
subjected to the JPEG2000 compression scheme. A~typical block diagram of the
scheme is shown in Figure~\ref{fig2}.

\noindent As JPEG2000 works with precise rate control (within a range of
target bit rates), initially we have used fixed bit per pixel allocation
for both $I$ and $P$~frames. A~parameter $\alpha$ has been considered
indicating the relative proportions (fractions) of bit investment on $I$
and $P$~frames. Let $R$ be the target rate in bits  per second (bps), $G$
be the GOP (with single $I$~frame and $(G-1)$ $P$~frames) and $f$ be the
frame per second. Let $h$ and $w$ be the height and width of a frame. The
bit per pixels for $I$~frame ($\beta_I$) and for $P$ frames ($\beta_P$)
are given~by:\vs{-2}
\begin{eqnarray}
\beta_I&=&\alpha\cdot \frac{R\cdot G}{f\cdot w\cdot h}\label{eqn_bpp_I}\\
\beta_P&=&(1-\alpha)\cdot \frac{R\cdot G}{f\cdot w\cdot h\cdot
(G-1)}.\label{eqn_bpp_P}
\end{eqnarray}
For studying the performance of the compression schemes, we have used the
Peak Signal to Noise Ratio (PSNR) measure. Let $I_s(x,y)$ denote the pixel
value at $(x,y)$ for the $s$th component ($s \in \{Y,U,V\}$) of the
original frame and $I'_s(x,y)$ denote the corresponding recovered value
after decompression. Then the PSNR is defined~as:\vs{4}
\begin{eqnarray}
\hspace*{-9pt}
\h{PSNR}&=&20\log_{10}\kern-2pt\left(\kern-2pt\frac{255}{\sqrt{\frac{\Sigma_{\forall
s \in \{Y,U,V\}}\Sigma_{\forall x}\Sigma_{\forall
y}(I_s(x,y)-I^{'}_s(x,y))^2}{\frac{3}{2}\cdot h\cdot
w}}}\kern-2pt\right)\kern-3pt.\nonumber\\[-2pt] \label{eqn_psnr}
\end{eqnarray}

\vspace*{-4pt} \noindent For the entire video, the {\it average} PSNR
value of all the frames has been considered. The effect of using
reversible and irreversible wavelets has also been studied. The JPEG2000
compression has been implemented using the {\it
Kakadu}\endnote{http://www.kakadusoftware.com} software. The number of
levels of decomposition for $I$~frame is taken as 5 and for $P$~frame the
value is~0. These values are chosen empirically. It~can~be observed that
the use of reversible wavelets leads to higher PSNR values at higher bit
rate. Experimental results have been shown in Figures~\ref{fig3}
and~\ref{fig4}, with the proposed MCJ2K using reversible wavelets.\vs{-1}

In Figures~\ref{fig3} and~\ref{fig4} we present the PSNR vs. \h{{\it bit-rate}}
performances for the proposed schemes for different values of $\alpha$ using
reversible and irreversible wavelets on QCIF and CIF videos respectively. The
graph show that the MCJ2K using reversible wavelets provides higher PSNR values
over MCJ2K with irreversible wavelets at high bit-rates. The PSNR values at
typical \h{bit-rates} using reversible wavelets with different values
of~$\alpha$ for QCIF and CIF videos, are shown in Tables~\ref{brqcif}
and~\ref{brcif} respectively. It~is observed that lower values of $\alpha$
performs better at high bit-rate. In Tables~\ref{cmpbrqcif} and~\ref{cmpbrcif}
the PSNR values are compared with those obtained by MPEG-2 and MJPEG2000 for
QCIF and CIF videos respectively. One may observe that the use of feedback from
the compressed stream improves the performance remarkably. The use of feedback
makes the lossless reconstruction of $P$ possible. In~fact, the PSNR values of
recovery of error frames and recovery of $P$~frame become equivalent in such
cases. This fact is stated in the Lemma~\ref{lemma_error_psnr}.\vs{2}

\begin{lemma}\label{lemma_error_psnr}
The PSNR value of the error recovery remains as same as the PSNR value of the
decompressed frame, when errors are computed from the previous decompressed
frame.\vs{2}
\end{lemma}

\proof{Proof} Let $\widehat{I}_{\rm prev}^{(d)}$ be the motion compensated
values from the decompressed previous frame.  Let $I_{\rm curr}$ be the original
current frame. Then the error values for the current frame are
obtained~as:\vs{2}
\begin{equation}
 E=I_{\rm curr}-\widehat{I}_{\rm prev}^{(d)}.
\end{equation}

\begin{table*}[!t]%Tab5
\caption{Results on measures of skewness and curtosis of error frames for {\it
Container. cif} video at 2449~Kbps}\label{curtcontainer}
{\NINE\begin{tabular*}{\textwidth}{@{\extracolsep{\fill}}lccccc@{}} \toprule
{\it Frame component} & {\it Avg\_Skewness} & {\it Avg\_Curtosis} & $\sigma$
{\it Skewness} & $\sigma$ {\it Curtosis} & {\it Min\_Curtosis}\\ \midrule Y &
\p{$-$}0.45 & 41.82 & 1.53  & 42.29 &9.53\\ U & $-$0.49 & \p{0}7.90  & 1.55 &
47.17 & 3.05\\
 V & $-$0.40 & \p{0}6.88  & 0.51  & 6.53 & 4.01\\
\botrule
\end{tabular*}}{}
\vspace*{4pt}
\end{table*}

\begin{table*}[!t]%Tab6
\caption{Results on measures of skewness and curtosis of error frames for {\it
News.cif} video at 2640~Kbps}\label{curtnews}
{\NINE\begin{tabular*}{\textwidth}{@{\extracolsep{\fill}}lccccc@{}} \toprule
{\it Frame component} & {\it Avg\_Skewness} & {\it Avg\_Curtosis} & $\sigma$
{\it Skewness} & $\sigma$ {\it Curtosis} & {\it Min\_Curtosis}\\ \midrule Y &
\p{$-$}0.39 & 96.08 & 1.98  & 46.56 & 5.90\\ U & $-$0.52 & 48.04  & 1.41 & 49.52
& 4.38\\
 V & $-$0.01 & 57.19  & 1.42 & 54.97 & 4.31\\
\botrule
\end{tabular*}}{}
\end{table*}

\begin{table*}[!t]%Tab7
\caption{Results on measures of skewness and curtosis of error frames for {\it
Foreman.cif} video at 3043~Kbps}\label{curtforeman}
{\NINE\begin{tabular*}{\textwidth}{@{\extracolsep{\fill}}lccccc@{}} \toprule
{\it Frame component} & {\it Avg\_Skewness} & {\it Avg\_Curtosis} & $\sigma$
{\it Skewness} & $\sigma$ {\it Curtosis} & {\it Min\_Curtosis}\\ \midrule Y &
$-$0.03 & 17.20 & 0.70  & 10.42 & 4.75\\
 U & $-$1.16 & 47.16  & 4.58  &165.36\p{0} & 2.79\\
V & \p{$-$}0.11 & 22.83  & 1.34  & 28.01 & 2.91\\ \botrule
\end{tabular*}}{}
\end{table*}

\noindent After decompression, let the error values be $E^{(d)}$. Hence,
the $I_{\rm curr}^{(d)}$ after decompression could be expressed~as:
\begin{equation}
I_{\rm curr}^{(d)}=\widehat{I}_{\rm prev}^{(d)}+E^{(d)}.
\end{equation}
Hence,\vs{-3}
\begin{eqnarray}
|E-E^{(d)}| & = & |(I_{\rm curr}-\widehat{I}_{\rm
prev}^{(d)})-E^{(d)}|\nonumber\\
 &=& |(I_{\rm curr}-\widehat{I}_{\rm prev}^{(d)})-(I_{\rm curr}^{(d)}-\widehat{I}_{\rm prev}^{(d)})|\nonumber\\
  &=&  |I_{\rm curr}-I_{\rm curr}^{(d)}|.
\end{eqnarray}
Hence the PSNR values for error recovery and frame recovery are same.\vs{-3}
\endproof

\noindent From the above lemma, we can observe that if the error recovery is
{\it lossless}, the decompressed frame is also {\it lossless}. This motivates us
to use the reversible wavelets. We~have observed that usage of reversible
wavelet slightly improves the scheme at the higher \h{bit-rate}. However, as the
dynamic ranges of errors increase by two folds, we have scaled the values by
$\frac{1}{2}$ before compression. For this reason, our scheme behaves as a
semi-lossless scheme at higher bit rates.

\subsection{Gaussian modelling of error frames}

Lemma~\ref{lemma_error_psnr} is also significant as it~converts the problem of
compressing individual frames to the compression of error frames. It~would be
interesting to study how best these errors are modelled by a Gaussian
distribution. In~that case we would be able to use the well known rate
distortion relationship of encoding of Gaussian distributed values, as described
by the following\break lemma.

\begin{lemma}\label{lemma_RD_gauss}
 The values of a random variable following Gaussian
distribution with the standard
 deviation $\sigma$ would require   $R(D)$ bits per symbol at a distortion
 {\rm (}the
 Mean Square Error {\rm (}MSE{\rm ))} of $D$ where $R(D)$
  holds the following relationship
with~$D$
\begin{equation}
R(D) = \begin{cases}
         \dfrac{1}{2} \log_2 \dfrac{\sigma^2}{D} &\quad 0 \le D \le
                    \sigma^2\\[1pc]
  0  &\quad D > \sigma^2\end{cases}.
\end{equation}
\end{lemma}

\proof{Proof} Refer \cite{20}.
\endproof

\noindent
 It was observed that in most cases the distributions
of error values do not  pass the significance levels (of~1\%) of {\it
Chi-Square} ($\chi^2$) {\it Goodness
 of Fit test} \citep{21} or the {\it Kolmogrov-Smirinov Test} \citep{21}.
 However, the distributions are
 observed as symmetric and highly peaked around its mean.
Skewness and curtosis are measures of asymmetry and sharpness of the height
relative to a normal distribution respectively.
In~Tables~\ref{curtcontainer}--\ref{curtforeman}, we~provide the average
skewness and average curtosis measures (along with their standard deviations).
They indicate that the Lemma~\ref{lemma_RD_gauss} would provide an upper bound
on the rate for a given distortion. We~have made use of this fact in designing
the rate control strategy.

\subsection{Varying macroblock sizes}

We have also studied the effect of different macroblock size on the performance
of the proposed scheme. If~errors are generated by larger macroblocks, it~is
expected the standard deviations of the error frames would be higher, on the
other hand lower macroblocks should have smaller standard deviations. But there
are increasing overheads on encoding the motion vectors for lower macroblock
sizes. If~we~assume the bit assignment on encoding full pixel motion vectors is
same (say $mv_b$ for each motion vector, in our case $mv_b = 8$~bits), a
macroblock of size $b_w \times b_h$ would require $\big\lceil \frac{w\cdot
h}{b_w\cdot b_h\vphantom{\sum_{1_1}}}\big\rceil$ number of motion vectors.
If~$\sigma_i$, $i=1,2,\dots,N$ is the standard deviation for the $i$th error
frames the average bit requirement $E_b$ for an error frame is given~by:\vs{-3}
\begin{equation}
E_b= \frac{1}{N}\sum_{i=1}^{N}\log_2(\sigma_i)+\frac{mv_b\cdot\big\lceil
\frac{w\cdot h}{b_w\cdot b_h}\big\rceil}{w\cdot h}.
\end{equation}
In Table~\ref{Avgbit}, we have shown the theoretical average bit
requirements for varying macroblock size. In~Figure~\ref{fig5}, PSNR vs.
rate curves are drawn for varying macroblock sizes. It~is empirically
observed that the macroblock size $16 \times 16$ yield better performance
empirically. Theoretical bounds on the average bit requirement are also
lower for different videos in such case. In~Figure~\ref{fig5}(a),
macroblock size $32 \times 32$ is showing better performance while in
Figure~\ref{fig5}(b) and (c), macroblock size $16 \times 16$ shows better
performance at lower bit-rates. Moreover \h{MPEG-2} also uses a macroblock
size of $16 \times 16$ and since our scheme uses the fundamentals of
\h{MPEG-2}, thus the macroblock size $16 \times 16$ is chosen for
implementing our scheme.

\begin{table}%%Tab8
\def\arraystretch{1.1}
\caption{Average bit requirement for Y-component with different macroblock
size ($8\times8$, $16\times16$ and $32\times32$) at higher bit-rate for
CIF videos (Container, News, Foreman)}\label{Avgbit}
{\NINE\begin{tabular*}{\columnwidth}{@{\extracolsep{\fill}}lccc@{}}
\toprule {\it Video} & {\it Vb\_8} & {\it Vb\_16} & {\it Vb\_32}\\
\midrule Container & 0.80 & 0.72 & {\bf 0.70}\\
 News & {\bf 0.95} & 1.05 & 1.24\\
 Foreman & {\bf 1.82} & 1.85 & 1.99\\
\botrule
\end{tabular*}}
{}
\end{table}

\subsection{Varying length of GOP}

We have also observed the effect of variation of the length of GOP. In our
scheme, as propagation of errors is arrested by the feedback from the compressed
stream, the effect on variation of the GOP length is less (refer
Figure~\ref{fig6}). In~fact, the scheme works better for larger GOP length, as
long as, there is a gain in rate for encoding a $P$ frame compared to that of an
Intra encoding. However, larger GOP will also cause, increasing values of
standard deviations for the error frames at its trailers. This puts a
restriction on the length of the GOP. This leads us to design a compression
scheme for adaptive GOP. If~the standard deviation of an error frame exceeds a
threshold (called as $\sigma$th in our work) the GOP is initialised.
In~Figure~\ref{fig7}, we~have demonstrated the performances for adaptive GOPs.
In~Table~6, we~have also provided the average lengths of GOP at
varying~$\sigma$th.\vs{-3}

\section{Rate control} \label{rate_control}

As discussed earlier, the use of JPEG2000 provides an opportunity for precise
rate control. Moreover, the Gaussian nature of error frames motivates us to use
the following lemma for the rate control of the MCJ2K scheme with fixed GOP.

\begin{lemma}\label{lemma_rt_gauss}
Let $\overline{\beta_P}$ be the average bit per pixel for encoding a $P$ frame.
Let $\sigma_i$, $i=1,2,\dots,(G-1)$ be the standard deviation of the $i$th
$P$~frame of a GOP of the length~$G$. Then, the optimal bit per pixel at which
the $i$th $P$~frame would be encoded is given~by:
\begin{equation}\label{eqn_bpp_optimal}
\beta_{P_i}=\overline{\beta_P}+ \frac{1}{2} \log_2 \frac{\sigma^2}{\sigma^{*^2}}
\end{equation}
where
\begin{equation}\label{eqn_sigma_gm}
\sigma^*= \big(\Pi_{i=1}^{(G-1)} \sigma_i\big)^{\frac{1}{(G-1)}}
\end{equation}
with the optimum rate allocation the $i$th frame distortion $D_i$ becomes,
\begin{equation}
D_i=\sigma^{*^2} 2^{-2\overline{\beta_P}}.
\end{equation}
\end{lemma}

\proof{Proof} Refer \cite{22}.\vs{-1}
\endproof

\subsection{Rate control of the MCJ2K}

The above lemma motivates us to design the following rate control algorithm for
the MCJ2K.\vs{-1}

\vspace*{9pt} \noindent
 Algorithm {\it Rate\_Control\_MCJ2K}

\vspace*{8pt}
 \noindent
{\it Encoding Parameters}: Length of GOP ($G$), relative bit investment for
$I$~frame ($\alpha$), rate~($R$).\vs{-1}

\begin{NLh}{3}
\item[1] Determine bit per pixel for encoding $I$ frame
($\beta_I$) and average
 bpp for encoding a $P$ frame ($\overline{\beta_P}$) using
 equations~(\ref{eqn_bpp_I}) and~(\ref{eqn_bpp_P}).

\item[2] Compute the geometric mean of the standard
deviations (refer equation~(\ref{eqn_sigma_gm}))  for the next $GOP$ from the
current one and use it for the purpose of~optimal bit assignment following the
(refer equation~(\ref{eqn_bpp_optimal})).

\item[3] Compute the bpp for $i$th $P$ frame ($\beta_{P_i}$)
following~the Lemma~\ref{lemma_rt_gauss} and encode them\break accordingly.
\end{NLh}
\vspace*{0.3pc} \noindent End {\it Rate\_Control\_MCJ2K}

\vspace*{8pt} \noindent
 In this algorithm, $P$ frames for the first GOP are
encoded using the average bpp as computed from (refer
equation~(\ref{eqn_bpp_P})). The geometric mean of the standard deviations of
all the $P$~frames is also computed and used for encoding $P$~frames of the next
GOP using (refer equation~(\ref{eqn_bpp_optimal})). The process is repeated for
all the subsequent GOPs in the same way. The performance is shown in the
Figure~\ref{fig7}. From the Table~\ref{brcif}, it is observed that $\alpha=50\%$
gives better PSNR value. As such $\sigma$ value at 5 and $\alpha$ at 50\% are
typically chosen for studying the performance of {\it Rate\_Control\_MCJ2K}.

At lower bit-rate, the achieved bit-rate is almost as same as the given target
bit-rate but at higher bit-rate the achieved bit-rate gets saturated even if the
target bit-rate is increased (refer Figure~\ref{fig8}).

\subsection{Distortion control of the MCJ2K}\label{distortion_control}

Interestingly, the MCJ2K scheme is more suitable for controlling distortion or
quality of the decompressed video. In~this case, given a target PSNR~$\Omega$,
the MCJ2K attempts to encode at the optimal rate for achieving the target rate.
The
 distortion $D$ is computed from $\Omega$ as follows:
\begin{equation}\label{eqn_D_PSNR}
 D=255^2\times 10^{-\frac{\Omega}{10}}.
\end{equation}
Then following the Lemma~\ref{lemma_RD_gauss}, $P$ frames are encoded. The
performance  of this strategy with different bpp's used for encoding $I$~frames
is shown  in Figure~\ref{fig9}. The achieved PSNR values given the target PSNR
values are also
 shown in the Figure~\ref{fig10}.

However, one may observe that there is a considerable gap between the achieved
PSNR values and the target PSNR values. To~reduce these gaps, we have adjusted
the target PSNR values after encoding of each frame and measuring its PSNR with
the original frame. Let the achieved PSNR value for the $i$th frame
be~$\Omega^{(d)}$.
 Then the target PSNR value for the $(i+1)$th frame is computed~as:\vs{-3}
\begin{equation}
    \Omega^{(i+1)}=2\Omega^{(i)}-\Omega^{(d)}.
\end{equation}
For the first $P$ frame of a GOP, the target PSNR is initialised to~$\Omega$.
The PSNR vs. rate curves with this distortion control strategy (referred as {\it
Quality\_Control\_MCJ2K})
 are shown in the Figure~\ref{fig11}. In~Figure~\ref{fig12},
 achieved PSNR values against the target
 PSNR values are also shown. It~may be noted that due to the
PSNR feedback, the achieved  bit-rate gets saturated faster as compared to the
scheme without using this feedback (refer Figure~\ref{fig9}).\vs{-3}

\section{Discussion}\label{discussion}

In the {\it base compression} scheme of MCJ2K (refer Section~\ref{base_algo}),
we~have experimented the bit-investment on $I$ and $P$~frames, using a
parameter~$\alpha$. It~was found empirically that lower value of $\alpha$
($\alpha = 0.5$), produces better PSNR value at the higher bit-rate. We~also
experimented on using both reversible and irreversible wavelets and found that
reversible wavelets outperforms irreversible wavelets (refer
Lemma~\ref{lemma_error_psnr}).

%\vfill\eject

Using Gaussian modelling of error frames (refer Lemma~\ref{lemma_RD_gauss}),
MCJ2K scheme is subjected to varying length of GOP base on a threshold
value~$\sigma$. We~also found that GOP size increases with higher value of
$\sigma$ (refer Table~\ref{Avggop}).

\def\thetable{9}
\begin{table}[h]%%Tab9
\def\arraystretch{1.1}
\caption{Average length of GOP for different $\sigma$th for CIF
videos}\label{Avggop}
{\NINE\begin{tabular*}{\columnwidth}{@{\extracolsep{\fill}}lcc@{}} \toprule {\it
Video} & $\sigma={\it 4.0}$ & $\sigma={\it 5.0}$\\ \midrule Container & 150 &
300\\
 News & \p{0}30 & \p{0}75\\
  Foreman & \p{00}5 & \p{0}10\\
\botrule
\end{tabular*}}{}
\vspace*{5pt}
\end{table}

\noindent In {\it rate control} of MCJ2K (refer Section~\ref{rate_control}),
Gaussian modelling of error frames along with $\alpha$ (the relative
bit-investment for $I$~frame), has been implemented and found that it performs
better than our base compression scheme at higher bit-rate.

In {\it distortion control} (quality control without feedback) of MCJ2K
scheme (refer Section~\ref{distortion_control}), given a target
PSNR~$\gamma$, the proposed MCJ2K algorithm attempts to encode at the
optimal rate for achieving the target rate using equation~(11) and
Lemma~\ref{lemma_RD_gauss}. We~also~have quality control with feedback in
which the target PSNR values are adjusted after encoding each frame (refer
equation~(12)). The effect of having feedback can be observed from
Figures~\ref{fig10} and~\ref{fig12}.

From the Table~\ref{res1}, it can be observed that quality control with
feedback scheme of MCJ2K produces better PSNR values than that of quality
control without feedback and rate control schemes of MCJ2K
(refer~Table~\ref{res1}). However, it~may be noted that quality control
with feedback saturates faster than both \h{quality-control} without
feedback and rate control schemes of MCJ2K (refer Table~\ref{res2}). It~is
also worth mentioning that due to lower bound constraints, in both types
of quality control schemes (with and without feedback), the scheme works
from a certain high bit-rate only as compared to the rate control scheme
of MCJ2K (refer Figures~\ref{fig7},~\ref{fig9} and~\ref{fig11}).

Finally, in optimal bit allocation strategy, the MCJ2K scheme adjusts
$\alpha$ for bit-investment for both the $I$ and $P$~frames.
It~behaves like MCJ2K rate control scheme at lower bit-rate but at
the higher bit-rate it tends to behave like MJPEG2000, where the GOP
size tends to become one. Hence this scheme works better than MPEG-2
at very high bit-rate (refer Figure~\ref{fig13}).

\section{Conclusion}\label{conclusion}
\vspace*{6pt}

In this paper we have  proposed a novel video compression algorithms which
use JPEG2000 as the core compression engine for compressing {\it intra}
frames as well as the motion compensated {\it prediction} error frames.
The rate and quality control algorithms for the MCJ2K scheme have been
developed and the MCJ2K  is found to perform better than MPEG-2 and
MJPEG2000 at higher bit-rates for many  cases. Extensive experimentations
have been carried out and the results are compared with standard schemes.
It~is also found that MCJ2K performs better for low motion videos as
compared to high motion videos. The proposed MCJ2K scheme can find its
applications like HDTV video conferencing, medical video transmission
etc., where high quality video is required, even at the cost of high
bit-rate.

\begin{thebibliography}{10}

\bibitem[\protect\citeauthoryear{Antonini et~al.}{1992}]{10}
Antonini, M., Barlaud, M., Mathieu,~P. and Daubechies,~I. (1992) `Image coding
using wavelet transform', {\it IEEE  Transactions on Image Processing}, Vol.~1,
No.~2, \h{pp.205--222}.

%\enlargethispage{-2pt}

\bibitem[\protect\citeauthoryear{Dufaux and Ebrahimi}{2003}]{18}
Dufaux, F. and Ebrahimi, T. (2003) `Motion JPEG2000 for wireless applications',
{\it Signal Processing Laboratory (EPLF)}, {\it Proceedings of First
International JPEG 2000 Workshop, Motion Analysis and Image Sequence
Processing}, IEEE, Lugano, Switzerland, July, pp.1--5.% \tc{AUTHOR PLEASE SUPPLY
%PAGE LOCATION AND PAGE RANGE.}

\bibitem[\protect\citeauthoryear{Gray and Neuhoff}{1998}]{22}
Gray, R.M. and Neuhoff, D.L. (1998) `Quantization: invited paper', {\it IEEE
Transactions on Information Theory}, Vol.~44, No.~6, pp.2325--2383.

\bibitem[\protect\citeauthoryear{He~and Mitra}{2005}]{20}
He, Z. and Mitra, S.K. (2005) `From rate-distortion analysis to
resource-distortion analysis', {\it IEEE Transaction on Circuit and
System for Video Technology}, Vol.~5, No.~3, pp.6--18.

\bibitem[\protect\citeauthoryear{LeGall}{1991}]{17}
LeGall, D. (1991) `MPEG: a video compression standard for multimedia
applications', {\it Communication of ACM}, Vol.~34, No.~4, pp.46--58.

\bibitem[\protect\citeauthoryear{Leung and Taubman}{2005}]{13}
Leung, R. and Taubman, D. (2005) `Transform and embedded coding techniques for
maximum efficiency and random accessibility in 3-D scalable compression', {\it
IEEE Transactions on Image Processing}, Vol.~14, No.~14, pp.1632--1646.

\bibitem[\protect\citeauthoryear{Liang et~al.}{2005}]{15}
Liang,~J., Tu,~C. and Tran,~T.D. (2005) `Optimal block boundary
pre/postfiltering for wavelet-based image and video compression', {\it IEEE
Transactions on Image Processing}, Vol.~14, No.~12, pp.2151--2158.

\bibitem[\protect\citeauthoryear{Montgomery et~al.}{2001}]{21}
Montgomery, D.C., Runger, G.C. and Hubele,~N.F. (2001) {\it Engineering
Statistics}, 2nd~ed., John~Wiley, New~York.

\bibitem[\protect\citeauthoryear{Rath et~al.}{2006}]{16}
Rath,~A., Tuithung,~T., Ghosh, S.K. and Mukherjee,~J. (2006)
`Compressing videos using JPEG2000', {\it Proceedings of the 3rd
Workshop on Computer Vision, Graphics and Image Processing}, 12--13th
January, Hyderabad, Andra~Pradesh, India, pp.93--97.

\bibitem[\protect\citeauthoryear{Senda}{1995}]{1}
Senda, Y. (1995) `A~simplified motion estimation using an approximation for the
MPEG-2 real time encoder', {\it International Conference on Acoustics, Speech
and Signal Processing}, Detroit, MI, USA, pp.3221--3224. %\tc{AUTHOR PLEASE
%SUPPLY LOCATION.}

\bibitem[\protect\citeauthoryear{Sikora}{1997}]{3}
Sikora, T. (1997) `The MPEG-4 video standard verification model', {\it IEEE
Transactions on Circuit and System for Video Technology}, Vol.~7, pp.19--31.

\bibitem[\protect\citeauthoryear{Skodras et~al.}{2000}]{9}
Skodras, A.N., Christopoulos, C.A. and Ebrahimi,~T. (2000) `JPEG2000:
the upcoming still image compression standard', {\it Proceeding of
the 11th Portuguese Conf. on Pattern Recognition (RECPA00D~20:
Invited paper)}, \h{11--12th} May, Prto, Portugal,  pp.359--366.

\bibitem[\protect\citeauthoryear{Tudor}{1995}]{2}
Tudor, P.N. (1995) `MPEG-2 video compression', {\it Electronics and
Communication Engineering Journal}, pp.257--264.

\bibitem[\protect\citeauthoryear{Tuithung et~al.}{2006}]{19}
Tuithung, T., Sinha, D., Ghosh,~S.K. and Mukherjee,~J. (2006)
`Bit-investment policy of MCJ2K: a~new video codec', {\it Proceedings
of the 1st IEEE International Conference on Signal and Image
Processing}, \h{7--9th}~December, Hubli, Karnataka, India,
pp.648--652.

\bibitem[\protect\citeauthoryear{Taubman and Marcellin}{2002}]{7}
Taubman, D. and Marcellin, M.W. (2002) {\it JPEG 2000: Image Compression
Fundamentals, Standards and Practice}, Kluwer, MA, Norwell.

\bibitem[\protect\citeauthoryear{Taubman}{1999}]{8}
Taubman, D. (1999) `High performance scalable image compression with EBCOT',
{\it Proceeding of IEEE International Conf. on Image Processing (ICIP99)},
Tokyo, Japan, Vol.~3, pp.344--348.

\bibitem[\protect\citeauthoryear{Taubman and Zakhor}{1994}]{12}
Taubman, D. and Zakhor, A. (1994) `Multirate 3-D subband coding of video', {\it
IEEE Transactions on Image Processing}, Vol.~3, No.~5, pp.572--589.

\bibitem[\protect\citeauthoryear{Wallace}{1991}]{6}
Wallace, G.K. (1991) `The JPEG still picture compression standard', {\it
Communication of ACM}, Vol.~34, No.~4, pp.30--44.

\bibitem[\protect\citeauthoryear{Wiegand et~al.}{2003}]{4}
Wiegand, T., Sullivan, G.J., Bjontegaard,~G. and Luthra,~A. (2003) `Overview of
the H.264/AVC video coding standard', {\it IEEE Transaction on Circuit and
System for Video Technology}, Vol.~13, pp.560--576.

\bibitem[\protect\citeauthoryear{Wiegand and Sullivan}{2005}]{5}
Wiegand, T. and Sullivan, G.J. (2005) `Video compression-from concepts to the
H.264/AVC standard', {\it Proceeding of the IEEE}, Vol.~93, No.~1, pp.18--31.
%\tc{AUTHOR PLEASE SUPPLY PAGE RANGE.}

\bibitem[\protect\citeauthoryear{Xu~et~al.}{1994}]{11}
Xu, J., Xiong, Z., Li, S. and Zhang,~Y. (1994) `3-D embedded subband coding with
optimal truncation (3-D~Escot)', {\it J.~Appl. Comput. Harmon. Anal.}, Vol.~10,
pp.290--315.

\bibitem[\protect\citeauthoryear{Xu et~al.}{2002}]{14}
Xu, J., Xiong, Z., Li, S. and Zhang,~Y.Q. (2002) \h{`Memory-constrained} 3-D
wavelet transform for video coding without boundary effects', {\it IEEE
Transaction on Circuit and System for Video Technology}, Vol.~12, No.~9,
\h{pp.812--818}.

\end{thebibliography}

\def\notesname{Note}

\theendnotes

%\section*{Query}
%
%\tc{AQ1: AUTHOR PLEASE CITE FIGURE 8 IN TEXT.}

\end{document}
