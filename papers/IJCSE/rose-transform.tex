\subsection{Fortran to OpenCL code conversion}

Programmers using ForOpenCL write kernel code in standard Fortran, and
a sequence of code transformations are applied to generate equivalent
code in C-based OpenCL.  These transformations are fairly
straightforward, and map a restricted subset of Fortran directly to
the C equivalent.  The transformation is based on a regular traversal
of the ROSE Sage AST that represents the Fortran code.  The visitor
used in the traversal handles four types of Sage AST Nodes: {\tt
  SgAllocateStatement}, {\tt SgFunctionDeclaration}, {\tt
  SgVariableDeclaration}, and {\tt SgExprStatement}.  These encompass
the primary activities that are commonly encountered in computational
kernels: allocation, variable and function declaration, and
expressions that correspond to arithmetic operations.  Allocate
statements are mapped to local tile variables in the kernel.  The
function declarations for the pure elemental Fortran functions are
mapped to OpenCL kernel function declarations.  Variable declarations and
expressions based on primitive (non-record) numerical types are mapped
to their direct equivalent on the C side.  

Our current prototype contains a visitor for {\tt SgFunctionCallExp}
nodes, but no conversion is performed.  We plan to provide support to
map numerical intrinsic functions to the equivalent math primitives
provided by OpenCL.  In many cases, this is a simple one to one
mapping from Fortran intrinsic to OpenCL primitive.  The more complex
case is invocation of user-defined functions from within kernel, which
will require additional analysis by the ForOpenCL conversion tool to
perform inlining of the called code into the kernel when appropriate.

In order to perform the language conversion, two ROSE {\tt SgSourceFile}
objects are created -- one corresponding to the input Fortran code, and
one corresponding to the output OpenCL code.  The transformation performs
a traversal of the input Fortran code during which AST nodes are inserted
into the OpenCL code to represent the equivalent code in the output language.
The input Fortran code is the unmodified code that defines the pure
elemental functions and related code that represent the computations to be
performed on the accelerator.  The output is seeded with a template C
source file ({\tt cl\_template.c}) that comes pre-defined with a set of
helper functions that are used by the generated OpenCL code.  These
generic pre-defined functions are primarily related to memory indexing (such
as mapping 2D logical array indices to physical 1D addresses).  Our current
prototype includes a limited set of helper functions sufficient to demonstrate
the 2D shallow water model code, but could conceivably be expanded to include
higher dimensional arrays with little effort.
