\subsection{Fortran to OpenCL code conversion}

Programmers using ForOpenCL write their code in standard Fortran, and
a sequence of code transformations are applied to generate equivalent
code in C-based OpenCL.  These transformations are fairly
straightforward, and map a restricted subset of Fortran directly to
the C equivalent.  The transformation is based on a regular traversal
of the ROSE Sage AST that represents the Fortran code.  The visitor
used in the traversal handles five types of Sage AST Nodes: {\tt
  SgAllocateStatment}, {\tt SgFunctionDeclaration}, {\tt
  SgVariableDeclaration}, {\tt SgFunctionCallExp}, and {\tt
  SgExprStatement}.  These encompass the primary activities that are
commonly encountered in computational kernels: allocation, function
invocation, variable and function declaration, and expressions that
correspond to arithmetic operations.

In order to perform the language conversion, two ROSE {\tt SgSourceFile}
objects are created -- one corresponding to the input Fortran code, and
one corresponding to the output OpenCL code.  The transformation performs
a traversal of the input Fortran code during which AST nodes are inserted
into the OpenCL code to represent the equivalent code in the output language.
The input Fortran code is the unmodified code that defines the pure
elemental functions and related code that represent the computations to be
performed on the accelerator.  The output is seeded with a template C
source file ({\tt cl\_template.c}) that comes pre-defined with a set of
helper functions that are used by the generated OpenCL code.  These
generic pre-defined functions are primarily related to memory indexing (such
as mapping 2D logical array indices to physical 1D addresses).  Our current
prototyle includes a limited set of helper functions sufficient to demonstrate
the shallow water model code, but could conceivably be expanded to include
higher dimensional arrays with little effort.
