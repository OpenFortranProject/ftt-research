\section{CAFe Implementation}

This section briefly describes how CAFe extensions to Fortran have been implemented as
source-to-source transformations via rewrite rules (for expressing basic transformations)
and rewriting strategies (for controlling the application of the rewrite rules).  A CAFe
file is transformed to Fortran and OpenCL files through generative programming techniques
using Stratego/XT tools\cite{Bravenboer200852} and is accomplished in three phases: (1)
parsing to produce a CAFe Abstract Syntax Tree (AST) represented in the Annotated Term
Format (ATerm\cite{DBLP:journals/spe/BrandJKO00}); (2) transformations of CAFe AST nodes
to Fortran and C AST nodes; and finally (3) pretty-printing to the base Fortran and OpenCL
languages.

The foundation of CAFe is the syntax definition of the base language expressed in SDF
(Syntax Definition Formalism) as part of the Open Fortran Project (OFP)\cite{OFP:git:url}.
CAFe is defined in a separate SDF module that extends the Fortran 2008 language standard
with only 11 context-free syntax rules.  Parsing is implemented in Stratego/XT by a
scannerless generalized-LR parser and the conversion of transformed AST nodes to text is
accomplished with simple pretty-printing rules.


\subsection{Transformations for Concurrent Procedures}

A key component of code generation for CAFe is the targeting of a Fortran pure procedure
or a \texttt{do concurrent} construct for execution on a particular hardware architecture.
The execution target can be one of several choices, including serial execution by the
current program image, parallel execution by inlining with OpenMP compiler directives, or
parallel execution by heterogeneous processing elements with a language like OpenCL.

We have developed rewrite rules and strategies in Stratego/XT to rewrite Fortran AST
nodes to C AST nodes (extended with necessary OpenCL keywords).  The C AST ATerms have mostly a
one-to-one correspondence with Fortran terms: a pure Fortran procedure is transformed to an
OpenCL kernel; Fortran formal parameters are transformed to C parameters (with a direct mapping of
types); and local Fortran variable declarations are rewritten as C declarations.  Similarly,
Fortran executable statements are rewritten as C statements.  The only
minor complication is mapping the CAFe local, array index view to the global C index space.  This
translation is facilitated by a Fortran symbol table that stores array shape and halo size
information.  However, these transformation are not entirely completed and some of the
OpenCL kernels used in the example section have been coded by hand.

\subsection{Transformations at the Calling Site}

Transformations of a CAFe procedure call site are more difficult, though technically
straight forward.  The Fortran function call must be transformed to a call to run the
OpenCL kernel (generated as described above).  This is facilitated by use of the ForOpenCL
library which provides Fortran bindings to the OpenCL
runtime\cite{Sottile:2013:FTE:2441516.2441520}.  However, this usage requires the
declaration of extra variables, allocation of memory on the OpenCL device (subimage),
transfer of memory, marshalling of kernel arguments, and synchronization.

These transformations are accomplished using several rewrite stages using Stratego/XT
strategies:
(1) a symbol table is produced in the first pass to store information related to arrays
including, array shape and allocation status;
(2) additional variables are declared that are used to maintain the OpenCL runtime state,
including the OpenCL device, the OpenCL kernel, and OpenCL variables used to marshall
kernel arguments; and
(3) all CAFe code related to subimage usage is desugared (lowered) to standard Fortran
with calls to the ForOpenCL library as needed.
The latter step includes the allocation of OpenCL device memory, transfer of memory to and
from the OpenCL device, marshalling of kernel arguments, running of the OpenCL kernel, and
synchronization.

Though not yet available, similar rewrite strategies are planned for targeting programming
models other than OpenCL including parallel execution with OpenMP and OpenACC directives.
However, as CAFe is designed, simple serial execution --- with respect to subimage tasks
--- is automatically provided by the regular CAF compiler.
