\section{Introduction}

\begin{comment}
Fortran is essential for exascale programming. The explicit policy of non-overlapping arrays (enabling compilers to more easily vectorize and parallelize automatically), simple loop syntax, longstanding support for robust real and complex arithmetic, along with the vast quantity of encoded scientific knowledge mean Fortran will remain the primary development language for many domain scientists who write software. Even as clean-slate Fortran development efforts decline, the value of the legacy Fortran software base and its ongoing refactoring and extension will be essential for decades to come.

Execution environments must undergo more substantial changes than programming languages and environments. Current runtime systems are very good at concurrent execution, but are not designed for lightweight threading (except GPUs) or locality-aware task mapping. Thread-scalable computing will require better, more transparent on-node thread parallel environments. Fortunately, we see much progress lately, underneath existing programming languages and environments.

\emph{Adding Tasking is Critical}

Although programming and execution environments are evolving smoothly toward exascale capabilities, we do have disruptive changes ahead. Most scalable parallel applications today have simple data and work decompositions: Each MPI rank owns a static portion of large data objects, e.g., a subdomain of a large distributed global domain, and each rank executes its code sequentially (potentially vectorizing), or with modest thread-parallel capabilities. This approach works on existing NUMA multicore systems by assigning multiple MPI ranks to a node and using OpenMP across a handful of cores, but performance using this approach is not sustainable as core counts continue to increase.

Tasking, with work granularities sufficiently large to make effective use of one or a few cores, must be added to most applications in order to sustain performance improvement as concurrency demands increase. Specifically, tasking requires one or more levels of additional decomposition (at least logically) of data objects, e.g., create multiple patches or tiles from each MPI subdomain, and assign tasks to execute concurrently on these patches. Within a single shared memory node, tasks can in principle cooperate closely, executing dataflow patterns, sharing data and otherwise collaborating in lightweight

\cite{HPCWire:Heroux}
\end{comment}

A goal of high-level programming languages should be to \emph{allow} a programmer to
develop software for a specific hardware architecture, without actually \emph{exposing} hardware
features.  In fact, allowing low-level hardware details to be controlled by the programmer is actually
a barrier to software performance portability: the explicit use of low-level hardware details 
leads to code becoming excessively specialized to a given target system.  The path forward for high-level
programming languages should be to allow a programmer to provide more semantic information
about their \emph{intent}, thus allowing the compiler more freedom to choose how to instantiate this
intent in a specific implementation in order to retarget applications to new architectures.

Performance portability (or, even simply portability) has been a perpetual problem in computing,
especially within the high performance computing community in which relatively new and immature
emerging technologies are constantly being tried and tested.  As processor architectures hit limitations due
to physical and manufacturing constraints, we are seeing a trend towards more diversity in
processor designs than were experienced for the bulk of the 1990s and 2000s.  Portability, especially
performance portability, is becoming more challenging as moving to a new CPU may entail more than simple
tuning if the architecture is fundamentally different than previous ones.

A number of options are available to an HPC programmer beyond just using a serial
language with MPI or OpenMP.  New parallel languages have been under development for over a decade, 
for example, Chapel and the PGAS languages including Coarray Fortran (CAF).  Options for 
heterogeneous computing
include usage of OpenACC, CUDA, CUDA Fortran, and OpenCL for programming attached
accelerator devices.  Each of these choices provides the programmer with abstractions over parallel
systems that expose different levels of detail about the specific systems to compile to, and as such,
entail different degrees of effort to design for and realize performance portability.

A reasonable solution given today's language limitations is to use MPI for distributed memory
parallelism \emph{plus} OpenMP for on-chip parallelism.  This hybrid approach is taken by two
successful, large-scale AMR multiphysics code frameworks, FLASH and Chombo; currently however,
neither of these codes support hardware accelerators\cite{DBLP:journals/corr/DubeyS13}.
%%This is the solution proposed by Cray and PGI \cite{BOF_SC10} (currently Chapel does not target GPUs\cite{Brad?}).
Other choices for expressing on-chip parallelism are OpenCL~\cite{opencl11} and NVIDIA's CUDA.
Unfortunately, achieving high performance using either of these two languages can be a daunting
challenge and there are no guarantees that either language is suitable for future hardware
generations.  

Another potential option is to abandon the stricture of message passing within a procedural
programming language altogether.  For example, the Open Community Runtime (OCR)
project\cite{OCR:wiki:url} provides an asynchronous task-based runtime designed for
machines with high-core count\cite{Dokulil20151453}, such as can be expected in any
exascale hardware design.  The project's goal is to create a framework and reference
implementation to help developers explore programming methods to improve the power
efficiency, programmability, and reliability of HPC applications while maintaining
application performance.

\begin{comment}
OCR helps the application developer with the complex process of writing multi-core
applications.  It does this by providing facilities for creating and executing a set of
tasks.  The OCR runtime schedules the execution of these tasks based on events which
embody dataflow and code flow dependencies; when all of the dependencies for a task are
satisfied, the task can be run.
\end{comment}

It cannot be emphasized too strongly that using the OpenCL programming language and
libraries (or programming for the OCR environment) presents a high hurdle for the
application developer creating real scientific applications.  A goal of this paper is to
examine ways in which high-level syntax based on Fortran coarrays can be utilized by
compiler-based tools to \emph{target} programming environments like OpenCL, thus
freeing the developer from the task of frequently porting the application to new complex
environments.  The hope is that high-level syntax can hide the complexities of the
underlying hardware \emph{and software} by delegating some optimization tasks to the
compiler instead of the programmer.

%% MJS: this seems like a natural place for a subsection break, but it looks ugly
%%      in the IEEE two column style

With the addition of coarrays in the 2008 standard, Fortran is now a \emph{parallel}
language.  Originally called Coarray Fortran (CAF) by its
authors\cite{Numrich:1998:CFP:289918.289920}, parallelism in Fortran is similar to MPI in
that all cooperating CAF processes (called \emph{images}) execute a Single Program but with
different (Multiple) Data (SPMD).  Key concepts in CAF are parallel execution, distributed
memory allocation, and remote memory transfer.  Parallel execution and memory allocation
are straightforward as each program is replicated $NP$ times with each program image
allocating its own memory locally.  Special syntax is provided to indicate which arrays
are visible between distributed memory processes and remote memory transfer is indicated
with new array syntax using square brackets, for example,
\begin{verbatim}
   U(1) = U(1)[3]
\end{verbatim}
copies memory from the first element of \texttt{U} residing on image number 3 into
the corresponding element of the \emph{local} array \texttt{U}.  The fundamental programming
model is similar to that provided by message passing systems.

Since each CAF process executes the same program, there are no good avenues for exploiting
task-based parallelism.  Neither is there a adequate mechanism for dynamic coarray memory
allocation, nor a mechanism for the remote execution of functions.  In CAF, like MPI, the
programmer often combines distributed memory processes running the same program image with a
shared memory threaded programming model using OpenMP (or OpenACC).  This model breaks down
when dealing with architectures that are not isomorphic to the model for which MPI and
relatives were designed -- a distributed collection of homogenous processing elements in
which each processing element supports a small number of parallel threads of execution.  For
example, an HPC system that makes use of traditional multicore CPUs as well as GPGPU accelerators
breaks this model and, to effectively program it, currently requires a mixture of programming
systems in a single application.

In this paper we propose CAFe: \emph{Co-Array Fortran Extensions}.  These extensions to CAF introduce
relatively minor additions to the existing coarray portion of the Fortran language.  They
include the concept of subimages in order to nest coarray images, semantics for distributed
array declarations, memory placement, remote memory transfer, and remote task execution
and synchronization.  CAFe
builds upon the existing standardized coarray features of Fortran.  We will introduce the
proposed CAFe extensions, define their semantics, demonstrate their use in a simple
example, and provide experimental results demonstrating the relative
performance of computation and communication in a prototype implementation executing
in a heterogeneous environment.
CAFe is complementary to
previous work extending coarray Fortran\cite{mellor-crummey:2009:caf2,jin:2011:caf2}.

%Topics highlighted in this section are: 1. Distributed memory array allocation;
%2. Explicit memory placement; 3. Remote memory transfer; and 4. Remote execution.  This
%description is within the context of extensions to Fortran; as shorthand, these extensions
%are referred to as CAFe, for Coarray Fortran extensions.  

%%% MJS: stopped editing here 1/7/2016

\begin{comment}
Historically it has been hard to create parallel programs.  The
responsibility (and difficulty) of creating \emph{correct} parallel
programs can be viewed as being spread between the programmer and the
compiler.  Ideally we would like to have parallel languages that make
it easy for the programmer to express correct parallel programs --- and
conversely it should be difficult to express incorrect parallel
programs.  Unfortunately, many current languages and standards place
all of the responsibility on the user.  The best example of this are
programs written using MPI (Message Passing Interface), where the
programmer expresses all parallelism in terms of calls to library
routines and the serial C or Fortran compiler knows nothing about the
parallel semantics of the combined language plus MPI library.

Of course one would hope that over time code complexity goes down as
better languages allow compilers to take on a greater share of the
parallel complexity burden.  Multithreaded code written using OpenMP
can be significantly simpler than the corresponding code in which the
programmer explicitly manages a pool of threads themselves.  This is
somewhat true for distributed memory parallelism using Unified
Parallel C (UPC) and Coarray Fortran.  With these PGAS
(Partitioned Global Address Space) extensions, C and
Fortran compilers are now aware of parallelism and now generate
message passing code that previously had been handled by the MPI
library.  In some instances the compiler is able to perform
optimizations that it is not able to do with a library-based scheme
like MPI \cite{preissl:2011:SC}.

%%% Would like to say something here about HPC as a language based solution to reducing code complexity

However, in large part these languages are mostly syntactic sugar for
message passing and do not provide a substantial decrease in code
complexity when application performance is a
goal\cite{hasert:2011:EuroMPI}.  While skilled programmers in the HPC
community have become accustomed to the level of complexity of an MPI
program, the problem for programmers is that hardware is changing in
ways that increase the level of on-chip parallelism.  Indeed, the
current generation of machines could be the last to provide
homogeneous multi-core parallelism\cite{JORSjors.aw}.  For ultimate
performance, programmers must envision coding for heterogeneous
architectures with huge new levels of on-node parallelism, at the same
time they account for off-node parallelism.
Since languages have not evolved that allow the compiler to take up
this increase, the complexity for a programmer has necessarily
dramatically increased.  Unfortunately, considering the complexity of
large multiphysics code frameworks, it could take several person years
to retarget an application for \emph{each} new heterogeneous
platform\cite{DBLP:journals/corr/DubeyS13}.


In this paper we examine a language-based paradigm that allows the
compiler to take on a larger portion of the code complexity burden.
Anyone programming in OpenCL is aware that explicit loop
structures over array elements in a serial program are removed and
replaced by a kernel program that is run over all of the elements of
the input arrays.  We propose \emph{Locally Orientated Programming
extensions} (LOPe) to the Fortran and potentially C languages that
formally adopt this programming model.  The LOPe programming model
explicitly separates numerical complexity from parallel complexity ---
leaving much of the staging and assembling of data to the compiler ---
and is applicable to the solution of stencil-based algorithms that
provide data access patterns that are regular and spatially local.

\end{comment}

% OpenACC directives and features:
%\begin{verbatim}
%● acc_copy(in|out)( ptr, bytes )
%● acc_create( ptr, bytes )
%● acc_delete( ptr, bytes )
%● acc_is_present( ptr, bytes )
%● acc_update_(device|local)( ptr, bytes )
%● acc_deviceptr( ptr )
%● acc_hostptr( devptr )
%● acc_[un]map_data( devptr, hostptr, bytes )
%● acc_memcpy_(to|from)_device
%\end{verbatim}

