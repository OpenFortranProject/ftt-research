\section{Coarray Fortran Extensions}

Topics highlighted in this section are: 1. Distributed memory array allocation;
2. Explicit memory placement; 3. Remote memory transfer; and 4. Remote execution.  This
description is within the context of extensions to Fortran; as shorthand, these extensions
are referred to as CAFe, for Coarray Fortran extensions.  CAFe is complementary to
previous work extending coarray Fortran\cite{mellor-crummey:2009:caf2,jin:2011:caf2}.


\subsection{Subimages}

We must first introduce the important new concept of a \emph{CAFe subimage}.  Fortran
images are a collection of distributed memory processes that all run the same program
(image).  CAFe extends the concept of a Fortran image by allowing images to be
hierarchical.  By this we mean that each image \emph{may} have a subimage (or subimages),
but this subimage is not visible to other regular Fortran images.  Subimages also execute
differently than normal images and may execute on different non-homogeneous hardware,
e.g., an attached accelerator device.  Subimages are task based while images all execute a
Single Program but with different (Multiple) Data (SPMD).  A task can be given to a
subimage, but execution on the subimage terminates once the task is finished.  Memory on a
subimage is permanent, however, and must be explicitly allocated and deallocated.

One obtains a subimage by executing a new CAFe function,
\begin{verbatim}
   device = GET_SUBIMAGE(device_id)
\end{verbatim}
where the integer argument represents an attached hardware device (or a separate process
or a team of threads).  If the function fails (e.g., the requested device is unavailable)
it returns the local image number of the process that is executing the
current program, obtained by a call to the Fortran 2008 function \texttt{this\_image()}.
Returning the current image (similar to MPI rank) allows program execution to proceed
correctly even if there are no attached devices.

Having obtained an subimage, we now consider how a subimage can be used within a program
to allocate memory, transfer memory, and execute tasks.


\subsection{Distributed Memory Array Declaration}

In Fortran, arrays that are visible to other program images must be declared with the
codimension attribute, for example,
\begin{verbatim}
 real, allocatable :: U(:,:)[:], V(:,:)[:]
\end{verbatim}
declares that coarrays \texttt{U} and \texttt{V} are allocatable with a rank of two
and a corank of 1.  In Fortran, square brackets \texttt{[ ]} are used to denote operations
(possibly expensive) on distributed memory where parentheses \texttt{( )} are used to
denote array elements; square brackets denote array location.

When declared thusly, coarrays must be allocated on all program images.  CAFe extends this
declaration to include memory that \emph{may} be allocated on a subimage, but does not
need to be allocated unless it is specifically used by a task executing on a subimage.


\subsection{Explicit Memory Placement}

As discussed, CAFe allows coarray memory to be conditionally allocated on a subimage.
It should be allocated conditionally because the requested subimage in a texttt{GET\_SUBIMAGE}
call may not be available.  In addition, not all coarrays need be allocated on each
subimage.  For example, the following code segment allocates memory for coarrays
\texttt{U} and \texttt{V} on all images, but only coarrays \texttt{U} is allocated on
the subimage specified by \texttt{device}:
\begin{verbatim}
   allocate(U(M,N)[*], V(M,N)[*])
   if (device \= this_image()) then
     allocate(U(M,N)[*])  [[device]]
   end if
\end{verbatim}
where \texttt{M} and \texttt{N} are constant parameters.  The first allocate statement is
mandatory and allocates the declared coarrays on all program images.  The use of the
\texttt{*} symbol is used for the last corank dimension in CAF (in this case there is only
one) to allow for the number of program images to be a runtime parameter, which can be
ascertained with a \texttt{num\_images()} function call.  If \texttt{device} is
available, its numeric value will be different from \texttt{this\_image()} and thus
coarray \texttt{U} will also be allocated on the device.  Note that placement of the
memory is specified by the use of double square bracket notation \texttt{[[ ]]}.  Like
single square brackets, double square brackets indicate something special (and possibly
expensive) is to occur, in this case possibly remote memory allocation.


\subsection{Remote Memory Transfer}

Once memory is allocated on the device, it can be initialized by copying memory from the
hosting image to the device.  This is explicitly done with normal CAF syntax.  For example,
\begin{verbatim}
  U[device] = 3.14
\end{verbatim}
assigns 3.14 to all of the array elements on \texttt{device}.  Specific array elements
can also be transferred, 
\begin{verbatim}
  U(1,:) = U(1,:)[device]
  U(M,:) = U(M,:)[device]
\end{verbatim}
where the first and last columns of \texttt{U} are copied from the device to the hosting image.

CAFe restricts memory transfer between a subimage and its hosting image only. Transferring
memory to another images subimage is not allowed.  In addition, memory transfer can only be
initiated by the hosting image; code executing on a subimage cannot use coarray notation to
transfer memory.

\subsection{Remote Execution}

The final CAFe concept introduced is that of remote execution.  As discussed, in Fortran 2008
all images (similar to MPI ranks) execute the same program.  The only way for an image
to execute something different is to conditionally execute a block of code based on explicitly
checking that an executing image's rank (\texttt{this\_image()}) meets some established criteria.
There exists no mechanism for one image to execute a block of code or a procedure on another
image.

CAFe allows images to execute tasks on hosted subimages using standard procedure calls.
For example,


\subsection{CAFe Example}

We start with the declaration of an array \texttt{U} with two local dimensions (rank) and
two distributed memory codimensions (corank).
\begin{verbatim}
   real, allocatable, :: U(:,:)[:,:]
\end{verbatim}
The corank of the array is chosen to be identical to the rank of the array so that the logical
process topology aligns in a way that allows a natural halo exchange between logically neighboring
processes (this could not occur if corank and rank are not the same).  For example, if the process
location is \texttt{[pcol,prow]}, then the right-hand halo for the local array \texttt{U} can be
obtained by the assignment
\texttt{U(M+1,:) = U(1,:)[pcol+1,prow]} where the size and cosize of
\texttt{U} are given by the allocation statement,
\texttt{allocate(U(0:M+1,0:N+1)[MP,*])}.
This allocation statement specifies (given the one element halo size provided earlier for
\texttt{U}) that the left halo column is \texttt{U(0,:)}, the right column is \texttt{U(M+1,:)},
the bottom row is \texttt{U(:,0)} and the top row is \texttt{U(:,N+1)}, leaving the interior region
\texttt{U(1:M,1:N)}.

In this allocation statement, the total number of process columns $NP$ can be obtained at runtime,
but \emph{may not} be explicitly provided (according to Coarray Fortran (CAF) rules) because the actual number of
participating processes (in Fortran called images) is variable, depending on how many processes are
requested at program startup.  In this discussion, it is \emph{assumed} that there are no holes in
the logical processor topology, thus $MP*NP = P$, where $MP$ is the number of process rows and $P$
is the total number of participating processes (images).

Once a subimage is obtained, memory on the device can be allocated,
\begin{verbatim}
   if (device /= this_image()) then
      allocate(U[device], HALO_SRC=U)   [[device]]
   end if
\end{verbatim}
There are four points to note regarding this memory allocation: 1. Memory is only allocated if a
subimage has been obtained; 2. The location where memory is allocated is denoted by regular coarray
notation \texttt{U[device]}; 3. The allocated size and halo attribute of the new array are obtained
from the previously allocated local array \texttt{U} via the notation \texttt{HALO\_SRC=U} (using
\texttt{HALO\_SRC} will also initially copy \texttt{U} to the subimage); and finally 4. The
allocation itself is \emph{executed} on the subimage device with the notation \texttt{[[device]]}.

Fortran uses square bracket notation, e.g. \texttt{[image]}, to specify on what process the
memory reference is physically located.  Square brackets are a visual clue to the
programmer that the memory reference may be remote and therefore potentially suffer a
performance penalty.  CAFe extends this by employing double-bracket notation to indicate
possibly \emph{remote subimage execution}.

Execution of the \texttt{Laplacian} task is done using the \texttt{do}
\texttt{concurrent} construct:
\begin{verbatim}
   do while (.not. converged)
      do concurrent (i=1:M, j=1:N)   [[device]]
         call Laplacian( U(i,j)[device] )
      end do
      call HALO_TRANSFER(U, BC=CYCLIC)
   end do
\end{verbatim}
There are several points that require highlighting: 1. Iteration occurs over the interior
of the array domain, \texttt{(i=1:M, j=1:N)}; 2. Execution of the loop body occurs on the
specific subimage indicated by \texttt{[[device]]}; 3. Execution of the iterates may occur
in any order, even \emph{concurrently}; 4. The local element of the array (as
defined above in reference to the definition of the concurrent procedure
\texttt{Laplacian}) is given by the indices \texttt{(i,j)}; 5. Location of memory for the
task is to be taken from the subimage as noted by \texttt{[device]}; 6. All threads must finish
execution of the loop body before further execution of the program proceeds; and 7. Transfer of
all requisite halo regions is effected by the call to the new CAFe intrinsic function
\texttt{HALO\_TRANSFER()}.  This function is a synchronization event in that all images must
complete the halo transfer before program execution continues.

Note that a transfer of halo memory is necessary after each completion of the do concurrent loop.
This must be done in order for the halo region of a coarray on a given process to be consistent
with the corresponding interior of the coarray on a logically neighboring process.
Finally, memory for the entire array \texttt{U} can be copied from the
subimage device with the statement,
\texttt{U = U[device]}, 
and memory deallocation (not shown) is similar to memory allocation.

%%\begin{verbatim}
%%   deallocate(U)
%%   if (device /= this_image()) then
%%      deallocate(U[device])   [[device]]
%%   end if
%%\end{verbatim}


%%\subsection{Execution Semantics and Memory Management}

%%This section describes the hierarchical memory layout and how memory consistency between the 3-levels of memory is maintained.  It describes what is the compiler's responsibility and what is the programmer's responsibility.

%% Completion of a do concurrent construct indicates that all executing threads (if a threading model is used by the compiler) have completed and that all memory on the executing subimage is in a consistent state.

%%The {\tt HALO} function returns a copy of the 3x3 region of {\tt a} and its surrounding neighbors.  The {\tt convolve} function is free of race conditions because of the copy semantics of {\tt HALO} and the implied synchronization of the {\tt CONCURRENT} attribute, whereby no output variables can be updated before all threads have completed execution.  In addition, while any thread may load from an extended region about \emph{its} element with the {\tt HALO} function, it may only store into its own elemental location.


%% A note from a conversation with Matt regarding memory management
%%

%The Fortran language has much tighter restrictions on aliasing than does C.
%So unless a variable has the pointer or target attribute, it cannot be aliased.
%Thus the compiler is able to aggressively optimized for memory movement between
%the CPU and accelerator.  However, because the design philosophy of coarrays
%is that memory transfer between images can be expensive, the programmer must
%explicitly transfer memory between images with explicit syntax with square
%bracket notation, i.e., $a[1] = a[2]$.  So we allow the compiler to manage
%memory within an image but require the use of {\tt halo\_exchange} for the
%transfer of halo memory between images.


%Fortran currently supports:

%1. Array syntax: e.g., C = A + B, where A, B, C are arrays.  Note that this is implicitly a loop structure, but that no loop indices need be provided.  Also the programmer need not specify where in memory these arrays reside.  Thus this high level syntax allows the compiler more freedom in both memory placement (even across distributed memory nodes) and in runtime code execution (individual array element may be computed by different hardware threads).  This is a simple example and it is a research question as to what compiler directives would be useful for memory placement and other directions to the compiler for efficient code generation.

%2. Pure procedures:  Fortran has syntax for specifying procedures that have no side effects during execution.  Specifying code that is side-effect free code is important information to provide to the compiler so that it can generate efficient multi-threaded code.

%3. Pure elemental procedures: Fortran has syntax for specifying procedures that take only scalar arguments, but may be applied across array elements.  Elemental procedures are ideal for writing code to be executed within a hardware thread.  They resemble OpenCL kernels, but are simpler because they leave all indexing up to the compiler.

%We have determined that additional syntax is needed, in addition to the three language features described above, to allow programmers the ability to express code in Fortran to be targeted for multi-threaded hardware architectures like GPUs.  This additional syntax is provided by functions that return a copy of a small region of memory surrounding an array element (as seen within an elemental procedure) and with functions for thread synchronization.  This additional syntax will allow pure procedures to perform stencil and other convolution-like operations on a copy of memory, synchronize, then store the computed results back to the array element associated with the given thread.

%In addition Fortran has syntax like the target attribute the specifies when variables can be aliased.  This allows for much easier program analysis as the compiler knows that ordinary variables cannot be aliased.  Fortran also has excellent facilities for interoperability with C so that programming in a mixed language environment is easily accomplished, including interoperability with native Fortran arrays.

\subsection{Comparison to Coarray Fortran}

LOPe provides a purely \emph{local} viewpoint; the programmer is only provided read and write access
to the local array element and write access to a small halo region surrounding the local element.
There is simply \emph{no} mechanism provided for the programmer to even know \emph{where} the local
element is in the context of the broader array.  On a distributed memory architecture, the halo
elements may not even be physically located on the same processor.  If executed on a cluster
containing hybrid processing elements (e.g. GPUs), the halo elements may be as far as three hops
away: one to get to the host processor and another two to get to memory on the hybrid processor
executing on another distributed memory node.  LOPe provides a complete separation between algorithm
development and memory management and synchronization (between memory copies of the same logical
array region covered by halos).  By explicitly describing the existance and size of an array's halo
region, the compiler is provided with enough information to manage most of the hard and detailed work
involved in memory transfer and synchronization.

Additionally, the semantics of the LOPe execution model (on which more will be said later)
remove even the \emph{possibility} of race conditions developing during execution of a concurrent
procedure.

We emphasize some of these advantages by compairing the \texttt{Laplace} implementation shown above
with the implementation of the same algorithm from the original Numrich and Reid paper describing
coarrays in Fortran.  We should point out that this comparison is somewhat unfair because
Numrich and Reid were describing the advantages of coarray notation for transferring memory
on distributed memory architectures, not how ideally to use coarrays within a major code.  But
this example serves to highlight the advantages of LOPe as described.  In the coarray example
shown below type declarations have been removed to save space:
{\small \begin{verbatim}
subroutine Laplace (nrow,ncol,U)
   left = me-1     ! me refers to the current image
   if (me == 1) left = ncol
   right = me + 1
   if (me == ncol) right = 1
   call sync_all( [left,right] ) ! Wait if left and right have
                                 ! not already reached here
   new_u(1:nrow)=new_u(1:nrow)+u(1:nrow)[left]+u(1:nrow)[right]
   call sync_all( [left,right] )
   u(1:nrow) = new_u(1:nrow) - 4.0*u(1:nrow)
end subroutine
\end{verbatim}}

The advantages of LOPe over this example are now described.  Please note the the following
is not a criticism of Coarray Fortran (CAF) as CAF is a general purpose parallel programming
language and LOPe only pertains to the halo pattern useful in stencil-based algorithms.

%%Please note that this example is somewhat unfair because in practice CAF is usually refactored in a locally oriented way by so that communication and synchronization separated into separate procedures.  Locally-oriented programming should be viewed as programming methodology with LOPe as a particular instance.
\subsubsection{LOPe advantages.}
\begin{itemize}

\item
LOPe requires that the implementation of the algorithm to be separate from the call to
effect the halo transfer.  Removing boundary condition specification (e.g., the cyclic
boundary conditions implemented in the CAF example) from the algorithm allows the boundary
conditions to be changed and without changing algorithm code.

\item
LOPe applies the transfer of halo memory across multiple (possibly) levels of memory with
the LOPe intrinsic \texttt{TRANSFER\_HALO} function (described later).  Thus the LOPe
algorithm can be run on a machine with many interconnected nodes, each containing hybrid
processor cores.  The coarray example can only be run on multiple nodes (called images in
Fortran) without accelerator cores.

\item
The algorithm implementation is separate from user-specified synchronization, e.g.,
\texttt{call sync\_all}.  In LOPe, synchronization is subsummed in the semantics of the
\texttt{CONCURRENT} attribute and the \texttt{TRANSFER\_HALO} function described later.

\item
The algorithm implementation is separate from any specification as to where the array
memory is located.  The CAF example explicity denotes where memory is located with the
\texttt{[left]} and \texttt{[right]} syntax where left and right specifiy a processor
topology.

\item
The algorithm implementation is separate from any specification as to where the algorithm
is to be executed.  The CAF example explicity denotes where a statement is to executed
with control flow construct like \texttt{if (me == 1)}.

\item
The LOPe implementation is easier to understand and frequently follows the mathematical
algorithm directly.  For example, the CAF example adds 4 neighbors plus the center value
to make the implementation with direct remote coarray access possible, while the LOPe
example is able to implement the same algorithm with fewer operations by adding 4
neighbors (not including the center array element) and then only subtracting 3 center
values.

\item
The semantics of LOPe makes explicit management of array temporaries (e.g., \texttt{U} and
\texttt{new\_U} by programmers unnecessary (though still possible).  Because in LOPe the
halo region is a language construct, the compiler is better able to manage temporary
buffers than users on the target hardware platform.

\end{itemize}

\subsubsection{Errors that are constrained by the language.}
\begin{itemize}

\item
A programmer is not able to store data to the halo region.  If this were allowed, one
thread could overwrite another threads data at undefined times.  The compiler is able to
catch this class of error.

\item
A programmer can't make indexing errors in a concurrent routine by going out of bounds of
the array plus halo memory.  The compiler is able to catch this class of error at compile
time as long as compile-time constants are used to specify halo sizes.

\item
A programmer is not able to cause race conditions by forgetting to create and
use temporary arrays properly.  In LOPe it is the comilers responsibility to
store data in temporary memory.

\item
A programmer can't make synchronization errors as synchronization is implicit in
the \texttt{CONCURRENT} attribute.  A thread running a concurrent procedure is provided
with a copy of it's array element (plus halo) that is consistent with the state
of memory at the time of invocation of procedure.  Stores to an individual
thread's local array element (by that thread) is never visible to other threads.
LOPE encourages the creation of small functions and lets the compiler
fuse the procedures together to provide the necessary synchronization.

\end{itemize}


%
% This could be part of conclusions
%

%%Note that the use of halo cells is the normal way that large and complex MPI and CAF programs are implemented.  LOPe proposes to formalize this common pattern into the Fortran language allowing the compiler access to this information in order to spread computation over more hardware resources, improve performance, and to reduce complexity for the programmer.
