\section{Coarray Fortran Extensions}

Topics highlighted in this section are: 1. Distributed memory array allocation;
2. Explicit memory placement; 3. Remote memory transfer; and 4. Remote execution
and synchronization.  This
description is within the context of extensions to Fortran; as shorthand, these extensions
are referred to as CAFe, for Coarray Fortran extensions.
Please note that throughout this paper, text appearing in all caps (such as
\texttt{GET\_SUBIMAGE}) indicates new syntax or functionality extended by CAFe.



\subsection{Subimages}

We first introduce the important new concept of a \emph{CAFe subimage}.  Fortran images
are a collection of distributed memory processes that all run the same program.  CAFe
extends the concept of a Fortran image by allowing images to be hierarchical; by this we
mean that an image \emph{may} host a subimage (or several subimages).  It is important to
note that a subimage is not visible to Fortran images other than its hosting image.
Subimages also execute differently than normal images and may even execute on different
non-homogeneous hardware, e.g., an attached accelerator device.  Subimages are task based
while images all execute a Single Program but with different (Multiple) Data (SPMD).  A
task can be given to a subimage, but execution on the subimage terminates once the task is
finished.  Memory on a subimage is permanent, however, and must be explicitly allocated
and deallocated.

A programmer requests a subimage by executing the new CAFe function,

%%\small
\begin{verbatim}
   device = GET_SUBIMAGE(device_id)
\end{verbatim}
%%\normalsize

\noindent where the integer argument represents an attached hardware device (or perhaps a separate
process or a team of threads; it is the compilers responsibility to determine precisely
what a subimage means in terms of the software and hardware environment under its
control).  If the function fails (e.g., the requested device is unavailable) it returns
the local image number of the process that is executing the current program, obtained by a
call to the Fortran 2008 function \texttt{this\_image()}.  Returning the current image if
the \texttt{GET\_SUBIMAGE} call fails, allows program execution to proceed correctly even if
there are no attached devices.

Once a variable representing a subimage has been obtained, it can be used within a program
to allocate memory, transfer memory, and execute tasks.  This functionality is described below.


\subsection{Distributed Memory Array Declaration}

In Fortran, arrays that are visible to other program images must be declared with the
codimension attribute, for example,
%%\small
\begin{verbatim}
real, allocatable :: U(:,:)[:], V(:,:)[:]
\end{verbatim}
%%\normalsize
declares that coarrays \texttt{U} and \texttt{V} are allocatable with a rank of two and a
corank of one.  In Fortran, square brackets \texttt{[ ]} are used to indicate operations
(possibly expensive) on distributed memory where parentheses \texttt{( )} are used to
select specific array elements; square brackets, on the other hand, denote the
\emph{location} of the array or array elements selected.

When declared thusly, coarrays must subsequently be allocated on all program images.  With
CAFe, this declaration also allows memory to be allocated on a subimage, however memory
allocation on a subimage is not required unless it is specifically used by some task
executing that will be executed on the subimage.


\subsection{Explicit Memory Placement}

As discussed, CAFe allows coarray memory to be conditionally allocated on a subimage.
It should be allocated conditionally because the requested subimage in a \texttt{GET\_SUBIMAGE}
call may not be available.  In addition, not all coarrays need be allocated on each
subimage.  For example, the following code segment allocates memory for coarrays
\texttt{U} and \texttt{V} on all images, but only coarray \texttt{U} is allocated on
the subimage specified by the variable \texttt{device}:
%%\small
\begin{verbatim}
   allocate(U(M,N)[*], V(M,N)[*])
   if (device /= this_image()) then
     allocate(U(M,N)[*])  [[device]]
   end if
\end{verbatim}
%%\normalsize
where \texttt{M} and \texttt{N} are constant parameters.  The first allocate statement
allocates coarrays \texttt{U} and \texttt{V} on all program images.  The \texttt{*} symbol
in the allocation is required for the last corank dimension in CAF (in this case there is
only one) to allow for the number of program images to be a runtime parameter, which can
be ascertained with a \texttt{num\_images()} function call.  If \texttt{device} is
available, its numeric value will be different from \texttt{this\_image()} and thus
coarray \texttt{U} will also be allocated on the device.  Note that placement of the
memory is specified by the use of double square bracket notation \texttt{[[ ]]}.  Like
single square brackets, double square brackets indicate something special (and possibly
expensive) is to occur, in this case possibly remote memory allocation.

Deallocation of subimage memory is similar to allocation,
%%\small
\begin{verbatim}
   if (device /= this_image()) then
     deallocate(U)  [[device]]
   end if
\end{verbatim}
%%\normalsize


\subsection{Remote Memory Transfer}

Once memory is allocated on the device, it can be initialized by copying memory from the
hosting image to the device.  This is explicitly done with normal CAF syntax.  For example,
%%\small
\begin{verbatim}
  U[device] = 3.14
\end{verbatim}
%%\normalsize
assigns 3.14 to all of the array elements of \texttt{U} located on \texttt{device}.
Specific array elements can also be transferred, 
%%\small
\begin{verbatim}
  U(1,:) = U(1,:)[device]
  U(M,:) = U(M,:)[device]
\end{verbatim}
%%\normalsize
where the first and last columns of \texttt{U} are copied from the device to the hosting image.

CAFe restricts memory transfer between a subimage and its hosting image only; transferring
memory to a subimage hosted by another image is not allowed.  In addition, memory transfer
can only be initiated by the hosting image.  Thus code executing on a subimage cannot use
coarray notation to select memory on another subimage nor on its host.

\subsection{Remote Execution}

The final CAFe concept introduced is that of remote execution.  As discussed above,
all CAF images (similar to MPI ranks) execute the same program.  The only way for
an image to execute something different is to conditionally execute a block of code based
on explicitly checking that an executing image's rank (\texttt{this\_image()}) meets some
established criterion.  There currently exists no mechanism for one image to execute a
block of code or a procedure on another image.

However, CAFe allows images to execute tasks on hosted subimages using standard procedure
calls.  For example,
%%\small
\begin{verbatim}
  call relax(U[device])  [[device]]
\end{verbatim}
%%\normalsize
executes the subroutine relax, on the subimage \texttt{device}, using coarray \texttt{U} \emph{located}
on the same subimage.  This follows the same pattern as introduced for remote coarray allocation,
whereby the double square bracket notation indicates \emph{where} the function will execute.
Functions to be executed on a subimage must be pure procedures (i.e., declared with the Fortran
keyword \texttt{pure}) and scalar formal parameters must be passed by value (i.e., declared with the
\texttt{value} attribute).

CAFe expands on CAF execution semantics by providing a task-based mechanism, where tasks
are defined in terms of pure Fortran procedures.  Since the procedures must be declared
as pure, they cannot perform any I/O or call any impure procedures; nor can they initiate
any communication between the subimage and the hosting image.

In addition, CAFe subimage tasks may also be defined by the body of a Fortran do
concurrent construct and executed on a subimage.  For example, the code segment,
%%\small
\begin{verbatim}
do concurrent(i=1:N-1)  [[device]]
 begin block
   real :: w = 0.67
   S(i) = (1-w)*T(i) + w*(T(i-1)+T(i+1)/2)
 end block
end do
\end{verbatim}
%%\normalsize
will execute all of the code within the \texttt{do concurrent} construct on \texttt{device}.
Execution of a \texttt{do concurrent} loop on a subimage acts \emph{as if} the loop were
extracted as an outlined function and the subsequent task executed on a subimage.  Thus
there is an implicit barrier at the end of the loop that blocks the hosting image
from continuing execution until \emph{all} of the iterates of the loop have finished.

In Fortran, the execution semantics of a \texttt{do concurrent} loop is that loop
iterates can be executed concurrently in any order.  Note the similarities between this
example and the OpenMP pragma \texttt{!\$omp parallel do, private(w)}.  The use of the
Fortran block construct to declare private variables allows a natural and clear way to
distinguish between variables that are private to a ``thread'' (although in the example
above, the ``private'' distinction for the variable \texttt{w} is unnecessary because it
is a constant over all loop iterations).

The comparison with OpenMP also suggests that subimage execution may not be restricted to
a single hardware execution unit, but may indeed be executed by a team of threads in a
single (shared) memory address space.  This association with a team of threads will be
exploited below when considering a team of subimages.

\begin{comment}
  (NOTE FOR CRAIG: need to read the standard to see if a block construct can be used in this
way. ALSO, there is a lot more work here to think about how do concurrent merges with
OpenMP and OpenACC with regards to synchronization, barriers, teams, atomics, , critical
sections, ...)
\end{comment}


\subsection{Integration with Standard Coarray Features}

The new CAFe features introduced above work in concert with the existing 2008 Fortran
coarray constructs and with the new parallel features to appear in the next Fortran
standard\cite{TS:18508} (referred to here as Fortran 2015).  This section describes
in more detail how the proposed extensions of CAFe interact with the parallel
language features of Fortran 2015.

\subsubsection{Fortran 2008 Standard}

Perhaps the most important of the 2008 parallel constructs are those involving the
control of image execution, including \texttt{sync all} and \texttt{sync images}
statements.  A \texttt{sync all} statement implements a barrier synchronizing all CAF
images while a \texttt{sync images} statement synchronizes a specified set of images.
Neither of these synchronization primitives affect tasks potentially executing concurrently
on a subimage.  Once subimage tasks begin execution they run to completion from the perspective
of the hosting image.

Other features of 2008 include the \texttt{critical} construct and the \texttt{lock} and
\texttt{unlock} statements.  The \texttt{critical} construct limits the execution of a
block of code to one image at a time, while lock variables provide a facility for atomic
operations.  Both of these features are defined in terms of normal CAF images so subimages
do not directly participate unless the parent (or host) of a subimage does.

\subsubsection{Fortran 2015 Standard}

The most important concept introduced by the 2015 standard is that of a team of images.
Subimages may not be included within a team of standard CAF images.  However, if one
considers the comparison of the execution of a \texttt{do concurrent} block on a subimage
with a team of OpenMP threads (as described above), the subimage could be considered as
a team of one member that \emph{may} be executing its tasks on a team of shared memory threads.
This allows the use of a \texttt{sync team} statement to synchronize with
possibly executing threads.

Thus we introduce another CAFe intrinsic function, \texttt{GET\_SUBIMAGE\_TEAM()}, that returns a
variable representing a fully-formed (established) team consisting of only one team
member, the current subimage.  If this function is called outside the context of a task
executing on a subimage it will return only a team consisting of the current image,
\texttt{this\_image()}.  Synchronization across possible subimage threads is then possible
using the statement
%%\small
\begin{verbatim}
  sync team (GET_SUBIMAGE_TEAM())
\end{verbatim}
%%\normalsize
This allows implementation of the technique of double buffering within a subimage task
so that all of the \emph{potential} subimage threads will have completed operating on the
temporary buffer before it is copied back to the primary array variable.

Fortran 2015 also introduces the concept of events.  An event is a Fortran type that
can be posted to by one image and waited for by another image.  CAFe relaxes the
constraint that events must be coarrays so that events can be used locally within
a single image to control concurrent execution of multiple subimage tasks.  For example,
the following code segment shows how two tasks can be run concurrently on two separate
subimages:
%%\small
\begin{verbatim}
 type(event_type) :: evt
 call update_ocn [[dev1, WITH_EVENT=evt]]
 call update_atm [[dev2, WITH_EVENT=evt]]
 event wait (evt, until_count=2)
\end{verbatim}
%%\normalsize

In this example, the syntax \texttt{[[dev1, WITH\_EVENT=evt]]} indicates that the called
subroutine is to be run on \texttt{dev1} and that the event variable \texttt{evt} will receive
notification (an implicit posting) once the task has completed.  The \texttt{event wait}
statement will cause execution on the hosting image to wait until \emph{both} tasks have
completed.  One may even run multiple tasks on the \emph{same} device using events,
however unless the subimage hardware allows concurrent execution of multiple tasks, only
one task will be executed at a time.

Fortran 2015 also provides new intrinsic procedures for atomic and collective operations.
At this time we do not consider atomic operations for usage with the CAFe extensions.
Atomic variables and associated synchronization constructs place too great of requirements
on the executing software and hardware environment.  For example, the Open Community
Runtime environment does not provide for atomic variables as the restriction
helps OCR to better scale on a wider range of parallel computers.

Collective operations, however, could potentially be very useful and fit nicely within
CAFe concepts.  For example, a team consisting of hosted subimages could be formed and
then data be broadcasted to team members and reduction operations such as \texttt{co\_add}
performed.  However since all team members of the current team must participate in
collective operations \emph{and} since CAFe restricts subimages from being part of the
same team with regular images, it makes it impossible for an image to participate with a
team of its hosted subimages in collective operations.  So for now, collectives are
excluded from CAFe.
