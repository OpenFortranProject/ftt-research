\section{Conclusions}

Fortran is a general-purpose programming language and is extremely important to the
scientific and high performance computing (HPC) community.  While new code development is
often in languages other than Fortran (such as C++), surveys show that Fortran usage is
still high\cite{prabhu2011survey}.  Fortran will be with us well into the future as the
lifetime of important scientific applications is measured in decades and Fortran usage
dominates in scientific communities that are critical to understanding the future of
life on our planet such as the climate community.

However, significant challenges arise as hardware adapts to the end of Moore's
Law\cite{exascale:workshop:2011}.  The revolution occurring in hardware architecture
has completely altered the landscape for scientific computing as a code designed for the
distributed memory bulk synchronous model of parallelism may actually run slower on
advanced hardware\cite{Dubey:2014:SSC:2686745.2686756}.

Parallelism was added to Fortran with the addition of coarrays in the 2008
standard\cite{fortran:2008} to enable the migration of codes as hardware evolves.
Significant language features were left out of this early version of coarrays as
identified by \emph{Mellor-Crummy et al}\cite{mellor-crummey:2009:caf2}.  Many of these features
have been adopted in what is likely to be called the 2015 standard of the language\cite{fortran:2015},
such as teams of program images and improved synchronization constructs like events.

%%
%% MJS - this stuff is so familiar these days, no need to cite specific machine examples
%%
%, for example, in the early
%RoadRunner Machine at Los Alamos\cite{RoadRunner} and new machines arriving such as Blue
%Waters\cite{blue:waters}.

However, the Fortran coarray model fails to address heterogeneity in hardware as is already
being seen in HPC computing platforms with the adoption of distributed nodes consisting of
general purpose CPUs with attached accelerator devices like GPUs.  To address this shortcoming
of the coarray model, CAFe introduces the concept of subimages executing on a hosting image.  
CAFe includes:

\begin{itemize}
\item
  Dynamic creation of subimages.
\item
  Dynamic memory allocation and placement on a subimage.
\item
  Explicit addressing of memory on subimages using standard coarray notation.  It follows
  the Partitioned Global Address Space (PGAS) model, although with subimages the address
  space is hierarchical in the sense that the memory partitioning only allows memory exchange
  between a subimage and its hosting image as initiated by the hosting image.
\item
  Task creation and execution on subimages using extended coarray syntax with double
  square brackets \texttt{[[ ]]}.  Task mays be defined in terms of standard \emph{pure}
  Fortran procedures and task execution continues without interruption until completion.
  In addition, a Fortran 2015 event type may be notified upon completion of the task to allow
  concurrent execution on the subimage and on the hosting image.
\item
  CAFe integrates Fortran 2015 features to allow relatively simple programs to be written
  that employ all of the heterogenous components expected on exascale platforms.
\end{itemize}

\subsection{Limitations}

In this paper we have defined CAFe and described its semantics and relationship to the 2008 and
2015 Fortran standards.  Possible criticism of this work includes the following:

\begin{itemize}
\item
  The source-to-source transformations that have been developed to implement CAFe
  are rudimentary and have not been optimized.  For example, memory transfers have
  not been aggregated into larger message blocks to improve performance.  Memory transfer
  is also blocking so that prefetching of data cannot be employed via techniques
  involving the reordering of statements (code motion) that are effectively employed in
  CAF compilers.
\item
  The Laplacian example is in reality a ``toy'' problem.  Real multiphysics codes
  employ hundreds (or more) of variables and would have entirely different performance
  characteristics because of the increased pressure on cache and register usage.
%\item
%  Existing scientific codes often separate communication into separate modules from
%  the computational kernels.  This makes adoption of CAF itself slow (because these
%  MPI modules already work fine) and thus additional extensions like CAFe also
%  would be slow to adopt. However CAFe constructs can be used in an incremental
%  fashion making it easier to adopt
%\item
%  In reality CAF (and therefore CAFe) are relatively low-level parallel programming
%  languages that require the programmer to \emph{explicitly} control the placement
%  of memory, communication, and synchronization.  
\item
  Thorough examination of performance capabilities on real scientific applications
  (or at least scientific ``mini-apps'') has not been attempted.  Such a study of both
  CAFe and CAF itself is necessary to demonstrate the coarray model as a viable alternative
  for application codes with complexity above that of an idealized mini-app.
\item
  Programming models like that proposed by CAFe already exist in the form of a combination
  of MPI and OpenMP (or OpenACC) programs.  The advantage of CAFe --- by being a purely
  language construct --- is that it opens up the possibility of compiler optimizations
  that would otherwise be impossible with a library approach intermixed with compiler directives.
\end{itemize}

\subsection{Future Work}

In spite of the limitations discussed, 
adoption of the task-based parallelism of CAFe and the synchronization
event types in Fortran 2015 suggests intriguing possibilities.  Tasks plus the
dependencies represented by events suggest a correspondence with the Open Community
Runtime (OCR) libraries.  In OCR, the entire program must be broken into a set of tasks
represented by a directed acyclic graph (DAG).  In OCR, tasks are available for executing
when all of their input dependencies are satisfied, such as occurs when computation of a new
iteration of data elements have been updated and other upstream tasks have been completed.

%%%
%%% MJS: I cut the following out - conversion of MPI -> OCR is not in scope for us, and 
%%% we don't want to argue one way or the other about that topic.  Stay focused on CAF/CAFe.
%%%
%Tools to automatically convert existing applications to run under the OCR model faces challenges.  For
%example, the hundreds of MPI functions would have to be semantically understood by the tools,
%integrated with existing OpenACC directives (for example), a DAG generated for the complete
%program, and then transformed into an OCR program.

An intriguing \emph{possibility} exists for automatic conversion of CAFe codes to OCR:
\begin{itemize}
\item
  CAF codes have well defined boundaries (called segment boundaries) indicating where new
  OCR tasks must be created.
\item
  CAF has many fewer library routines (primarily collectives) that must be converted.
\item
  CAF has events that already define some of the dependencies necessary for the creation
  of OCR tasks.
\item
  CAFe introduces the notion of task based parallelism (in addition to the existing
  Fortran constructs like \texttt{do concurrent}).
\end{itemize}

In the future we hope to examine \emph{if} the possibility exists for CAFe codes to be
automatically converted to an entirely different programming paradigm like
the Open Community Runtime.  Given the change in architectures being considered for
future HPC platforms it is likely that similar, novel runtime models will be proposed,
and we believe that CAFe provides a viable programming model for compiler-based mapping to
these new systems.

\begin{comment}
These choices often hide the opportunity for
optimizations by the compiler \cite{Dubey:2014:SSC:2686745.2686756}.
\end{comment}

\begin{comment}
\subsubsection{Limitations.}
LOPe only supports regular structured grids through Fortran multi-dimensional arrays and a
corresponding multi-dimensional processor layout.  It does not allow the composability of stencils
required by non-linear physics operators, nor does it provide automatic support for the storage of
intermediate results resulting from multiple intermediate update steps.  Adaptive Mesh Refinement
(AMR) Shift Calculus provides a generalized abstraction that addresses many of these concerns
(see\cite{Dubey:2014:SSC:2686745.2686756} and references therein).  However, it may be possible to
support AMR in LOPe through locally-structured grid methods based on the work of Berger and
Oliger\cite{colella2007performance}.  In this instance, LOPe could be used to update the regular
array regions in each rectangular patch.
\end{comment}
