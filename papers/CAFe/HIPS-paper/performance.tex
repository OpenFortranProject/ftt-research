\section{Performance Measurements}

It is important to note that the emphasis of this work is to explore and explain the new
syntax and execution semantics of a CAFe application.  A thorough examination of potential
performance gains (if any) using CAFe for parallelization of code is beyond the scope of
this paper.  The primary purpose of CAFe is to combine the parallel features of Fortran coarrays
--- executing a \emph{single} program --- with concurrent execution of separate tasks executing on
potentially heterogeneous hardware.  However, the relative performance of computation on
a cluster of GPUs compared with the necessary communication of halo information is
of interest; especially considering complete exchance of halo data between multiple
coarray images \emph{and} between each individual host image and its subimage (an attached
OpenCL device).

%%However, preliminary work using the Stratego/XT rewrite system as described indicates that similar performance can be expected to that of previous work\cite{foropencl} (see also\cite{StencilCompilers}).

This section shows and compares performance measurements for: 1. Memory bandwidth usage on
GPU; 2. Computaional time on GPU; 3. Halo Communication time (total plus breakdown to and
from device and MPI); 4. Total time scaling on ACISS; 5. Comparison with OpenACC
(hopefully).

\subsection{Halo Transfer Performance}

The current implementation does \emph{not} take advantage of optimization strategies such
as prefetching of array tiles (including halos) into OpenCL local memory or of loop
unrolling on the OpenCL device.

%%either does it take advantage of the potential of CAFe to overlap communication with computation (for example, computing on 

\begin{comment}
Since many scientific codes are dominated by memory performance, including and especially
stencil algorithms as they typically only involve a computation on a small locally central
array element and a small overlapping halo region.  Stencil operations frequently do not
contain enough floating point operations per memory load to allow for floating point
performance to operate at peak (though this is entirely application and domain specific).
Thus we illustrate the \emph{potential} for performance by noting the latency and
throughput performance of an attached GPU in conjunction with MPI distributed memory
performance associated with halo transfer in Table 1.
\end{comment}

\begin{comment}
The results in Table 1 indicate that the primary bottleneck in using accelerators attached
to the TODO bus using OpenCL is likely to be the latency in transferring memory to and
from the device for distributed memory clusters of only a few nodes exchanging halo data
using MPI.
\end{comment}

