\section{Shallow Water Model}
\label{sec:shallow-water}

The numerical code used for this work is from a presentation at the
NM Supercomputing Challenge~\cite{Robey07}.
The algorithm solves the standard 2D shallow water equations. This
algorithm is typical of a wide range of modeling equations based on
conservation laws such as compressible fluid dynamics (CFD), elastic
material waves, acoustics, electromagnetic waves and even traffic
flow~\cite{Leveque02}. For the shallow water problem there are
three equations with one based on conservation of mass and the other
two on conservation of momentum.

\begin{eqnarray*}
h_{t}+(hu)_{x}+(hv)_{y} & = & 0\quad\mbox{(mass)}\\
(hu)_{t}+(h{u}^{2}+\tfrac{1}{2}gh^{2})_{x}+(huv)_{y} & = & 0\mbox{\quad($x$-momentum)}\\
(hv)_{t}+(huv)_{x}+(h{v}^{2}+\tfrac{1}{2}gh^{2})_{y} & = & 0\mbox{\quad($y$-momentum)}
\end{eqnarray*}


%h_{t}+(hu)_{x}+(hv)_{y} & = & 0\qquad\mbox{(conservation of mass)}\\
%(hu)_{t}+(h{u}^{2}+\tfrac{1}{2}gh^{2})_{x}+(huv)_{y} & = & 0\mbox{\qquad(conservation of x-momentum)}\\
%(hv)_{t}+(huv)_{x}+(h{v}^{2}+\tfrac{1}{2}gh^{2})_{y} & = & 0\mbox{\qquad(conservation of y-momentum)}\end{eqnarray*}

\noindent
where h = height of water column (mass), $u$ = x velocity, $v$ =
y velocity, and $g$ = gravity. The height $h$ can be used for mass
because of the simplification of a unit cell size and a uniform water
density. Another simplifying assumption is that the water depth is
small in comparison to length and width and so velocities in the z-direction
can be ignored. A fixed time step is used for simplicity though it
must be less than $dt \leqq dx / (\sqrt{gh}+|u|)$ to fulfill the CFL
condition.

The numerical method is a two-step Lax-Wendroff scheme. The method
has some numerical oscillations with sharp gradients but is adequate
for simulating smooth shallow-water flows. In the following explanation,
$U$ is the conserved state variable at the center of the cell . This
state variable, $U=(h,hu,hv)$ in the first term in the equations
above. $F$ is the flux quantity that crosses the boundary of the cell
and is subtracted from one cell and added to the other. The remaining
terms after the first term are the flux terms in the equations above
with one term for the flux in the x-direction and the next term for
the flux in the y-direction. The first step estimates the values a
half-step advanced in time and space on each face, using loops on
the faces.\begin{eqnarray*}
U_{i+\frac{1}{2},j}^{n+\frac{1}{2}} & = & (U_{i+1,j}^{n}+U_{i,j}^{n})/2+\frac{\triangle t}{2\triangle x}\left(F_{i+1,j}^{n}-F_{i,j}^{n}\right)\\
U_{i,j+\frac{1}{2}}^{n+\frac{1}{2}} & = & (U_{i,j+1}^{n}+U_{i,j}^{n})/2+\frac{\triangle t}{2\triangle y}\left(F_{i,j+1}^{n}-F_{i,j}^{n}\right)\end{eqnarray*}


The second step uses the estimated values from step 1 to compute the
values at the next time step in a dimensionally unsplit loop.\[
U_{i,j}^{n+1}=U_{i,j}^{n}-\frac{\triangle t}{\triangle x}(F_{i+\frac{1}{2},j}^{n+\frac{1}{2}}-F_{i-\frac{1}{2},j}^{n+\frac{1}{2}})-\frac{\triangle t}{\triangle y}(F_{i,j+\frac{1}{2}}^{n+\frac{1}{2}}-F_{i,j-\frac{1}{2}}^{n+\frac{1}{2}})\]


\subsection{Serial code examples}

The coding style to implement the serial version of the shallow water model
is now shown.  The Fortran kernel procedure {\tt wave\_advance} 
is declared as:

{\small
\begin{verbatim}
pure subroutine wave_advance(H,U,V,dx,dy,dt)
  real, dimension(:,:),intent(inout):: H,U,V
  real, intent(in) :: dx,dy,dt
  !$OFP CONTIGUOUS :: H,U,V
  !$OFP KERNEL :: wave_advance
end subroutine
\end{verbatim}
}

\noindent
where {\tt H, U}, and {\tt V} are state variables for height and
$x$ and $y$ momentum respectively.  The \emph{OFP} compiler directives
{\tt CONTIGUOUS} and {\tt KERNEL} are added because contiguous is Fortran
2008 syntax (not in current compilers) and kernel is an \emph{OFP} extension.

Temporary arrays are required for the interior copies {\tt iH, iU}, and
{\tt iV} of the state variables and for the quantities {\tt Hx, Hy, Ux,
Vx}, and {\tt Vy} defined on cell faces.  These temporary arrays are declared as,

{\small
\begin{verbatim}
real, allocatable, dimension(:,:) :: iH,iU,iV
real, allocatable, dimension(:,:) :: Hx,Hy,Ux
real, allocatable, dimension(:,:) :: Uy,Vx,Vy
\end{verbatim}
}

Halo variables for the interior and cell faces are

{\small
\begin{verbatim}
integer, dimension(4) :: halo,face_lt,face_rt
integer, dimension(4) :: face_up,face_dn
halo    = [1,1,1,1]
face_lt = [0,1,1,1]; face_rt = [1,0,1,1]
face_dn = [1,1,0,1]; face_up = [1,1,1,0]
\end{verbatim}
}

\noindent
Note that the halos for the four faces each have a 0 in the
definition. Thus the returned array copy will have size that is larger
than the interior regions that use {\tt [1,1,1,1]}.  This is because
there is one more cell face quantity than there are cells in a given
direction.

The first Lax-Wendroff step updates state variables on the cell
faces, for example,

{\small
\begin{verbatim}
Hx = 0.5*(region(H,face_lt)+ &
          region(H,face_rt)) &
   + (0.5*dt/dx) &
   * (region(U,face_lt)-region(U,face_rt))
\end{verbatim}
}

\noindent
updates the $x$-direction height variable.  The second step then uses
these face variables to update the interior region, for example,

{\small
\begin{verbatim}
face_lt = [0,1,0,0];  face_rt = [1,0,0,0]
face_dn = [0,0,0,1];  face_up = [0,0,1,0]
iH = iH + (dt/dx) * ( region(Ux, face_lt) - &
                      region(Ux, face_rt) ) &
        + (dt/dy) * ( region(Vy, face_dn) - &
                      region(Vy, face_up) )
\end{verbatim}
}

\noindent
Note that face halos have been redefined so that the array copy
returned has the same size as the interior region.

These simple code segments show how the shallow water model is
implemented in standard Fortran (2003) using the data-parallel
programming model described above.  The resulting code is simple,
concise, and easy to understand.  However it does \emph{not}
necessarily perform well because of the temporary array variables,
especially those produced by the {\tt region} function.  This is
generally true of algorithms that use Fortran shift functions as well,
as some Fortran compilers (e.g., gfortran) do not generate optimal
code for shifts.  We note (as shown below) that these temporary array
copies are replaced by scalars in the transformed Fortran code so
there is no performance penalty for using data-parallel statements as
outlined.
