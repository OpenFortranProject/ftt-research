\section{Introduction}
\label{sec:intro}

This paper presents a compiler-level approach for targeting a single program to
multiple, and possibly fundamentally different, processor architectures.  This
technique allows the application programmer to adopt a single, high-level
programming model without sacrificing performance.  We suggest that existing
data-parallel features in Fortran are well suited to applying automatic
transformations that generate code specifically tuned for different hardware
architectures using low-level programming models such as OpenCL.  For algorithms
that can be easily expressed in terms of whole array, data-parallel operations,
writing code in Fortran and transforming it automatically to specific low-level
implementations removes the burden of creating and maintaining multiple versions
of architecture specific code.

The peak performance of these newer accelerator architectures can be substantial.
Intel expects a teraflop for the SGEMM benchmark with their Knights Ferry
processor while the performance of the M2090 NVIDIA Tesla processor is in the
same neighborhood~\cite{hpcwire11manycore}.  Unfortunately the performance that
many of the new accelerator architectures offer comes at a cost.  Architectural
changes are trending toward multiple heterogeneous cores and less of a reliance
on superscalar instruction level parallelism and hardware managed memory
hierarchies (such as traditional caches).  These changes place a heavy burden on
application programmers as they are attempt to adapt to these new systems.  An
especially challenging problem is not only how to program to these new
architectures --- considering the massive scale of concurrency available --- but
also how to design programs that are portable across the changing landscape of
computer architectures.  How does a programmer write one program that can
perform well on both a conventional multicore CPU \emph{and} a GPU (or another
of the emerging many-core architectures)?

A directive-based approach, such as OpenMP or the Accelerator programming model
from the Portland Group~\cite{pgi10accelerator}, is one solution to this
problem.  However, in this paper we take a somewhat different approach.  A
common theme amongst the new processors is the emphasis on data-parallel
programming.  This model is well suited to architectures that are
based on either vector processing or massively parallel collections of simple
cores.  The recent CUDA and OpenCL programming languages are are intended to
support this programming model.

The problem with OpenCL and CUDA is that they expose too much detail about the
machine architecture to the programmer~\cite{wolfe08gpgpu}.  The programmer is
responsible for explicitly managing memory (including the staging of data back
and forth between the host CPU and the accelerator device) and specifically
taking into account architectural differences (such as whether the architecture
contains vector units).  While these languages have been attractive as a method
for early adopters to utilize these new architectures, they are less attractive
to programmers who do not have the time or resources to manually port their code
to every new architecture and programming model that emerges.

\subsection{Approach}

We demonstrate that a subset of Fortran map surprisingly well onto GPUs when
transformed to OpenCL kernels.  This data-parallel subset includes: array syntax
using assignment statements and binary operators, array constructs like {\tt
  WHERE}, and the use of pure and elemental functions.  In addition, we provide
new functions that explicitly take advantage of the stencil geometry of the
problem domain we consider.  Note that this subset of the Fortran language is
implicitly parallel.  This programming model \emph{does not require explicit
  declaration of parallelism within the program.}  In addition, programs are
expressed using entirely standard Fortran so it can be compiled for and executed
on a single core without concurrency.

Transformations are supplied that provide a mechanism for converting Fortran
procedures written in the Fortran subset described in this paper to OpenCL
kernels.  We use the ROSE compiler
infrastructure\footnote{\url{http://www.rosecompiler.org/}} to develop these
transformations.  ROSE uses the Open Fortran
Parser\footnote{\url{http://fortran-parser.sf.net/}} to parse Fortran 2008
syntax and can generate C-based OpenCL.  Since ROSE's intermediate
representation (IR) was constructed to represent multiple languages, it is
relatively straightforward to transform high-level Fortran IR nodes to C OpenCL
nodes.  This work is also applicable to transformations to vendor-specific
languages, similar to OpenCL, such as the NVIDIA CUDA language.

Transformations for arbitrary Fortran procedures are not attempted.
Furthermore, a mechanism to transform the calling site to automatically invoke
OpenCL kernels is not provided at this time.  While it is possible to accomplish
this task within ROSE, it is considered outside the scope of this paper.
However, ForOpenCL provides via Fortran interfaces a mechanism to call the C OpenCL
runtime and enable Fortran programmers to access OpenCL kernels generated by the
supplied transformations.

We study the automatic transformations for an application example that is
typical of stencil codes that update array elements according to a fixed
pattern.  Stencil codes are often employed in applications based on
finite-difference or finite-volume methods in computational fluid dynamics
(CFD).  The example described later in this paper is a simple shallow-water
model in two dimensions using finite volume methods.

Finally, we examine the performance of the Fortran data-parallel abstraction
when transformed to OpenCL to run on GPU architectures.  The performance of
automatically transformed code is compared with a hand-optimized OpenCL
version of the shallow-water code.
