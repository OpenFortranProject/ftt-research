
%%%%%%%%%%%%%%%%%%%%%%% file typeinst.tex %%%%%%%%%%%%%%%%%%%%%%%%%
%
% This is the LaTeX source for the instructions to authors using
% the LaTeX document class 'llncs.cls' for contributions to
% the Lecture Notes in Computer Sciences series.
% http://www.springer.com/lncs       Springer Heidelberg 2006/05/04
%
% It may be used as a template for your own input - copy it
% to a new file with a new name and use it as the basis
% for your article.
%
% NB: the document class 'llncs' has its own and detailed documentation, see
% ftp://ftp.springer.de/data/pubftp/pub/tex/latex/llncs/latex2e/llncsdoc.pdf
%
%%%%%%%%%%%%%%%%%%%%%%%%%%%%%%%%%%%%%%%%%%%%%%%%%%%%%%%%%%%%%%%%%%%


\documentclass[runningheads,a4paper]{llncs}

\usepackage{amssymb}
\setcounter{tocdepth}{3}
\usepackage{graphicx}

\usepackage{url}
\urldef{\mailsa}\path|rasmus@uoregon.edu, mjsottile@computer.org,|
\urldef{\mailsb}\path|dannagle@verizon.net, soren.rasmussen@aggiemail.usu.edu|    
\newcommand{\keywords}[1]{\par\addvspace\baselineskip
\noindent\keywordname\enspace\ignorespaces#1}

\begin{document}

\mainmatter  % start of an individual contribution

% first the title is needed
\title{Locally-Oriented Programming:\\ A Simple Programming Model for Stencil-Based Computations
       on Multi-Level Distributed Memory Architectures}

% a short form should be given in case it is too long for the running head
\titlerunning{Locally-Oriented Programming}

\author{Craig Rasmussen%
\thanks{The authors would like to thank Robert Robey and Wayne Weseloh
at Los Alamos National Laborabory for several stimulating
conversations with respect to programming models.}%
\and Matthew Sottile\and Dan Nagle\and Soren Rasmussen}

\authorrunning{Locally-Oriented Programming}
% (feature abused for this document to repeat the title also on left hand pages)

% the affiliations are given next; don't give your e-mail address
% unless you accept that it will be published
\institute{441 McKenzie Hall,\\
5246 University of Oregon, Eugene, Oregon, USA\\
\mailsa\\
\mailsb\\
\url{https://openfortran.uoregon.edu}}

%
% NB: a more complex sample for affiliations and the mapping to the
% corresponding authors can be found in the file "llncs.dem"
% (search for the string "\mainmatter" where a contribution starts).
% "llncs.dem" accompanies the document class "llncs.cls".
%

%\toctitle{Lecture Notes in Computer Science}
%\tocauthor{Authors' Instructions}
\maketitle


\begin{abstract}

Emerging hybrid accelerator architectures for high performance computing are often well suited for
usage of a data-parallel programming model.  However, programmers for these emerging architectures
face a steep learning curve that frequently requires learning a new language, e.g., OpenCL or CUDA,
or employing OpenMP compiler directives.  Furthermore, the distributed (and frequently multi-level)
nature of the memory organization of clusters of these architectures provides an additional level of
complexity for the programmer.  This paper presents preliminary work examining how programming with
a local orientation can be employed to provide simple access to accelerator architectures.  A
locally-oriented programming model is especially useful for stencil-based codes or algorithms
requiring the application of convolution kernels.  In this programming model, a programmer codes the
algorithm by modifying \emph{only a single array element}, but has read-only access to a small
sub-array surrounding the given array element, so that a stencil can be applied.  We demonstrate how
a locally-oriented programming model can be adopted as an embedded, domain-specific language using
source-to-source program transformations.


\keywords{domain specific language, stencil compilers, distributed memory parallelism}
\end{abstract}

\section{Motivation}

The concepts in this paper were motivated by problems encountered during
the development of PetaVision, a C++ software framework designed for simulating
large networks of spiking neurons in the visual cortex of primates.  PetaVision
was designed to run on massively parallel hardware architectures and a primitive
version of PetaVision achieved over a Petaflop of sustained single precision
performance on Roadrunner, the fastest computer in the world for a brief
period of time.

TODO: Describe neural layers, spiking neurons, synaptic connections, weights,
and plasticity. Cite Heubel.

TODO: If space allows, give a simple 1D example of connections within a neural network in
Figure 1.  Describe the size of visual cortex in terms of the number of neurons,
the number of synaptic connections, and performance.  Reference SC Gorden Bell
paper from IBM.

TODO: Explain difficulties encountered.

\section{Programming Model}

The LOPe programming model
restricts the programmer to a local view of the index
space of an array.  Within a LOPe function, only a single array
element (called the local element) is mutable.  In addition, a small
halo region surrounding the local element is visible to the
programmer, but this region is immutable.  Restricting the programmer
to a local index space serves to reduce complexity by separating all
data- and task-decomposition concerns from the implementation of the
element-level array calculations.

%%This reduction in complexity reduces programming errors.  While developing the convolution example described later in the paper we made an indexing error in applying the 2D stencil loops in the standard serial Fortran test implementation.  This error required over 3 hours of programming time to repair.  First the error had to be isolated to the function implementing the convolution (it was first thought to be in the complicated tiff image output routine as the convolution code was ``thought'' to be too simple to wrong.  Then the index error has to be understood.  As will be seen, LOPe makes it more difficult to make these errors as Fortran array intrinsics can be used.  In some instance, the restricted semantics of LOPe allows the compiler to catch errors (e.g., some errors involving race conditions).

LOPe is a domain specific language (DSL) implemented as a small extension to the Fortran 2008
standard.  Fortran was chosen as the base language for LOPe because it provides a rich array-based
syntax.  Although, in principle, the same techniques could be applied to languages such as C or C++.

%%Readers who are unfamiliar with Fortran syntax may wish to consult Appendix A, for a brief description of Fortran notation.

\subsection{Related work}

LOPe builds upon prior work studying how to map Fortran to accelerator
programming models like OpenCL.  In the ForOpenCL project~\cite{Sottile:2013:FTE:2441516.2441520}
we exploited Fortran's pure and elemental functions to express
data-parallel kernels of an OpenCL-based program.  In practice, array
calculations for a given index $i,j$ will require read-only access to
a local neighborhood of size $[M,N]$ around $i,j$.  LOPe extends this work by introducing
a mechanism for representing these neighborhoods as array declaration
type annotations.

ForOpenCL was based on concepts explored in the ZPL programming language~\cite{chamberlain04zpl} in
which the programmer can define regions and operators that are applied over the index sets
corresponding to the sub-array regions.  This approach is quite powerful for
compilation purposes since it provides a clean decoupling of the operators applied over an array
from the decomposition of that array over a potentially complex distributed memory hierarchy.
However, unlike the ZPL operations on entire sub-arrays, LOPe expresses operations based on
a \emph{single} local array-element.

%
% For space reasons a ZPL example is not shown and text reworded accordingly
%
%%For example, the following ZPL code implements the same stencil as the LOPe code in Fig. 1:

%%\begin{verbatim}
%%ZPL jacobi example here.
%%\end{verbatim}
%%This approach as demonstrated by ZPL is quite powerful for compilation purposes since it provides a clean decoupling of the operators applied over an array from the decomposition of that array over a potentially complex distributed memory hierarchy.  

\subsection{LOPe Syntax Extensions}

There are only a few syntax additions required for a LOPe program.
These additions include syntax for describing halo regions and
concurrent procedures.  In code examples that follow,
language additions are highlighted by the usage of capitalization for
keywords that are either new or that acquire new usage.

\subsubsection{Halo regions.}
The principle semantic element of LOPe is the concept of a halo.
A halo is an ``artificial'' or ``virtual'' region surrounding
an array that contains boundary-value information.  Halo (also called
ghost-cell) regions are commonly employed to unify array indexing
schemes in the vicinity of an array boundary so that an array may be
referenced using indices that fall ``outside'' of the logical domain
of the array.  In LOPe, the halo region is given explicit syntax so
that the compiler can exploit this information for purposes of memory
allocation, data replication and thread synchronization.  For example,
a halo region can be declared with a statement of the form,

\begin{verbatim}
  real, allocatable, dimension(:), HALO(1:*:1) :: A
\end{verbatim}
This statement indicates that \texttt{A} is a rank one array, will be
allocated later, and
has a halo region of one element surrounding the array on either side.
The halo notation \texttt{M:*:N} specifies a halo
of \texttt{M} elements to the left, \texttt{N} elements to the
right, and an arbitrary number of ``interior'' array elements.
When used to describe a formal parameter of a
function, such as the type-declaration statement, \texttt{real, HALO(:,:) :: U},
the halo size is inferred by the compiler
from the actual array argument provided at the
calling site of the function.

%%In LOPe, there is no need (in this instance) for a repetitive \texttt{dimension(:,:)} specification, as it is inferred from the \texttt{HALO} specification.

\subsubsection{Concurrent functions.}

The second keyword employed by LOPe is \texttt{concurrent} which already exists in the form of a
\texttt{do} \texttt{concurrent} loop, whereby the programmer asserts that specific
iterations of the loop body may be executed by the compiler in \emph{any order,} even
\emph{concurrently.}  LOPe allows a function with the attributes \texttt{pure} (assertion of no
side effects) and \texttt{concurrent} (assertion of no dependencies between iterations) to be
called from within a \texttt{do} \texttt{concurrent} loop.  An example of a LOPe function is
shown in Fig. 1 and an example calling this function will be provided later in the text.  One
should imagine that a LOPe function is called \emph{for each} \texttt{i,j} index of the interior of
the array \texttt{U}.  Note that this usage introduces a race condition as new values of
elements of \texttt{U} are created on the left-hand side of the assignment statement that may use
\emph{new or old} values of \texttt{U} on the right-hand side.  LOPe requires the compiler to
guarantee that race conditions won't occur by using, e.g., double-buffering techniques as needed.

\vspace{-.1in}

\begin{figure}
\begin{verbatim}
           pure CONCURRENT subroutine Laplacian(U)}
               real, HALO(:,:) :: U
               U(0,0) =                 U(0,+1)              &
                        +  U(-1,0)  - 3*U(0, 0)  +  U(+1,0)  &
                                    +   U(0,-1)
           end subroutine Laplacian
\end{verbatim}
\vspace{-.1in}
\caption{A LOPe function implementing a Laplacian kernel in two dimensions.}
\end{figure}

\vspace{-.3in}

\subsubsection{LOPe index notation.}

In the \texttt{Laplacian} example the \texttt{U(0,0)} array element is the \emph{local} array
element and only the local element may be modified.  This zero-based indexing for the local-array
element differs from conventional Fortran, where by default, array indices start at 1.  The use of
zero-based indexing gives a clean symmetry for indices on either side of the central element at
zero. The other array elements are in the halo region and are \texttt{U(-1,0)} and
\texttt{U(+1,0)} (left and right of local, respectively) and \texttt{U(0,-1)} and \texttt{U(0,+1)}
(below and above of local).  The geometric positioning of the array elements can be
seen by examining the arrangement of the expressions on the right-hand side of Fig. 1.

\section*{Appendix: A Fortran Primer for the C Programmer}

This section provides a brief description of Fortran syntax for readers unfamiliar with Fortran
syntax.  The most common source of confusion is array notation.  Fortran uses parentheses
\texttt{()} for both functions arguments and array indexing; thus the confusing expression
\texttt{foo(i,j)} can either be an array reference or a function call depending on context.  Where C
reserves square brackets \texttt{[]} to denote an array reference, Fortran uses square brackets to
reference coarray memory on a remote, distributed memory process (called an image in Fortran).

Otherwise, Fortran syntax is usually straightforward though it may appear strange to the uninitiated
programmer.  A subroutine is a function with no return value and must be called in a statement
rather than as an expression.  Simple syntactic elements are: \texttt{!} the line comment symbol;
\texttt{\&} the line continuation symbol; and \texttt{::} is syntactic sugar used for the separation
of syntactic elements, e.g., separating type attributes from the variable list in a type declaration
statement.


\subsection{Acknowledgments}
This work was supported in part by the Department of Energy Office of Science,
Advanced Scientific Computing Research.

\end{document}
