
%%%%%%%%%%%%%%%%%%%%%%% file typeinst.tex %%%%%%%%%%%%%%%%%%%%%%%%%
%
% This is the LaTeX source for the instructions to authors using
% the LaTeX document class 'llncs.cls' for contributions to
% the Lecture Notes in Computer Sciences series.
% http://www.springer.com/lncs       Springer Heidelberg 2006/05/04
%
% It may be used as a template for your own input - copy it
% to a new file with a new name and use it as the basis
% for your article.
%
% NB: the document class 'llncs' has its own and detailed documentation, see
% ftp://ftp.springer.de/data/pubftp/pub/tex/latex/llncs/latex2e/llncsdoc.pdf
%
%%%%%%%%%%%%%%%%%%%%%%%%%%%%%%%%%%%%%%%%%%%%%%%%%%%%%%%%%%%%%%%%%%%


\documentclass[runningheads,a4paper]{llncs}

\usepackage{amssymb}
\setcounter{tocdepth}{3}
\usepackage{graphicx}

\usepackage{url}
\urldef{\mailsa}\path|rasmus@uoregon.edu, mjsottile@computer.org,|
\urldef{\mailsb}\path|dannagle@verizon.net, soren.rasmussen@aggiemail.usu.edu|    
\newcommand{\keywords}[1]{\par\addvspace\baselineskip
\noindent\keywordname\enspace\ignorespaces#1}

\begin{document}

\mainmatter  % start of an individual contribution

% first the title is needed
\title{Locally-Oriented Programming:\\ A Simple Programming Model for Stencil-Based Computations
       on Multi-Level Distributed Memory Architectures}

% a short form should be given in case it is too long for the running head
\titlerunning{Locally-Oriented Programming}

\author{Craig Rasmussen
\and Matthew Sottile\and Dan Nagle\and Soren Rasmussen}

\authorrunning{Locally-Oriented Programming}
% (feature abused for this document to repeat the title also on left hand pages)

% the affiliations are given next; don't give your e-mail address
% unless you accept that it will be published
\institute{441 McKenzie Hall,\\
5246 University of Oregon, Eugene, Oregon, USA\\
\mailsa\\
\mailsb\\
\url{https://openfortran.uoregon.edu}}

%
% NB: a more complex sample for affiliations and the mapping to the
% corresponding authors can be found in the file "llncs.dem"
% (search for the string "\mainmatter" where a contribution starts).
% "llncs.dem" accompanies the document class "llncs.cls".
%

%\toctitle{Lecture Notes in Computer Science}
%\tocauthor{Authors' Instructions}
\maketitle


\begin{abstract}

Emerging hybrid accelerator architectures for high performance computing are often suited for the
use of a data-parallel programming model.  Unfortunately, programmers for these emerging
architectures face a steep learning curve that frequently requires learning a new language, e.g.,
OpenCL or CUDA, or employing OpenMP compiler directives.  Furthermore, the distributed (and
frequently multi-level) nature of the memory organization of clusters of these architectures
provides an additional level of complexity for the programmer.  This paper presents preliminary work
examining how programming with a local orientation can be employed to provide simple access to
accelerator architectures.  A locally-oriented programming model is especially useful for the
solution of algorithms requiring the application of a convolution kernel.  In this programming
model, a programmer codes the algorithm by modifying \emph{only a single array element} (called the
local element), but has read-only access to a small sub-array surrounding the local element so that
a stencil can be applied.  We demonstrate how a locally-oriented programming model can be adopted as
an embedded, domain-specific language using source-to-source program transformations.


\keywords{domain specific language, stencil compilers, distributed memory parallelism}
\end{abstract}

\section{Motivation}

The concepts in this paper were motivated by problems encountered during
the development of PetaVision, a C++ software framework designed for simulating
large networks of spiking neurons in the visual cortex of primates.  PetaVision
was designed to run on massively parallel hardware architectures and a primitive
version of PetaVision achieved over a Petaflop of sustained single precision
performance on Roadrunner, the fastest computer in the world for a brief
period of time.

TODO: Describe neural layers, spiking neurons, synaptic connections, weights,
and plasticity. Cite Heubel.

TODO: If space allows, give a simple 1D example of connections within a neural network in
Figure 1.  Describe the size of visual cortex in terms of the number of neurons,
the number of synaptic connections, and performance.  Reference SC Gorden Bell
paper from IBM.

TODO: Explain difficulties encountered.

\section{Programming Model}

The LOPe programming model
restricts the programmer to a local view of the index
space of an array.  Within a LOPe function, only a single array
element (called the local element) is mutable.  In addition, a small
halo region surrounding the local element is visible to the
programmer, but this region is immutable.  Restricting the programmer
to a local index space serves to reduce complexity by separating all
data- and task-decomposition concerns from the implementation of the
element-level array calculations.

%%This reduction in complexity reduces programming errors.  While developing the convolution example described later in the paper we made an indexing error in applying the 2D stencil loops in the standard serial Fortran test implementation.  This error required over 3 hours of programming time to repair.  First the error had to be isolated to the function implementing the convolution (it was first thought to be in the complicated tiff image output routine as the convolution code was ``thought'' to be too simple to wrong.  Then the index error has to be understood.  As will be seen, LOPe makes it more difficult to make these errors as Fortran array intrinsics can be used.  In some instance, the restricted semantics of LOPe allows the compiler to catch errors (e.g., some errors involving race conditions).

LOPe is a domain specific language (DSL) implemented as a small extension to the Fortran 2008
standard.  Fortran was chosen as the base language for LOPe because it provides a rich array-based
syntax.  Although, in principle, the same techniques could be applied to languages such as C or C++.

%%Readers who are unfamiliar with Fortran syntax may wish to consult Appendix A, for a brief description of Fortran notation.

\subsection{Related work}

LOPe builds upon prior work studying how to map Fortran to accelerator
programming models like OpenCL.  In the ForOpenCL project~\cite{Sottile:2013:FTE:2441516.2441520}
we exploited Fortran's pure and elemental functions to express
data-parallel kernels of an OpenCL-based program.  In practice, array
calculations for a given index $i,j$ will require read-only access to
a local neighborhood of size $[M,N]$ around $i,j$.  LOPe extends this work by introducing
a mechanism for representing these neighborhoods as array declaration
type annotations.

ForOpenCL was based on concepts explored in the ZPL programming language~\cite{chamberlain04zpl} in
which the programmer can define regions and operators that are applied over the index sets
corresponding to the sub-array regions.  This approach is quite powerful for
compilation purposes since it provides a clean decoupling of the operators applied over an array
from the decomposition of that array over a potentially complex distributed memory hierarchy.
However, unlike the ZPL operations on entire sub-arrays, LOPe expresses operations based on
a \emph{single} local array-element.

%
% For space reasons a ZPL example is not shown and text reworded accordingly
%
%%For example, the following ZPL code implements the same stencil as the LOPe code in Fig. 1:

%%\begin{verbatim}
%%ZPL jacobi example here.
%%\end{verbatim}
%%This approach as demonstrated by ZPL is quite powerful for compilation purposes since it provides a clean decoupling of the operators applied over an array from the decomposition of that array over a potentially complex distributed memory hierarchy.  

\subsection{LOPe Syntax Extensions}

There are only a few syntax additions required for a LOPe program.
These additions include syntax for describing halo regions and
concurrent procedures.  In code examples that follow,
language additions are highlighted by the usage of capitalization for
keywords that are either new or that acquire new usage.

\subsubsection{Halo regions.}
The principle semantic element of LOPe is the concept of a halo.
A halo is an ``artificial'' or ``virtual'' region surrounding
an array that contains boundary-value information.  Halo (also called
ghost-cell) regions are commonly employed to unify array indexing
schemes in the vicinity of an array boundary so that an array may be
referenced using indices that fall ``outside'' of the logical domain
of the array.  In LOPe, the halo region is given explicit syntax so
that the compiler can exploit this information for purposes of memory
allocation, data replication and thread synchronization.  For example,
a halo region can be declared with a statement of the form,

\begin{verbatim}
  real, allocatable, dimension(:), HALO(1:*:1) :: A
\end{verbatim}
This statement indicates that \texttt{A} is a rank one array, will be
allocated later, and
has a halo region of one element surrounding the array on either side.
The halo notation \texttt{M:*:N} specifies a halo
of \texttt{M} elements to the left, \texttt{N} elements to the
right, and an arbitrary number of ``interior'' array elements.
When used to describe a formal parameter of a
function, such as the type-declaration statement, \texttt{real, HALO(:,:) :: U},
the halo size is inferred by the compiler
from the actual array argument provided at the
calling site of the function.

%%In LOPe, there is no need (in this instance) for a repetitive \texttt{dimension(:,:)} specification, as it is inferred from the \texttt{HALO} specification.

\subsubsection{Concurrent functions.}

The second keyword employed by LOPe is \texttt{concurrent} which already exists in the form of a
\texttt{do} \texttt{concurrent} loop, whereby the programmer asserts that specific
iterations of the loop body may be executed by the compiler in \emph{any order,} even
\emph{concurrently.}  LOPe allows a function with the attributes \texttt{pure} (assertion of no
side effects) and \texttt{concurrent} (assertion of no dependencies between iterations) to be
called from within a \texttt{do} \texttt{concurrent} loop.  An example of a LOPe function is
shown in Fig. 1 and an example calling this function will be provided later in the text.  One
should imagine that a LOPe function is called \emph{for each} \texttt{i,j} index of the interior of
the array \texttt{U}.  Note that this usage introduces a race condition as new values of
elements of \texttt{U} are created on the left-hand side of the assignment statement that may use
\emph{new or old} values of \texttt{U} on the right-hand side.  LOPe requires the compiler to
guarantee that race conditions won't occur by using, e.g., double-buffering techniques as needed.

\vspace{-.1in}

\begin{figure}
\begin{verbatim}
           pure CONCURRENT subroutine Laplacian(U)}
               real, HALO(:,:) :: U
               U(0,0) =                 U(0,+1)              &
                        +  U(-1,0)  - 3*U(0, 0)  +  U(+1,0)  &
                                    +   U(0,-1)
           end subroutine Laplacian
\end{verbatim}
\vspace{-.1in}
\caption{A LOPe function implementing a Laplacian kernel in two dimensions.}
\end{figure}

\vspace{-.3in}

\subsubsection{LOPe index notation.}

In the \texttt{Laplacian} example the \texttt{U(0,0)} array element is the \emph{local} array
element and only the local element may be modified.  This zero-based indexing for the local-array
element differs from conventional Fortran, where by default, array indices start at 1.  The use of
zero-based indexing gives a clean symmetry for indices on either side of the central element at
zero. The other array elements are in the halo region and are \texttt{U(-1,0)} and
\texttt{U(+1,0)} (left and right of local, respectively) and \texttt{U(0,-1)} and \texttt{U(0,+1)}
(below and above of local).  The geometric positioning of the array elements can be
seen by examining the arrangement of the expressions on the right-hand side of Fig. 1.

\section{Coarray Fortran Extensions}

%%We highlight the Fortran elemental source-code abstraction because it provides a local orientation similar to that of OpenCL or CUDA kernel functions that were designed to exploit streaming, highly-threaded architectures like GPUs.  Elemental functions are pure in that they are guaranteed to be free of side effects such as I/O.  Consider a convolution where a 3x3 filter is applied to individual pixel elements in a photograph to average of blur the original photograph.  A convolution is similar to stencil operations in computational fluid dynamics that are used to obtain spatial derivatives of state variables.  Simple code is shown with extensions to Fortran shown in capital letters.  The extensions could be provided to C or Fortran functions via compiler directives rather than explicit language syntax.

Consider the \texttt{Laplacian} concurrent function in Fig. 1.  In this section we demonstrate how
this function can be called in the normal context of a program, one that allows full access to all
of the interior elements of the array, as well as array elements within the logically exterior,
halo-boundary region.  Topics highlighted in this section are: 1. distributed memory array
allocation; 2. explicit memory placement; 3. remote memory transfer; and 4. remote execution.  This
description is within the context of extensions to Fortran; as shorthand, these extensions are
referred to as CAFe, for Coarray Fortran extensions.  CAFe is complementary to previous work
extending coarray Fortran\cite{mellor-crummey:2009:caf2,jin:2011:caf2}.

\subsection{Subimages}

We introduce the important new concept of a CAFe subimage.  Fortran images are a collection of
distributed memory processes that all run the same program (image).  LOPe extends the concept of a
Fortran image by allowing images to be hierarchical.  By this we mean that each image \emph{may}
have a subimage (or subimages), but this subimage is not visible to other regular Fortran images.
Subimages also execute differently than normal images and may execute on different non-homogeneous
hardware, .e.g, an attached accelerator device.  Subimages are task based while images all execute
a Single Program but with different (Multiple) Data (SPMD).  A task can be given to a subimage, but
execution on the subimage terminates once the task is finished.  Memory on a subimage is permanent,
however, and must be specifically allocated and deallocated.

One obtains a subimage by executing the new LOPe function call, \texttt{device = GET\_SUBIMAGE(1)},
where the integer argument represents an attached hardware device (or a separate process).
If the function fails (e.g., the requested device is unavailable) it returns the image number
\texttt{this\_image()} of the process that is executing the current program.  Returning the current
image allows program execution to proceed correctly even if there are no attached devices.

\subsection{CAFe Example}

We start with the declaration of an array with an explicit halo size and with two local
dimensions (rank) and two distributed memory codimensions (corank),
\begin{verbatim}
   real, allocatable, dimension(:,:), codimension[:,:]      &
         HALO(1:*:1,1:*:1) :: U
\end{verbatim}
The corank of the array is chosen to be identical to the rank of the array so that the logical
process topology aligns in a way that allows a natural halo exchange between logically neighboring
processes (this could not occur if corank and rank are not the same).  For example, if the process
location is \texttt{[pcol,prow]}, then the right-hand halo for the local array \texttt{U} can be
obtained by the assignment
\texttt{U(M+1,:) = U(1,:)[pcol+1,prow]} where the size and cosize of
\texttt{U} are given by the allocation statement,
\texttt{allocate(U(0:M+1,0:N+1)[MP,*])}.
This allocation statement specifies (given the one element halo size provided earlier for
\texttt{U}) that the left halo column is \texttt{U(0,:)}, the right column is \texttt{U(M+1,:)},
the bottom row is \texttt{U(:,0)} and the top row is \texttt{U(:,N+1)}, leaving the interior region
\texttt{U(1:M,1:N)}.

In this allocation statement, the total number of process columns $NP$ can be obtained at runtime,
but \emph{may not} be explicitly provided (according to Coarray Fortran (CAF) rules) because the actual number of
participating processes (in Fortran called images) is variable, depending on how many processes are
requested at program startup.  In this discussion, it is \emph{assumed} that there are no holes in
the logical processor topology, thus $MP*NP = P$, where $MP$ is the number of process rows and $P$
is the total number of participating processes (images).

Once a subimage is obtained, memory on the device can be allocated,
\begin{verbatim}
   if (device /= this_image()) then
      allocate(U[device], HALO_SRC=U)   [[device]]
   end if
\end{verbatim}
There are four points to note regarding this memory allocation: 1. Memory is only allocated if a
subimage has been obtained; 2. The location where memory is allocated is denoted by regular coarray
notation \texttt{U[device]}; 3. The allocated size and halo attribute of the new array are obtained
from the previously allocated local array \texttt{U} via the notation \texttt{HALO\_SRC=U} (using
\texttt{HALO\_SRC} will also initially copy \texttt{U} to the subimage); and finally 4. The
allocation itself is \emph{executed} on the subimage device with the notation \texttt{[[device]]}.

Fortran uses square bracket notation, e.g. \texttt{[image]}, to specify on what process the
memory reference is physically located.  Square brackets are a visual clue to the
programmer that the memory reference may be remote and therefore potentially suffer a
performance penalty.  CAFe extends this by employing double-bracket notation to indicate
possibly \emph{remote subimage execution}.

Execution of the \texttt{Laplacian} task is done using the \texttt{do}
\texttt{concurrent} construct:
\begin{verbatim}
   do while (.not. converged)
      do concurrent (i=1:M, j=1:N)   [[device]]
         call Laplacian( U(i,j)[device] )
      end do
      call HALO_TRANSFER(U, BC=CYCLIC)
   end do
\end{verbatim}
There are several points that require highlighting: 1. Iteration occurs over the interior
of the array domain, \texttt{(i=1:M, j=1:N)}; 2. Execution of the loop body occurs on the
specific subimage indicated by \texttt{[[device]]}; 3. Execution of the iterates may occur
in any order, even \emph{concurrently}; 4. The local element of the array (as
defined above in reference to the definition of the concurrent procedure
\texttt{Laplacian}) is given by the indices \texttt{(i,j)}; 5. Location of memory for the
task is to be taken from the subimage as noted by \texttt{[device]}; 5. All threads must finish
execution of the loop body before further execution of the program proceeds; and 6. Transfer of
all requisite halo regions is effected by the call to the new LOPe intrinsic function
\texttt{HALO\_TRANSFER()}.  This function is a synchronization event in that all images must
complete the halo transfer before program execution continues.

Note that a transfer of halo memory is necessary after each completion of the do concurrent loop.
This must be done in order for the halo region of a coarray on a given process to be consistent
with the corresponding interior of the coarray on a logically neighboring process.
Finally, memory for the entire array \texttt{U} can be copied from the
subimage device with the statement,
\texttt{U = U[device]}, 
and memory deallocation (not shown) is similar to memory allocation.

%%\begin{verbatim}
%%   deallocate(U)
%%   if (device /= this_image()) then
%%      deallocate(U[device])   [[device]]
%%   end if
%%\end{verbatim}


%%\subsection{Execution Semantics and Memory Management}

%%This section describes the hierarchical memory layout and how memory consistency between the 3-levels of memory is maintained.  It describes what is the compiler's responsibility and what is the programmer's responsibility.

%% Completion of a do concurrent construct indicates that all executing threads (if a threading model is used by the compiler) have completed and that all memory on the executing subimage is in a consistent state.

%%The {\tt HALO} function returns a copy of the 3x3 region of {\tt a} and its surrounding neighbors.  The {\tt convolve} function is free of race conditions because of the copy semantics of {\tt HALO} and the implied synchronization of the {\tt CONCURRENT} attribute, whereby no output variables can be updated before all threads have completed execution.  In addition, while any thread may load from an extended region about \emph{its} element with the {\tt HALO} function, it may only store into its own elemental location.


%% A note from a conversation with Matt regarding memory management
%%

%The Fortran language has much tighter restrictions on aliasing than does C.
%So unless a variable has the pointer or target attribute, it cannot be aliased.
%Thus the compiler is able to aggressively optimized for memory movement between
%the CPU and accelerator.  However, because the design philosophy of coarrays
%is that memory transfer between images can be expensive, the programmer must
%explicitly transfer memory between images with explicit syntax with square
%bracket notation, i.e., $a[1] = a[2]$.  So we allow the compiler to manage
%memory within an image but require the use of {\tt halo\_exchange} for the
%transfer of halo memory between images.


%Fortran currently supports:

%1. Array syntax: e.g., C = A + B, where A, B, C are arrays.  Note that this is implicitly a loop structure, but that no loop indices need be provided.  Also the programmer need not specify where in memory these arrays reside.  Thus this high level syntax allows the compiler more freedom in both memory placement (even across distributed memory nodes) and in runtime code execution (individual array element may be computed by different hardware threads).  This is a simple example and it is a research question as to what compiler directives would be useful for memory placement and other directions to the compiler for efficient code generation.

%2. Pure procedures:  Fortran has syntax for specifying procedures that have no side effects during execution.  Specifying code that is side-effect free code is important information to provide to the compiler so that it can generate efficient multi-threaded code.

%3. Pure elemental procedures: Fortran has syntax for specifying procedures that take only scalar arguments, but may be applied across array elements.  Elemental procedures are ideal for writing code to be executed within a hardware thread.  They resemble OpenCL kernels, but are simpler because they leave all indexing up to the compiler.

%We have determined that additional syntax is needed, in addition to the three language features described above, to allow programmers the ability to express code in Fortran to be targeted for multi-threaded hardware architectures like GPUs.  This additional syntax is provided by functions that return a copy of a small region of memory surrounding an array element (as seen within an elemental procedure) and with functions for thread synchronization.  This additional syntax will allow pure procedures to perform stencil and other convolution-like operations on a copy of memory, synchronize, then store the computed results back to the array element associated with the given thread.

%In addition Fortran has syntax like the target attribute the specifies when variables can be aliased.  This allows for much easier program analysis as the compiler knows that ordinary variables cannot be aliased.  Fortran also has excellent facilities for interoperability with C so that programming in a mixed language environment is easily accomplished, including interoperability with native Fortran arrays.

\subsection{Comparison to Coarray Fortran}

LOPe provides a purely \emph{local} viewpoint; the programmer is only provided read and write access
to the local array element and write access to a small halo region surrounding the local element.
There is simply \emph{no} mechanism provided for the programmer to even know \emph{where} the local
element is in the context of the broader array.  On a distributed memory architecture, the halo
elements may not even be physically located on the same processor.  If executed on a cluster
containing hybrid processing elements (e.g. GPUs), the halo elements may be as far as three hops
away: one to get to the host processor and another two to get to memory on the hybrid processor
executing on another distributed memory node.  LOPe provides a complete separation between algorithm
development and memory management and synchronization (between memory copies of the same logical
array region covered by halos).  By explicitly describing the existance and size of an array's halo
region, the compiler is provided with enough information to manage most of the hard and detailed work
involved in memory transfer and synchronization.

Additionally, the semantics of the LOPe execution model (on which more will be said later)
remove even the \emph{possibility} of race conditions developing during execution of a concurrent
procedure.

We emphasize some of these advantages by compairing the \texttt{Laplace} implementation shown above
with the implementation of the same algorithm from the original Numrich and Reid paper describing
coarrays in Fortran.  We should point out that this comparison is somewhat unfair because
Numrich and Reid were describing the advantages of coarray notation for transferring memory
on distributed memory architectures, not how ideally to use coarrays within a major code.  But
this example serves to highlight the advantages of LOPe as described.  In the coarray example
shown below type declarations have been removed to save space:
{\small \begin{verbatim}
subroutine Laplace (nrow,ncol,U)
   left = me-1     ! me refers to the current image
   if (me == 1) left = ncol
   right = me + 1
   if (me == ncol) right = 1
   call sync_all( [left,right] ) ! Wait if left and right have
                                 ! not already reached here
   new_u(1:nrow)=new_u(1:nrow)+u(1:nrow)[left]+u(1:nrow)[right]
   call sync_all( [left,right] )
   u(1:nrow) = new_u(1:nrow) - 4.0*u(1:nrow)
end subroutine
\end{verbatim}}

The advantages of LOPe over this example are now described.  Please note the the following
is not a criticism of Coarray Fortran (CAF) as CAF is a general purpose parallel programming
language and LOPe only pertains to the halo pattern useful in stencil-based algorithms.

%%Please note that this example is somewhat unfair because in practice CAF is usually refactored in a locally oriented way by so that communication and synchronization separated into separate procedures.  Locally-oriented programming should be viewed as programming methodology with LOPe as a particular instance.
\subsubsection{LOPe advantages.}
\begin{itemize}

\item
LOPe requires that the implementation of the algorithm to be separate from the call to
effect the halo transfer.  Removing boundary condition specification (e.g., the cyclic
boundary conditions implemented in the CAF example) from the algorithm allows the boundary
conditions to be changed and without changing algorithm code.

\item
LOPe applies the transfer of halo memory across multiple (possibly) levels of memory with
the LOPe intrinsic \texttt{TRANSFER\_HALO} function (described later).  Thus the LOPe
algorithm can be run on a machine with many interconnected nodes, each containing hybrid
processor cores.  The coarray example can only be run on multiple nodes (called images in
Fortran) without accelerator cores.

\item
The algorithm implementation is separate from user-specified synchronization, e.g.,
\texttt{call sync\_all}.  In LOPe, synchronization is subsummed in the semantics of the
\texttt{CONCURRENT} attribute and the \texttt{TRANSFER\_HALO} function described later.

\item
The algorithm implementation is separate from any specification as to where the array
memory is located.  The CAF example explicity denotes where memory is located with the
\texttt{[left]} and \texttt{[right]} syntax where left and right specifiy a processor
topology.

\item
The algorithm implementation is separate from any specification as to where the algorithm
is to be executed.  The CAF example explicity denotes where a statement is to executed
with control flow construct like \texttt{if (me == 1)}.

\item
The LOPe implementation is easier to understand and frequently follows the mathematical
algorithm directly.  For example, the CAF example adds 4 neighbors plus the center value
to make the implementation with direct remote coarray access possible, while the LOPe
example is able to implement the same algorithm with fewer operations by adding 4
neighbors (not including the center array element) and then only subtracting 3 center
values.

\item
The semantics of LOPe makes explicit management of array temporaries (e.g., \texttt{U} and
\texttt{new\_U} by programmers unnecessary (though still possible).  Because in LOPe the
halo region is a language construct, the compiler is better able to manage temporary
buffers than users on the target hardware platform.

\end{itemize}

\subsubsection{Errors that are constrained by the language.}
\begin{itemize}

\item
A programmer is not able to store data to the halo region.  If this were allowed, one
thread could overwrite another threads data at undefined times.  The compiler is able to
catch this class of error.

\item
A programmer can't make indexing errors in a concurrent routine by going out of bounds of
the array plus halo memory.  The compiler is able to catch this class of error at compile
time as long as compile-time constants are used to specify halo sizes.

\item
A programmer is not able to cause race conditions by forgetting to create and
use temporary arrays properly.  In LOPe it is the comilers responsibility to
store data in temporary memory.

\item
A programmer can't make synchronization errors as synchronization is implicit in
the \texttt{CONCURRENT} attribute.  A thread running a concurrent procedure is provided
with a copy of it's array element (plus halo) that is consistent with the state
of memory at the time of invocation of procedure.  Stores to an individual
thread's local array element (by that thread) is never visible to other threads.
LOPE encourages the creation of small functions and lets the compiler
fuse the procedures together to provide the necessary synchronization.

\end{itemize}


%
% This could be part of conclusions
%

%%Note that the use of halo cells is the normal way that large and complex MPI and CAF programs are implemented.  LOPe proposes to formalize this common pattern into the Fortran language allowing the compiler access to this information in order to spread computation over more hardware resources, improve performance, and to reduce complexity for the programmer.

\section*{Appendix: A Fortran Primer for the C Programmer}

This section provides a brief description of Fortran syntax for readers unfamiliar with Fortran
syntax.  The most common source of confusion is array notation.  Fortran uses parentheses
\texttt{()} for both functions arguments and array indexing; thus the confusing expression
\texttt{foo(i,j)} can either be an array reference or a function call depending on context.  Where C
reserves square brackets \texttt{[]} to denote an array reference, Fortran uses square brackets to
reference coarray memory on a remote, distributed memory process (called an image in Fortran).

Otherwise, Fortran syntax is usually straightforward though it may appear strange to the uninitiated
programmer.  A subroutine is a function with no return value and must be called in a statement
rather than as an expression.  Simple syntactic elements are: \texttt{!} the line comment symbol;
\texttt{\&} the line continuation symbol; and \texttt{::} is syntactic sugar used for the separation
of syntactic elements, e.g., separating type attributes from the variable list in a type declaration
statement.


\section*{Acknowledgments}
This work was supported in part by the Department of Energy Office of Science, Advanced Scientific
Computing Research.  The authors would also like to thank Robert Robey and Wayne Weseloh at Los
Alamos National Laborabory for several stimulating conversations with respect to programming models.

\end{document}
