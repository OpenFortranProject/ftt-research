\section{Conclusions}

Fortran is a general-purpose programming language and as such it provides limited facilities for
expressing concepts useful for stencil operations. For example, halo regions must be expressed in
terms of the existing syntax of the language as there is no way to specify that the ``interior'' of
an array is in any way special from an ``outside'' region.  By not providing support for stencils in
the language, the programmer must make specific choices regarding the order of access to the data
and in the order of operations on the data.  These choices often hide the opportunity for
optimizations by the compiler \cite{dubey:2014:wosc}.  If the compiler had knowledge of the symantic
intent of halo regions, it could reorder operations so that border regions were computed
\emph{before} interior regions, allowing the the transfer of data in halo regions to overlap with
computations on the interior.

Just as Coarray Fortran originally extended Fortran to include domain specific knowledge (parallel
computation) replacing MPI library routines\cite{numeric:reid}, LOPe seeks to extend Fortran by
providing domain specific knowledge of stencil operations.  

CAFe extensions ...
* subimages

Specifically, LOPe and CAFe together provide:
* A local view of stencil operations on data.  This allows a complete separation of the implementation of a stencil algorithm with data access patterns.
* Memory placement via allocation routines allow the specification of the allocation location.
* Task execution placement with [[ ]] specifying which subimage is to execute a particular operation.
* Memory exchange via the \texttt{HALO\_TRANSFER} intrinsic procedure.
* Ability to run correctly even if subimages are not available.  Similar to OpenMP.
* Ability to associate arrays on subimages with coarrays on host thus allowing compiler to effect the halo transfers between the multiple memory levels.
* Halo abstraction that allows for more flexibility for the compiler to retarget for different architectures.
* A simple way to express variables as ``thread private'' as all local variables declared in a \texttt{CONCURRENT} procedure are private to the calling procedure.  This is distinguished from OpenMP.

\subsubsection{Benefits.}
* extends the programmer
* restricts the programmer
* no race conditions
* Abstractions provide ample opportunities for compiler optimizations.

\subsubsection{Limitations.}
* no composition useful for nonlinear dynamics\cite{coella}
* difficult (but not impossible) to implement algoriths that perform operations between arrays of
differing sizes (multigrid implementations for example).

By implementing LOPe
we have demonstrated that LOPe can be ... It remains to history to see if LOPe is general purpose
enough to be included in Fortran or if it should remain as a special purpose preprocessing tool.
