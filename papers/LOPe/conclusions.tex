\section{Conclusions}

Fortran is a general-purpose programming language and as such it provides limited facilities for
expressing concepts useful for stencil operations. For example, halo regions must be expressed in
terms of the existing syntax of the language and there is no way to specify that the ``interior''
of an array is in any way special from an ``outside'' region.  By not providing support for
stencils in the language, the programmer must make specific choices regarding data-access patterns
and the order of operations on the data.  These choices often hide the opportunity for
optimizations by the compiler \cite{Dubey:2014:SSC:2686745.2686756}.  For example, if the compiler
had knowledge of the semantic intent of halo regions, it could reorder operations so that border
regions were computed \emph{before} interior regions, allowing the transfer of data in halo regions
to overlap with computations on the interior.

Just as Coarray Fortran originally extended Fortran to include domain specific knowledge (parallel
computation) replacing MPI library routines\cite{Numrich:1998:CFP:289918.289920}, LOPe seeks to extend Fortran by
providing domain specific knowledge of stencil operations.
Specifically, LOPe and CAFe together provide:
(1) a local view of stencil operations on data that allows a complete separation of the implementation of a stencil
algorithm with data-access patterns;
(2) memory placement via allocation routines that allow the specification of allocation location;
(3) task execution placement with double-bracket syntax specifying which subimage is to execute a particular operation; and
(4) memory exchange via the \texttt{TRANSFER\_HALO} intrinsic procedure.

%
% This could be part of conclusions
%

%%Note that the use of halo cells is the normal way that large and complex MPI and CAF programs are implemented.

\subsubsection{Benefits.}
LOPe proposes to formalize the common, halo software pattern in language syntax, thus providing the
compiler with access to halo information in order to spread computation over more hardware
resources, improve performance, and to reduce complexity for the programmer.  Furthermore, LOPe
semantics provide important \emph{language restrictions} that remove the possibility of race conditions
that occur when multiple threads have write access to overlapping data regions.

\subsubsection{Limitations.}
LOPe only supports regular structured grids through Fortran multi-dimensional arrays and a
corresponding multi-dimensional processor layout.  It does not allow the composability of stencils
required by non-linear physics operators, nor does it provide automatic support for the storage of
intermediate results resulting from multiple intermediate update steps.  Adaptive Mesh Refinement
(AMR) Shift Calculus provides a generalized abstraction that addresses many of these concerns
(see\cite{Dubey:2014:SSC:2686745.2686756} and references therein).  However, it may be possible to
support AMR in LOPe through locally-structured grid methods based on the work of Berger and
Oliger\cite{colella2007performance}.  In this instance, LOPe could be used to update the regular
array regions in each rectangular patch.

\vspace{.1in}

By implementing LOPe we have demonstrated that LOPe can be used to easily and succinctly code
the stencil algorithms that are common to many areas of science, and furthermore, that LOPe is
suitable for transformation to languages like OpenCL that support heterogeneous computing.  It
remains to history to ascertain if LOPe is sufficiently general purpose to be included in a
general-purpose programming language or if it is better suited to remain as a DSL and to be used as a
special-purpose preprocessing tool.
