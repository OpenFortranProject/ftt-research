\section{Programming Model}

The LOPe (for Locally-Oriented Programming extensions) programming model restricts the programmer ---
when implementing a stencil algorithm --- to a local view of the array index space.  Within a LOPe
function, only the local array element (plus a small halo region surrounding the local element) is
visible to the programmer.  This serves to reduce complexity by removing all boundary conditions and
processor topology information from the implementation of the algorithm.

%%This reduction in complexity reduces programming errors.  While developing the convolution example described later in the paper we made an indexing error in applying the 2D stencil loops in the standard serial Fortran test implementation.  This error required over 3 hours of programming time to repair.  First the error had to be isolated to the function implementing the convolution (it was first thought to be in the complicated tiff image output routine as the convolution code was ``thought'' to be too simple to wrong.  Then the index error has to be understood.  As will be seen, LOPe makes it more difficult to make these errors as Fortran array intrics can be used.  In some instance, the restricted semantics of LOPe allows the compiler to catch errors (e.g., some errors involving race conditions).

LOPe was implemented as an embedded domain specific language (DSL).  Fortran was chosen as the base
language for LOPe, although in principle, the same techniques could be applied to language such as C
or C++ (in conjuction with some form of array extensions such as those proposed by Intel).  Fortran
was chosen because it already provides a rich array-based syntax.  Readers who are unfamiliar with
Fortran syntax may wish to consult Appendix A for a brief description of Fortran notation.

\subsection{LOPe Syntax Extensions}

There are only a few syntax additions required for LOPe.  In the code examples describing these
additions (and throughout this paper as well), DSL syntax additions are highlighted by the usage of
capitalization for all new keywords.

\subsubsection{HALO.}

The principle semantic element of LOPe is the concept of a halo region.  A halo is an ``artificial''
or ``virtual'' region surrounding an array that contains boundary-value information.  Halo (also
called ghost-cell) regions are commonly employed to unify array indexing schemes in the vicinity of
an array boundary, so that the array may be referenced using indicies that fall ``outside'' of the
domain of the array.  In LOPe, the halo region is given explicit syntax so that the compiler can
exploit this knowledge.  For example, a halo region can be explicitly provided in a declaration
statement of the form,
\begin{verbatim}
  real, allocatable, dimension(:), HALO(1:*:1) :: A
\end{verbatim}
This statement indicates that array \texttt{A} is one dimensional, will be allocated later when it
is provided with a specific size, and has a halo region of 1 surrounding the array on both sides.
The notation \texttt{M:*:N} is to be read as describing a halo region of size \texttt{M} to the
left, size \texttt{N} to the right, and with an arbitrary size in the middle (denoted by the $*$
symbol).  When used to describe a formal parameter of a function, the \texttt{HALO} keyword may
``assumed'' (from the actual argument by the compiler) as is the following statement,
\begin{verbatim}
  real, HALO(:,:) :: U
\end{verbatim}
which in this instance indicates that \texttt{U} is a two-dimensional array, with a halo region
whose size is ``assumed'' by the compiler from the explicit halo size of the actual array argument
(provided at the calling site of the function).  In LOPe, there is no need in this instance for a
repetitive \texttt{dimension(:,:)} specification as it is inferred by the \texttt{HALO}
specification.

\subsubsection{CONCURRENT.}

The second keyword employed by LOPe (although in this case extended) is \texttt{CONCURRENT}.  The
\texttt{concurrent} keyword already exists in the form of a \texttt{do concurrent} loop construct (a
Fortran 2008 feature).  By using \texttt{do concurrent}, the programmer asserts that specific
iterations of the loop body may be executed by the compiler in \emph{any order;} even
\emph{concurrently.}  For example, the following,
\begin{verbatim}
  pure CONCURRENT subroutine Laplace(U)
     real, HALO(:,:) :: U
     U(0,0) =                 U(0,+1)               &
              +  U(-1,0)  - 3*U(0, 0)  +  U(+1,0)   &
                          +   U(0,-1)
  end subroutine Laplace
\end{verbatim}
defines a function that can be used as part of an iterative solution of Laplace's equation in two
dimensions.  A function with the \texttt{pure} and \texttt{CONCURRENT} attributes may be called from
with a \texttt{do concurrent} loop.  One should imagine that this function is called \emph{for each}
\texttt{(i,j)} iterate of the loop.

\subsubsection{LOPe index notation.}
In the \texttt{Laplace} example shown above, the \texttt{(0,0)} array element is the \emph{local} 
array element and only the local element may be modified.  

\subsection{Comparison to Coarray Fortran}

LOPe provides a purely \emph{local} viewpoint; the programmer is only provided read and write access
to the local array element and read access to a small halo region surrounding the local element.
There is simply \emph{no} mechanism provided for the programmer to even know \emph{where} the local
element is in the context of the broader array.  On a distributed memory architecture, the halo
elements may not even be physically located on the same processor.  If executed on a cluster
containing hybrid processing elements (e.g. GPUs), the halo elements may be as far as three hops
away: one to get to the host processor and another two to get to memory on the hybrid processor
executing on another distributed memory node.  LOPe provides a complete separation between algorithm
development and memory management (synchronization between memory copies of the same logical
array region covered by halos).  By explicitly describing the existance and size of an array's halo
region, the compiler is provided with enough information to manage most of the hard and detailed
work involved in memory transfer and synchronization.  Additionally, the semantics of the LOPe
execution model remove the possibility of race conditions developing during execution of a
concurrent procedure.

We emphasize some of these advantages by comparing the \texttt{Laplace} implementation with the
implementation of the same algorithm from the original Numrich and Reid paper first describing
coarrays in Fortran.  We should point out that this comparison is somewhat unfair, because Numrich
and Reid were introducing coarray notation for transferring memory on distributed memory
architectures, not demonstrating how ideally one should use coarrays within a large application.
%%Please note that this example is somewhat unfair because in practice CAF is usually refactored in a locally oriented way by so that communication and synchronization separated into separate procedures.  Locally-oriented programming should be viewed as programming methodology with LOPe as a particular instance.
However this example serves
to highlight some of the advantages of LOPe and CAFe as introduced above.  Note that in the coarray
example shown below, type declarations have been removed to save space:
{\small \begin{verbatim}
  subroutine Laplace (nrow,ncol,U)
    left = me-1      ! me refers to the current image
    if (me == 1) left = ncol
    right = me + 1
    if (me == ncol) right = 1
    call sync_all( [left,right] )   ! Wait for left and right
    new_u(1:nrow) = new_u(1:nrow) + u(1:nrow)[left] + u(1:nrow)[right]
    call sync_all( [left,right] )
    u(1:nrow) = new_u(1:nrow) - 4.0*u(1:nrow)
  end subroutine
\end{verbatim}}

The advantages of LOPe and CAFe over this example are now described.  Please note that the following
is not a criticism of coarray Fortran as CAF is a general purpose parallel programming language and
LOPe only pertains to the halo pattern most useful in the implementation of stencil-based algorithms.

\subsubsection{LOPe advantages.}
\begin{itemize}

\item
LOPe requires the implementation of the algorithm to be separate from the call to effect the halo
transfer.  Removing boundary condition specification (e.g., the cyclic boundary conditions
implemented in the CAF example) from the algorithm allows the boundary conditions to be changed
without changing algorithm code.

\item
LOPe applies the transfer of halo memory across possibly multiple levels of memory with the LOPe
intrinsic \texttt{TRANSFER\_HALO} function.  Thus the LOPe algorithm can be run on a machine with
many interconnected nodes, each containing hybrid processor cores.  The coarray example can only be
run on multiple nodes without accelerator cores.

\item
Algorithm implementation is separate from user-specified synchronization, e.g., \texttt{call
  sync\_all}.  In LOPe, synchronization is subsummed in the semantics of the \texttt{CONCURRENT}
attribute and the \texttt{TRANSFER\_HALO} function call.

\item
The algorithm implementation is separate from any specification as to where the array
memory is located.  The CAF example explicity denotes where memory is located with the
\texttt{[left]} and \texttt{[right]} syntax where left and right specifiy a processor
topology.

\item
The algorithm implementation is separate from any specification as to where the algorithm
is to be executed.  The CAF example explicity denotes where a statement is to executed
with control flow construct like \texttt{if (me == 1)}.

\item
The LOPe implementation is easier to understand and frequently follows the mathematical algorithm
directly.  For example, the CAF implementation of Numrich and Reid adds 4 neighbors plus the center
value to make the implementation with direct remote coarray access possible, while the LOPe example
is able to implement the same algorithm with one statement and no intervening synchronization.

\item
The semantics of LOPe makes explicit management of array temporaries (e.g., \texttt{u} and
\texttt{new\_u} by the programmer unnecessary (though still possible).  Because in LOPe the
halo region is a language construct, the compiler is better able to manage temporary
buffers than users on the target hardware platform.

\end{itemize}

\subsubsection{Errors that are constrained by the language.}
\begin{itemize}

\item
A programmer is not able to store data to the halo region during execution of a LOPe concurrent
function.  If this were allowed, one thread could overwrite another threads data at undefined times.
The compiler is able to catch this class of error.

\item
A programmer can't make indexing errors in a concurrent routine by going out of bounds of the array
plus halo memory.  The compiler is able to catch this class of error at compile time as long as
compile-time constants are used to specify halo sizes.

\item
A programmer is not able to cause race conditions by forgetting to create and use temporary arrays
properly.  In LOPe it is the compilers responsibility to store data in temporary memory.

\item
A programmer can't make synchronization errors as synchronization is implicit in the
\texttt{CONCURRENT} attribute.  A thread running a concurrent procedure is provided with a copy of
it's local array element plus halo that is consistent with the state of memory \emph{at the time of
  invocation of the procedure}.  Stores to an individual thread's local array element (by that
thread) is never visible to other threads.  LOPE encourages the creation of small functions and lets
the compiler fuse the functions together to improve performance and to provide necessary
synchronization.

\end{itemize}




\subsection{Locally-Oriented Programming}

%%We highlight the Fortran elemental source-code abstraction because it provides a local orientation similar to that of OpenCL or CUDA kernel functions that were designed to exploit streaming, highly-threaded architectures like GPUs.  Elemental functions are pure in that they are guaranteed to be free of side effects such as I/O.  Consider a convolution where a 3x3 filter is applied to individual pixel elements in a photograph to average of blur the original photograph.  A convolution is similar to stencil operations in computational fluid dynamics that are used to obtain spatial derivatives of state variables.  Simple code is shown with extentions to Fortran shown in capital letters.  The extensions could be provided to C or Fortran functions via compiler directives rather than explicit language syntax.

Consider a simple average over a five point stencil in two dimensions.  The algorithm updates the
stencil interior point $A(0,0)$ as shown in the following:


The 

\begin{verbatim}
   real, allocatable, HALO(-1:*:1) :: A(:)[:], B(:)[:]

   !! Allocate local memory
   !  -------------------------------------------
   allocate(A(N)[*], B(N)[*])
\end{verbatim}

\begin{verbatim}
   device = get_subimage(1)          ! THIS_IMAGE() if device unavailable

   !! Allocate memory on subimage if available
   !  ----------------------------------------
   if (device /= THIS_IMAGE()) then
      allocate(U1h(0:J)  [device], HALO=(-1:*:1))   [[device]]
      allocate(U2h(0:J/2)[device], HALO=(-1:*:1))   [[device]]
   end if

   !! Initialize and transfer memory
   !  ------------------------------
   do j1h = 0, J
      U1h(j1h) = j1h
   end do
   U1h[device] = U1h

   do concurrent (j2h = 1:J/2-1)  [[device]]
      call restrict(U1h(2*j2h)[device], U2h(j2h)[device])
   end do

   do concurrent (j2h = 1:J/2-1)  [[device]]
      call restrict(U1h(2*j2h)[device], U2h(j2h)[device], [0.25, 0.5, 0.25])
   end do

   deallocate(U1h, U2h)

   if (device /= THIS_IMAGE()) then
      deallocate(U1h[device], U2h[device])   [[device]]
   end if
         
end program
\end{verbatim}



%%where {\tt A} and {\tt Avg} are arrays, but where the {\tt a} argument in the {\tt convolve} definition is declared as a scalar.  The {\tt HALO} function returns a copy of the 3x3 region of {\tt a} and its surrounding neighbors.  The {\tt convolve} function is free of race conditions because of the copy semantics of {\tt HALO} and the implied synchronization of the {\tt CONCURRENT} attribute, whereby no output variables can be updated before all threads have completed execution.  In addition, while any thread may load from an extended region about \emph{its} element with the {\tt HALO} function, it may only store into its own elemental location.

\subsection{Memory Management}

This section describes the hierarchical memory layout and how memory consistency between the 3-levels of memory is maintained.  It describes what is the compiler's responsibility and what is the programmer's responsibility.

%% A note from a conversation with Matt regarding memory management
%%

%The Fortran language has much tighter restrictions on aliasing than does C.
%So unless a variable has the pointer or target attribute, it cannot be aliased.
%Thus the compiler is able to agressively optimized for memory movement between
%the CPU and accelerator.  However, because the design philosophy of coarrays
%is that memory transfer between images can be expensive, the programmer must
%explicitly transfer memory between images with explicit syntax with square
%bracket notation, i.e., $a[1] = a[2]$.  So we allow the compiler to manage
%memory within an image but require the use of {\tt halo\_exchange} for the
%transfer of halo memory between images.


%Fortran currently supports:

%1. Array syntax: e.g., C = A + B, where A, B, C are arrays.  Note that this is implicitly a loop structure, but that no loop indices need be provided.  Also the programmer need not specify where in memory these arrays reside.  Thus this high level syntax allows the compiler more freedom in both memory placement (even across distributed memory nodes) and in runtime code execution (individual array element may be computed by different hardward threads).  This is a simple example and it is a research question as to what compiler directives would be useful for memory placement and other directions to the compiler for efficient code generation.

%2. Pure procedures:  Fortran has syntax for specifying procedures that have no side effects during execution.  Specifying code that is side-effect free code is important information to provide to the compiler so that it can generate efficient multi-threaded code.

%3. Pure elemental procedures: Fortran has syntax for specifying procedures that take only scalar arguments, but may be applied across array elements.  Elemental procedures are ideal for writing code to be executed within a hardware thread.  They resemble OpenCL kernels, but are simpler because they leave all indexing up to the compiler.

%We have determined that additional syntax is needed, in addition to the three language features described above, to allow programmers the ability to express code in Fortran to be targeted for multi-threaded hardware architectures like GPUs.  This additional syntax is provided by functions that return a copy of a small region of memory surrounding an array element (as seen within an elemental procedure) and with functions for thread synchronization.  This additional syntax will allow pure procedures to perform stencil and other convolution-like operations on a copy of memory, synchronize, then store the computed results back to the array element associated with the given thread.

%In addition Fortran has syntax like the target attribute the specifies when variables can be aliased.  This allows for much easier program analysis as the compiler knows that ordinary variables cannot be aliased.  Fortran also has excellent facilities for interoperability with C so that programming in a mixed language invironment is easily accomplished, including interoperability with native Fortran arrays.
