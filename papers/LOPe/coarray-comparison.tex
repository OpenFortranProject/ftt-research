\subsection{Comparison to Coarray Fortran}

LOPe provides a purely \emph{local} viewpoint; the programmer is only provided read and write access
to the local array element and write access to a small halo region surrounding the local element.
There is simply \emph{no} mechanism provided for the programmer to even know \emph{where} the local
element is in the context of the broader array.  On a distributed memory architecture, the halo
elements may not even be physically located on the same processor.  If executed on a cluster
containing hybrid processing elements (e.g. GPUs), the halo elements may be as far as three hops
away: one to get to the host processor and another two to get to memory on the hybrid processor
executing on another distributed memory node.  LOPe provides a complete separation between algorithm
development and memory management and synchronization (between memory copies of the same logical
array region covered by halos).  By explicitly describing the existance and size of an array's halo
region, the compiler is provided with enough information to manage most of the hard and detailed work
involved in memory transfer and synchronization.

Additionally, the semantics of the LOPe execution model (on which more will be said later)
remove even the \emph{possibility} of race conditions developing during execution of a concurrent
procedure.

We emphasize some of these advantages by compairing the \texttt{Laplace} implementation shown above
with the implementation of the same algorithm from the original Numrich and Reid paper describing
coarrays in Fortran.  We should point out that this comparison is somewhat unfair because
Numrich and Reid were describing the advantages of coarray notation for transferring memory
on distributed memory architectures, not how ideally to use coarrays within a major code.  But
this example serves to highlight the advantages of LOPe as described.  In the coarray example
shown below type declarations have been removed to save space:
{\small \begin{verbatim}
subroutine Laplace (nrow,ncol,U)
   left = me-1     ! me refers to the current image
   if (me == 1) left = ncol
   right = me + 1
   if (me == ncol) right = 1
   call sync_all( [left,right] ) ! Wait if left and right have
                                 ! not already reached here
   new_u(1:nrow)=new_u(1:nrow)+u(1:nrow)[left]+u(1:nrow)[right]
   call sync_all( [left,right] )
   u(1:nrow) = new_u(1:nrow) - 4.0*u(1:nrow)
end subroutine
\end{verbatim}}

The advantages of LOPe over this example are now described.  Please note the the following
is not a criticism of Coarray Fortran (CAF) as CAF is a general purpose parallel programming
language and LOPe only pertains to the halo pattern useful in stencil-based algorithms.

%%Please note that this example is somewhat unfair because in practice CAF is usually refactored in a locally oriented way by so that communication and synchronization separated into separate procedures.  Locally-oriented programming should be viewed as programming methodology with LOPe as a particular instance.
\subsubsection{LOPe advantages.}
\begin{itemize}

\item
LOPe requires that the implementation of the algorithm to be separate from the call to
effect the halo transfer.  Removing boundary condition specification (e.g., the cyclic
boundary conditions implemented in the CAF example) from the algorithm allows the boundary
conditions to be changed and without changing algorithm code.

\item
LOPe applies the transfer of halo memory across multiple (possibly) levels of memory with
the LOPe intrinsic \texttt{TRANSFER\_HALO} function (described later).  Thus the LOPe
algorithm can be run on a machine with many interconnected nodes, each containing hybrid
processor cores.  The coarray example can only be run on multiple nodes (called images in
Fortran) without accelerator cores.

\item
The algorithm implementation is separate from user-specified synchronization, e.g.,
\texttt{call sync\_all}.  In LOPe, synchronization is subsummed in the semantics of the
\texttt{CONCURRENT} attribute and the \texttt{TRANSFER\_HALO} function described later.

\item
The algorithm implementation is separate from any specification as to where the array
memory is located.  The CAF example explicity denotes where memory is located with the
\texttt{[left]} and \texttt{[right]} syntax where left and right specifiy a processor
topology.

\item
The algorithm implementation is separate from any specification as to where the algorithm
is to be executed.  The CAF example explicity denotes where a statement is to executed
with control flow construct like \texttt{if (me == 1)}.

\item
The LOPe implementation is easier to understand and frequently follows the mathematical
algorithm directly.  For example, the CAF example adds 4 neighbors plus the center value
to make the implementation with direct remote coarray access possible, while the LOPe
example is able to implement the same algorithm with fewer operations by adding 4
neighbors (not including the center array element) and then only subtracting 3 center
values.

\item
The semantics of LOPe makes explicit management of array temporaries (e.g., \texttt{U} and
\texttt{new\_U} by programmers unnecessary (though still possible).  Because in LOPe the
halo region is a language construct, the compiler is better able to manage temporary
buffers than users on the target hardware platform.

\end{itemize}

\subsubsection{Errors that are constrained by the language.}
\begin{itemize}

\item
A programmer is not able to store data to the halo region.  If this were allowed, one
thread could overwrite another threads data at undefined times.  The compiler is able to
catch this class of error.

\item
A programmer can't make indexing errors in a concurrent routine by going out of bounds of
the array plus halo memory.  The compiler is able to catch this class of error at compile
time as long as compile-time constants are used to specify halo sizes.

\item
A programmer is not able to cause race conditions by forgetting to create and
use temporary arrays properly.  In LOPe it is the comilers responsibility to
store data in temporary memory.

\item
A programmer can't make synchronization errors as synchronization is implicit in
the \texttt{CONCURRENT} attribute.  A thread running a concurrent procedure is provided
with a copy of it's array element (plus halo) that is consistent with the state
of memory at the time of invocation of procedure.  Stores to an individual
thread's local array element (by that thread) is never visible to other threads.
LOPE encourages the creation of small functions and lets the compiler
fuse the procedures together to provide the necessary synchronization.

\end{itemize}
