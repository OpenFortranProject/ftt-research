\section{Introduction}

Historically it has been hard to create parallel programs.  The
responsibility (and difficulty) of creating \emph{correct} parallel
programs can be viewed as being spread between the programmer and the
compiler.  Ideally we would like to have parallel languages that make
it easy for the programmer to express correct parallel programs (and
conversely it should be difficult to express incorrect parallel
programs).  Unfortunately, many current languages and standards place
all of the responsibility on the user.  The best example of this are
programs written using MPI (Message Passing Interface), where the
programmer expresses all parallelism in terms of calls to library
routines and the serial C or Fortran compiler knows nothing about the
parallel semantics of the combined language plus MPI library.

Of course one would hope that over time code complexity goes down as
better languages allow compilers to take on a greater share of the
parallel complexity burden.  Multithreaded code written using OpenMP
can be significantly simpler than the corresponding code in which the
programmer explicitly manages a pool of threads themselves.  This is
somewhat true for distributed memory parallelism using Unified
Parallel C (UPC) and Coarray Fortran.  With the PGAS extensions, C and
Fortran compilers are now aware of parallelism and now generate
message passing code that previously had been handled by the MPI
library.  In some instances the compiler is able to perform
optimizations that it is not able to do with a library-based scheme
like MPI \cite{preissl:2011:SC}.

%%% Would like to say something here about HPC as a language based solution to reducing code complexity

However, in large part these languages are largely syntatic sugar for
message passing and do not provide a substantial decrease in code
complexity when application performance is a
goal\cite{hasert:2011:EuroMPI}.  While skilled programmers in the HPC
community have become accustomed to the level of complexity of an MPI
program, the problem for programmers is that hardware is changing in
ways that increase the level of on-chip parallelism.  Indeed, the
current generation of machines could be the last to provide
homogeneous multi-core parallelism\cite{JORSjors.aw}.  For ultimate
performance, programmers must envision coding for heterogeneous
architectures with huge new levels of on-node parallelism, at the same
time they account for off-node parallelism.
Since languages have not evolved that allow the compiler to take up
this increase, the complexity for a programmer has necessarily
dramatically increased.  Unfortunately, considering the complexity of
large multiphysics code frameworks, it could take several person years
to retarget an application for \emph{each} new heterogeneous
platform\cite{DBLP:journals/corr/DubeyS13}.

A reasonable solution given todays language limitations is to use MPI for distributed memory
parallelism \emph{plus} OpenMP for on-chip parallelism.  This hybrid approach is taken by two
successful, large-scale AMR multiphysics code frameworks, FLASH and Chombo; currently however,
neither of these codes currently support hardware accelerators\cite{DBLP:journals/corr/DubeyS13}.
%%This is the solution proposed by Cray and PGI \cite{BOF_SC10} (currently Chapel does not target GPUs\cite{Brad?}).
Other choices for expressing on-chip parallelism are OpenCL\cite{OPENCL} and NVIDIA's CUDA.
Unfortunately, achieving high performance using either of these two languages can be a daunting
challenge and there are no guarantees that either language is suitable for future hardware
generations.

In this paper we examine a language-based paradigm that allows the
compiler to take on a larger portion of the code complexity burden.
Anyone programming in OpenCL or CUDA is aware that explicit loop
structures over array elements in a serial program are removed and
replaced by a kernel program that is run over all of the elements of
the input arrays.  We propose \emph{Locally Orientated Programming
extensions} (LOPe) to the Fortran and potentially C languages that
formally adopt this programming model.  The LOPe programming model
explicitly separates numerical complexity from parallel complexity ---
leaving much of the staging and assembling of data to the compiler ---
and is applicable to the solution of stencil-based algorithms that
provide data access patterns that are regular and spatially local.

