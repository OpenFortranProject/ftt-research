%
% I'm adding an outline here so I can see how everything fits together

%
% \title{Exploiting First-Class Arrays in Fortran for Accelerator Programming}
%

%
% \abstract
%

%
% \section{Introduction}
%    \subsection{Approach}
%    \subsection{Why Fortran?}
%    \subsection{Comparison to Other Languages}
%

%
% \section{Programming model}
%    \subsection{Fortran Syntax}   
%       \subsubsection*{Array notation}
%       \subsubsection*{Elemental functions}
%       \subsubsection*{Pure procedures}
%       \subsubsection*{Shift functions}
%       \subsubsection*{Regions}
%    \subsection{New Functions}
%    \subsection{Parallelism}
%    \subsection{Limitations}
%

%
% \section{Source-To-Source Transformations}
%    \subsection{ForOpenCL}
%       \subsubsection{array syntax}
%       \subsubsection{where construct}
%    \subsection{New functions}
%       \subsubsection{region}
%    \subsection{Static Analysis}
%       \subsubsection{Analysis not required}
%    \subsection{Simplifying Assumptions}
%

%
% \section{Shallow Water Model}
%    \subsection{Equations}
%

%
% \section{Performance}
%

%
% \section{Conclusions}
%


\documentclass[10pt, conference, compsocconf]{IEEEtran}

\usepackage{cite}
\usepackage{graphicx}
\usepackage[cmex10]{amsmath}
\usepackage{array}
\usepackage{url}


\title{Exploiting First-Class Arrays in Fortran for Accelerator Programming}

%%
%% reorder appropriately later
%%
\author{\IEEEauthorblockN{Matthew J. Sottile}
\IEEEauthorblockA{Galois, Inc.\\
421 SW 6th Ave. Suite 300 \\
Portland, OR 97204\\
Email: matt@galois.com}
\and
\IEEEauthorblockN{Craig E Rasmussen}
\IEEEauthorblockA{Los Alamos National Laboratory\\
CCS-7, MS B287\\
Los Alamos, NM\\
Email: rasmussn@lanl.gov}
}

\begin{document}

\maketitle

\begin{abstract}
Emerging architectures for high performance computing often are well
suited to a data parallel programming model.  This paper presents a
simple programming methodology based on existing languages and compiler
tools that allows programmers to take advantage of these systems.
We will work with the array features of Fortran 90 to show how this
infrequently exploited, standardized language feature is easily
transformed to lower level accelerator code.  Our transformations are
based on a mapping from Fortran 90 to C++ code with OpenCL extensions.
\end{abstract}

\section{Introduction}
\label{sec:intro}

This paper presents a compiler-level approach for targeting a single
program to multiple, fundamentally different low-level execution
models.  This technique allows the application programmer to adopt
a single high-level programming model without sacrificing performance.
We show that features of
Fortran 90 for data-parallel programming are well suited to automatic
transformation to generate code specifically tuned for different
hardware architectures using low-level programming models such as
OpenCL and CUDA.  For algorithms that can be easily expressed in terms
of whole array, data-parallel operations, writing code in Fortran and
transforming it automatically to specific low-level implementations
removes the burden from the programmer of working with tedious, error
prone, low-level tools.

In the ideal situation, application programmers would like to adopt a
programming model in which they write their application once and use
automated tools to retarget it to many architectures.  This has proven
to be very challenging historically due to the subtle balance between
high-level expressiveness of code and the performance of the
lower-level code that is emitted by a compiler.  This ideal high-level
model that programmers work with should emphasize readability,
maintainability, and close proximity in abstraction to the problem
being solved --- in this instance, the abstraction that we care about
are mathematical formulae.  It should not be corrupted with details of
specific target architectures solely for the purpose of single-system
performance.  For certain classes of applications, specifically those
that map onto a data-parallel programming model, we show that
Fortran 90 contains language features that encourage high-level
programming abstractions without sacrificing performance during
low-level code generation.

%1. Goal is to write once, transform many. Otherwise potentially reprogram for every architecture.
%2. Goal is to code for readability and maintainability, not performance
%3. This goal requires expression in a high-level language.
%4. But language must be simple enough for compiler to analyze.
%5. Thus ideal if language maps well to accelerator architectures.
%6. Data parallel constructs in Fortran 90 are chosen.
%7. Up to 65 times speed up measured on automatically transformed code.

%At the Los Alamos National Laboratory (LANL), as with many supercomputing
%facilities today, users have a wide variety of computer platforms from
%which to choose.  The most common platform is made up of clusters of
%compute nodes with standard multi-core processors.  An increasingly
%common feature is that some nodes also have accelerators that range
%from the IBM Cell processor (such as the LANL Roadrunner system) to a
%variety of GPUs from NVIDIA and AMD.  Some nodes have hardware with
%vector instructions and others do not.  The peak performance of these
%accelerated nodes often resides in the hundreds of gigaflops.

%The peak performance of the
%accelerated nodes range from 200+ GFlops for the IBM Cell/B.E.
%processor to XXX for NVIDIA Fermi. %% Fermi has another name

The performance that new accelerator architectures offer comes at a
cost, as processor architectures are trending toward multiple cores
with instances of integrated accelerator units (with user managed
memory) and less of a reliance on superscalar instruction level
parallelism and hardware managed memory hierarchies (such as
traditional caches).  These changes place a heavy burden on
application programmers as they are forced to adapt to the new
systems.  An especially challenging problem faced by application
programmers is not only how to program to these new architectures
(considering the massive scale of concurrency available), but also how
to design programs that are portable across the changing landscape of
computer architectures with unique memory systems and programming
models.  Can a programmer easily write one program that can run on a
conventional multicore CPU, graphics processing unit, Cell processor,
and one of many emerging many core architectures?  The fundamental
question we address is what programming model and language constructs
are best suited to span this set of new hardware designs.

%% NEW - CER
%%%We will examine how the existing data-parallel constructs in Fortran can be combined with coarrays or MPI to provide effectively a new parallel programming language, one that is evolutionary in nature and provides complete compatibility with existing applications and libraries.  Data parallelism is a high-level abstraction that is, at the same time, both easier to program and gives the compiler more leeway (if fully exploited) in retargeting a program to different computer architectures.

A common theme amongst new processors is the emphasis on data-parallel
programming.  This model is well suited to emerging architectures that
are based on either vector processing or massively parallel
collections of simple cores.  The recent CUDA and OpenCL programming
languages are are intended to support this programming model, as are
directive-based methods such as OpenMP or the Accelerator programming
model from the Portland Group~\cite{pgi10accelerator}.

%% Deleted by CER
%%A proposed approach for programming that is well suited to legacy applications and languages are language extensions or libraries that allow programmers to avoid adopting entirely new languages.  

The problem with many of these choices is that they expose too much
detail about the machine architecture to the programmer.  This is
particularly true of CUDA and OpenCL.  In CUDA, programmers must adapt
their codes to fit the threading model used by NVIDIA GPUs, while
OpenCL requires programmers to provide specially tuned versions of
their code for different classes of machine.  In both cases, the
programmer is responsible for explicitly managing memory,
including staging of data back and forth from the host CPU and the
accelerator device memory.  While these models have been attractive as
a method for early adopters to utilize these new architectures, they
are less attractive to programmers who do not have the time or
resources to manually port their code to every new architecture and
programming model that emerges.

%At this point in time it is not really possible to write once and run
%efficiently on the wide variety of computer platforms we have
%available.  For some classes of applications, we believe that this
%goal is possible using language constructs already present in a
%popular mainstream scientific programming language -- Fortran.  A
%common, long-standing tongue-in-cheek response to new language
%developments in the scientific and high performance computing
%community is that a new language will arise to answer the needs of new
%systems, and it will be called Fortran.  We believe that the work
%presented in this paper validates that notion -- we need a new
%language to work with, and that language is Fortran.

\subsection{Approach}

%This paper addresses the accelerator programming problem by examining
%features in Fortran that allow programmers to express algorithms at a
%very high level that can be easily transformed by a compiler to run
%efficiently on a wide variety of platforms.  In particular we consider
%computers based on GPUs and related accelerator processors.

We demonstrate that the array syntax of Fortran maps surprisingly well
onto GPUs when transformed to OpenCL kernels.  These Fortran language
features include pure and elemental functions and array constructs
like {\tt where} and {\tt cshift}.  In addition we add a few functions
that enable a program to be deployed on machines with a hierarchy of
processing elements, such as nodes employing GPU acceleration,
\emph{without requiring explicit declaration of parallelism within the
  program.}  In addition the program uses entirely standard Fortran so
it can be compiled for and executed on a single core without concurrency.
This work also is applicable to vendor-specific languages similar to
OpenCL such as the NVIDIA CUDA language.

%We provide (via Fortran interfaces in the ForOpenCL library) a
%mechanism to call the C OpenCL runtime and enable Fortran programmers
%to access OpenCL kernels.  
Transformations are supplied that provide a mechanism for converting
Fortran procedures written in the Fortran subset described in this
paper to OpenCL kernels.  We use the ROSE compiler
infrastructure\footnote{\url{http://www.rosecompiler.org/}} to
develop these transformations.  ROSE uses the Open Fortran
Parser\footnote{\url{http://fortran-parser.sf.net/}} to parse
Fortran 2008 syntax and can generate C-based OpenCL.  Since ROSE's
intermediate representation (IR) was constructed to represent multiple
languages, it is relatively straightforward to transform high-level
Fortran IR nodes to C OpenCL nodes.

%  Furthermore, the Fortran array syntax maps directly to one,
%two, and three-dimensional thread groups in OpenCL.

Transformations for arbitrary Fortran procedures are not attempted.
Furthermore, a mechanism to transform the calling site to
automatically invoke OpenCL kernels is not provided at this time.
While it is possible to accomplish this task within ROSE, it is
considered outside the scope of this paper.

We examine the performance of the Fortran data-parallel abstraction
when transformed to OpenCL to run on GPU architectures.  Since single
node performance is often given as a reason for not using
data-parallel constructs within Fortran, we consider the performance
of serial data-parallel codes compared with the usage of explicit loop
constructs.

We study automatic transformations and the performance for an
application example that is typical of many applications that are
based on finite-difference or finite-volume methods in computational
fluid dynamics (CFD).  The example described later in this paper is a
simple shallow water model in two dimensions using finite volume
methods.

An initial study was made for an important procedure in PAGOSA, a
non-research, production-grade code at LANL completely written in
data-parallel Fortran.  We investigated automatically transforming
this code to run on LANL's Petaflop Roadrunner computer (a hybrid
mixture of AMD Opterons and IBM Cell processors).  We demonstrated that a
source-to-source compiler can automatically vectorize and parallelize
a small section of this code for the Cell processor.  Preliminary results
showed a 9 times performance gain of the transformed code when compared
with the original serial version on a traditional single-core processor.

\subsection{Why Fortran?}

Fortran is the oldest high-level programming language in continuous
use since its introduction, and was developed to facilitate the
translation of math formulae into machine code. Fortran was the first
major language to use a compiler to translate from a high-level
program representation to assembly language. Due to its age, it
carries certain arcane baggage.  However with the introduction of
Fortran 90 (and later revisions to the standard), Fortran became a
truly modern programming language.  It is now modular and has many
object-oriented features.  Most importantly for this work, it now
includes a type system in which rich first-class array data types and
corresponding syntax are part of the language, something that
languages like C continue to lack\footnote{The recent Intel 12.0 C/C++
  compiler supports extensions that provide array notation for C/C++
  code, as detailed here: http://intel.ly/g454zp}.  Furthermore,
Fortran should be of interest to those studying parallel programming
because of its functional and data-parallel constructs and because of
the coarray notation introduced in Fortran 2008. Unlike languages like
C and C++, Fortran has become a truly parallel language with features
added to recent language standards.

%% - deleted by CER
%% that later programming languages have evolved away from. As a result, Fortran has fallen into disfavor in certain programming circles.  Modern versions of the language standardized in 1990, 1995, 2003, and 2008 have removed much of this legacy baggage, but these changes are not widely known.  Modern Fortran exhibits features that are similar to other modern programming languages, and does not mandate the use of legacy features from decades old, deprecated versions of the language.  Compilers for Fortran (like those for other languages that have removed features over time) can prohibit the use of archaic features such as fixed format code or constructs that have been removed from the language.

Yet, a likely question that one may pose is ``\emph{Why Fortran and
  not a more modern language like X?}''  The recent rise in interest
in concurrency and parallelism at the language level driven by
multicore CPUs and manycore accelerators has driven a number of new
language developments, both as novel languages and extensions on
existing ones.  For scientific users, new languages and language
extensions to use novel new architectures present a challenge: how do
developers effectively use them while avoiding rewriting code
and potentially growing dependent on a transient technology that will
vanish tomorrow?


%% NEW - CER
%So perhaps it is time to replace Fortran with yet another computer language.  The
%problems with replacing Fortran with an entirely new language are two fold:
%the economics of replacing the existing application base and the difficulty in
%obtaining programmer acceptance.  It is estimated that replacing a major
%production application at Los Alamos National Laboratory would cost between 50
%and 150 million dollars.  In terms of programmer acceptance, there is always
%the "chicken and egg problem": programmers won't use a new language until they
%can expect good performance across a variety of platforms, and compiler
%vendors can't afford to produce quality compilers until there is a reasonable
%expectation of a market.


%% NEW - CER
History has also shown that an investment in rewriting code does not
guarantee success either, as seen in an effort at LANL to modernize a
legacy Fortran code with the newer C++ POOMA framework.  This
massive overhaul effort led to a code that was both slower and less
flexible than the original Fortran \cite{basili08hpc}.  

%Similar
%experiences have occurred in the past, notably during the development
%of the functional SISAL language in the early 1990s.

Fortran is unique in that it has contained language features that are
well suited to modern architectures for a number of years.  This
should be unsurprising --- Fortran was a primary language used to
target systems such as the vector supercomputers and massively
parallel systems of the 1970s and 1980s.  These are the systems in
which architectural features were developed that have led to single
chip high performance architectures of interest today.  Given that
these new systems have features very similar to their predecessors, it
is clear that the language features within Fortran for them are still
relevant.

\subsection{Comparison to Other Languages}

A number of previous efforts have exploited data-parallel programming
at the language level to utilize novel architectures, particularly in
previous decades during the reign of vector and massively parallel
computers in the high performance computing world.  The origin of the
array syntax that was adopted in Fortran 90 can be found in the APL
language.  Fortran 90 differed from previous % \cite{iverson79apl}
extensions of Fortran in that parallelism within whole-array
operations was expressed at the expression level instead of via
parallelism within explicit DO-loops (such as within IVTRAN for the
Illiac IV).

The High Performance Fortran (HPF) extension of Fortran 90 was
proposed to add features to the language that would enhance the
ability of compilers to emit fast parallel code for distributed and
shared memory parallel computers\cite{koelbel94hpf}.  One of the
notable additions to the language in HPF was syntax to specify the
distribution of data structures amongst a set of parallel processors.
HPF also introduced an alternative looping construct to the
traditional DO-loop called {\tt FORALL} that was better suited for
parallel compilation.  An additional keyword, {\tt INDEPENDENT}, was
added to allow the programmer to indicate when the order of execution
of the program (such as a sequence of loop iterations) can be flexible
in order to allow parallel execution.  Interestingly, the parallelism
features introduced in HPF did not exploit the new array features
introduced in 1990 in any significant way, relying instead on explicit
loop-based parallelism.  This was likely in order to support parallel
programming that wasn't easily mapped onto a pure data-parallel model.

In some instances though, a purely data-parallel model is appropriate
for part or all of the major computations within a program.  One of
the systems where programmers relied heavily on higher level
operations instead of explicit looping constructs was the Thinking
Machines Connection Machine 5 (CM-5).  A common programming pattern
used on the CM-5 that we exploit in this paper was to write
whole-array operations from a global perspective in which computations
are expressed in terms of operations over the entire array instead of
a single local index.  The use of the array shift intrinsic functions
(like {\tt CSHIFT}) were used to build computations in which arrays
were combined by shifting the entire arrays instead of working based
on local offsets from single indices.  A simple 1D example is one in
which an element is replaced with the average of its own value with
that of its two direct neighbors.  Ignoring boundary indices that wrap
around, explicit indexing will result in a loop such as:

{\small
\begin{verbatim}
  do i = 2,(n-1)
    X(i) = (X(i-1) + X(i) + X(i+1)) / 3
  end do
\end{verbatim}
}

\noindent When shifts are employed, this can be expressed as:

{\small
\begin{verbatim}
  X = (cshift(X,-1) + X + cshift(X,1)) / 3
\end{verbatim}
}

Similar whole array shifting was used in higher dimensions for finite
difference codes within the computational physics community, especially
at Los Alamos for codes targeting the CM-5 system that resided there until the
late 1990s.  A body of research in compilation of stencil-based codes
that use shift operators targeting these systems is related to the
work we present here~\cite{stencil-compiler}.

The whole-array model was attractive because it deferred
responsibility for optimally implementing the computations to the
compiler.  Instead of relying on a compiler to infer parallelism from
a set of explicit loops, the choice for how to implement loops was
left entirely up to the tool.  Unfortunately, this had two side
effects that have limited broad acceptance of the whole-array
programming model in Fortran.  First, programmers must translate their
algorithms into a set of global operations.  Finite difference
stencils and similar computations are traditionally defined in terms
of offsets from some central index.  Shifting, while conceptually
analogous, can be awkward to think about for high dimensional stencils
with many points.  Second, the semantics of these operations are such
that all elements of an array operation are updated as if they were
updated simultaneously.  In a program where the programmer explicitly
manages arrays and loops, double buffering techniques and user managed
temporaries are used to maintain these semantics.  When the compiler
is responsible for managing this intermediate storage, it has
historically proven that they are inefficient and generate code that
requires far more temporary storage than really necessary.  This is
not a flaw of the language constructs, but a sign of the lack of
sophistication of the compilers with respect to their internal
analysis to determine how to optimally generate this intermediate
storage.

An interesting line of language research that grew out of the early work
with HPF was that associated with the ZPL language work at the University
of Washington~\cite{chamberlain04zpl}.  In ZPL, programmers adopt a similar
global view of computation over arrays, but define their computations based
on the local view of indices that participate in the update of each element of
an array.



\section{Comparison to Other Languages}

%% Matt you should provide an historical (and current) context here


Techniques based on whole-array programming being mapped to high performance,
parallel implementations are not new.  

Similar to ZPL, CM5 stencil compilers, CMFortran stuff.

%% cites: ZPL, CM5 stencil compilers POOMA, HPL OpenMP, CUDA

% put something here on: regions, array notation, ...

\section{Programming Model}

The static analysis and source-to-source transformations used are very
basic but require the programmer to use a subset of Fortran
that employs a data-parallel programming model.  In particular, it
encourages use of language features that were introduced and
standardized in the Fortran 90 language specification.  In this
section we describe the set of Fortran 90 features that our analysis
and transformation method are based on.

This paper uses four simple features of Fortran that form a language subset
that does not require any language extensions and can be easily transformed
to a lower-level implementation in either CUDA or OpenCL.

\subsubsection*{Array notation}

Fortran 90 introduced a rich array syntax that allows programmers to
write code that is in terms of whole arrays or subarrays, with data
parallel operators to compute on the arrays.  This has the benefit of
avoiding explicit looping in the code and maintaining a high-level
style that is closer to the original mathematical specification of the
problem being solved.  More importantly from a compilation
perspective, this defers decisions about how to implement these
whole-array operations to a compilation tool.  When faced with a novel
architecture such as modern GPUs that are ideally suited to data parallel
programming models, the fit between Fortran arrays and these systems is
quite clean.

%% NEW - CER
Arrays are first class citizens in Fortran as they are part of the language
type system rather than just a pointer to raw memory as in C. A Fortran array
contains metadata associated with it that describes the array's shape and
size. This metadata is lacking in other languages, such as C, C++, and Java.
Even in the case where multidimensional array data structures are available
(such as in C++ or Java), they exist outside the language standard, limiting
the ability of the compiler to analyze them fully. The existence of this
metadata allows the use of array notation whereby explicit indexing of arrays
is not required. For example, if {\tt A}, {\tt B}, and {\tt C} are all arrays
of the same rank and shape and {\tt s} is a scalar, then the statement

\begin{verbatim}

 C = A + s*B

\end{verbatim}

results in the sum of elements of {\tt A} and {\tt s} times the elements of
{\tt B} being stored in the corresponding elements of {\tt C}. The first
element of {\tt C} will contain the value of the first element of {\tt A}
added to the first element of {\tt c*B}.  Note that no explicit iteration over
array indices is needed and that the individual operators, plus, times, and
assignment are applied by the compiler to individual elements of the arrays
independently.  Thus the compiler is able to spread the computation in the
example across any hardware threads under its control.

%% NEW - CER
%%% The use of array notation allows one to program in a data-parallel subset of Fortran. This style of programming makes use of array notation and pure and elemental functions to operate on array elements.

\subsubsection*{Elemental functions}

An elemental function consumes and produces scalar values, but can be applied
to variables of array type such that the function is applied to each and every
element of the array.  This allows programmers to avoid explicit looping and
instead simply state that they intend a function to be applied to every
element of an array in parallel, deferring the choice of implementation
technique to the compiler.  Elemental functions are intended to be used for
data parallel programming, and as a result must be side effect free (or,
\emph{pure}).

%% NEW - CER
%%% Because elemental functions return scalar values and are free from side effects, the compiler is free to distribute the computation over any hardware processing elements available to it, such as the multiple cores and vector units on an Intel or AMD processor or the Synergistic Processing Units on an IBM Cell processor.

%% NEW - CER
An example of shown above using array syntax could be refactored to use
the elemental function,

\begin{verbatim}
   pure elemental real function elem_example(a, b, s)
      real, intent(in) :: a, b, s
      elem_example = a + s*b
   end function
\end{verbatim}

and called with

\begin{verbatim}
   C = elem_example(A, B, s)
\end{verbatim}

Note that while {\tt elem\_example} is defined in terms of purely scalar
quantities, it can be \emph{applied} to arrays as shown.

While this may seem like a trivial example, such simple functions may be
composed with other elemental functions to perform powerful computations,
especially when applied to arrays.  ForOpenCL transforms elemental functions to inline OpenCL
functions.  Thus there is no penalty for usage of elemental functions and provide
a convient mechanism to express algorithms in simpler segments.

Also note that the plus and times operators shown in the data-statement
example can be seen as an elemental functions, as plus and times are defined
in terms of scalars but can be applied to whole arrays and can return an array
result.

\subsubsection*{Pure procedures}

Pure procedures, like elemental functions, must be free of side effects.
(athough it may change the value of array elements in an array argument,
unlike elemental arguments which must have the intent(in) attribute).
The absence of side effects removes ordering constraints from the compiler
allowing it to invoke pure functions out of order.  Procedures and functions
of this sort are also common in pure functional languages like Haskell,
and are exploited by compilers in order to emit parallel code automatically
due to their suitability for compiler-level analysis.

Since pure procedures don't have side effects they a candidates for running on
accelerators in OpenCL.  Currently ForOpenCL transforms pure procedures to
OpenCL kernels that emph{do not} call other procedures, except for elemental
functions.

\subsubsection*{Shift functions}

Many array-based algorithms require the same operation to be performed
on each element of the array using the value of that element and some small
set of neighboring cells.  Often programmers implement these operations that
are local to each element within the inner-loop of a set of nested FOR-
or DO-loops using offsets relative to the current array index.  An alternative
to this local-view of the algorithm is to take a global view and write the
algorithm in terms of the whole array.  For example, consider a 1D array in
which we wish to subtract the $(i-1)$th element from the $i$th for all
elements.  One way to look at this is that we are subtracting the entire array
shifted by one element from itself.

Fortran provides a set of shifting operators that allow programmers to
define operations based on shifted arrays.  These intrinsic operators take
an array, a dimension, and the amount by which it should be shifted (using
the sign to indicate direction).  By defining operations on entire arrays
based on a global view of them shifted relative to each other, programmers can
avoid explicit looping and potentially tricky index arithmetic.  Furthermore,
analysis of the extent of the set of shifted arrays in a given expression
allows analysis tools to determine the amount of temporary or buffer storage
necessary to hold intermediate values during whole array operations.  With
explicit loops, programmers must maintain this temporary storage manually.

\subsubsection*{Regions}

Borrowing from ZPL, we introduce the concept of regions to Fortran.  In
Fortran, regions are expressed as pointers to a subsection of an existing array.
Regions may refer to an interior portion of an array or the entire array.

The use of regions implies the use of the halo (or ghost) cell pattern 
where the size of an array is increased to provide extra array elements
surrounding the interior portion of the array.  
%parlab.eecs.berkeley.edu/wiki/_media/patterns/ghostcell.pdf

Regions are similar to the shift operator as they can be used to reference
portions of the array that are shifted with respect to the interior portion.
However, unlike the shift operator, regions are not expressed in terms of
boundary conditions and thus don't explicitly \emph{require} a knowledge of
not the application of boundary conditions locally.  Thus, as will be shown
below, regions are more suitable for usage by OpenCL thread groups which
access only local subsections of an array stored in global memory.

Unfortunately, as regions functions return pointers to arrays, they cannot
be used in pure procedures in current Fortran.  We relax this constraint in
ForOpenCL for region pointers only.

\subsection{New Functions}

% TODO - Craig - describe these functions
%   region
%   interior (can be written in terms of regions) but also specifies the thread
%       group domain)
%   more ...
%


\subsection{Parallelism}

% TODO - Matt - do you want a try at this.

% need to describe processor hierarchy of nodes and accelerators
% need to describe memory hierarchy as well in regards to global array view,
% node array view (with a copy is GPU global memory) and local GPU memory which
% further subdivides the node array memory

% figures on slicing and dicing of memory region go here


There are several advantages to this style of programming using array syntax, shifts, regions,
and pure and elemental functions:

\begin{itemize}
	\item There are no loops or index variables to keep track of.  Off by one
index errors are a common programming mistake.
        \item The written code is closer to the algorithm, easier to understand, and is usually substantially shorter.
	\item Semantically the intrinsic functions return arrays by value (though not regions).  This is usually what the algorithm requires.
	\item Because pure and elemental function are free from side effects, it is easier for a compiler to schedule the work to be done in parallel.
\end{itemize}

%% NEW - CER
An example of this style of programming in Fortran is shown in

\begin{verbatim}

  Bz = Bz + dt * (cshift(Ex,dim=2,shift=+1) - Ex) / dy  &
          - dt * (cshift(Ey,dim=1,shift=+1) - Ey) / dx

\end{verbatim}

%% NEW - CER
This example is a solution to Maxwell's equations for the $z$ component of the
magnetic field using Fortran array syntax.  Note that there are no explicit
loops in this example.  The operators {\tt +}, {\tt -}, and {\tt *} are
applied to all of the elements of the three-dimensional arrays {\tt Bz}, {\tt
  Ex}, and {\tt Ey}, individually.  This is why data parallelism has been
called collection-oriented programming by
Blelloch~\cite{blelloch90,rajopadhyedidlacs}.  As the {\tt cshift} function
and the array-valued expressions all semantically returns a value, this style
of programming is also similar to functional programming (or value-oriented
programming). %%~\cite{simonpeytonjones}).  The heart of the solution to
Maxwell's equations is the statement shown in Listing~\ref{lst:dpexample} and
five similar, simple equations.

%% NEW - CER
%%% The power in this notation is that the compiler is free to distribute the computation in these expressions over any hardware processing elements available to it, such as the vector units on an Intel or AMD processor or the Synergistic Processing Units on an IBM Cell processor.  If the arrays are declared as coarrays, this includes spreading the computation over \emph{all of the nodes in   a cluster as well.}  In this case, communication occurs within the \texttt{cshift} functions, though the compiler is free to overlap communication and computation by scheduling the communication early and performing the computation on the interior of the arrays while waiting for the communication to complete.

%% NEW - CER
Complete and very concise and elegant programs can be built with procedures
similar to the example shown above. To aid this effort, Fortran
supplies intrinsic functions like the array constructors (\texttt{CSHIFT},
\texttt{EOSHIFT}, \texttt{MERGE}, \texttt{TRANSPOSE}, ...), the array location
functions (\texttt{MAXLOC} and \texttt{MINLOC}), and the array reduction
functions (\texttt{ANY}, \texttt{COUNT}, \texttt{MINVAL}, \texttt{SUM},
\texttt{PRODUCT}, ...).  To this set we add region functions described above.

%% NEW - CER
This style of programming meets the requirements we have set for a programming
model for developing applications suitable for acceleration.  It allows the programmer to
program at a very-high level of abstraction while providing the compiler with
maximum flexibility in targeting the application for a particular hardware
architecture.  The data parallel programming model simultaneously meets the
seemingly conflicting goals of maintainability, portability, and performance.

%% NEW - CER
Unfortunately, this style of programming has never really caught on because
when Fortran 90 was first introduced, performance was relatively poor and thus
programmers shied away from using array syntax (even now, some are actively
counseling against its usage because of performance issues~\cite{Cray-SC07}).
Thus the Fortran community was caught in a classic ``chicken-and-egg''
conundrum: (1) programmers didn't use it because it was slow; and (2)
compilers vendors didn't improve it because programmers didn't use it.

%% NEW - CER
A goal of this paper is to demonstrate that parallel programs written in
this style of Fortran are maintainable and can achieve good performance on a
variety of processor architectures, including possibly, future terascale
architectures from Intel (such as the XXX architecture).

\subsection{Limitations}

Only Fortran procedures are transformed into OpenCL kernels.  The programmer
must currently explicitly call these kernels from Fortran using the ForOpenCL
library described below.  It is also possible using ROSE to modify the calling
site so that the entire program can be transformed but this functionality is
outside the scope of this paper.  Here we specifically examine transforming
Fortran procedures to OpenCL kernels.

Only elemental functions may be called from kernel functions.  These include
fortran functions that have an OpenCL analog and user-defined elemental functions.

Array sizes must be multiples of the local kernel size,
{\tt get\_local\_size(0)*get\_local\_size(1)}.  This may be relaxed in the future.


\section{Source-To-Source Transformations}

% TODO - craig - I need to go through this section and replace PetaVision
% with the shallow water code

\subsection{ForOpenCL}

ForOpenCL is library of Fortran modules.  It contains Fortran 2003 interface
descriptions that allow language interoperability with the C OpenCL runtime.

\section{Transformation examples}

This sections show short Fortran code examples and OpenCL equivalent.
The notation uses uppercase for arrays and lowercase for scalar quantities.
Interior array subsections are denoted by an i succeeding the array, e.g. Ai
is the interior region of array A.

\subsubsection{array syntax}

PetaVision updates the action potential with the statement,

\begin{verbatim}
   V = V - iA*(V - v_rest)                  !! Fortran

   V[l] = V[l] - iA[lex]*(V[l] - v_rest);   // OpenCL equivalent
\end{verbatim}

\subsubsection{where construct}

A neuron in PetaVision fires (activity is set to 1)
whenever the action potential is greater than some threshold.
This is easily expressed with a where construct.

\begin{verbatim}
   where (V > Vth)                        !! Fortran
      iA = 1
   elsewhere
      iA = 0
   end where

   iA[lex] = (V[l] > Vth[l]) ? 1 : 0;     // OpenCL equivalent
\end{verbatim}

\subsection{New functions}

%\subsubsection{transfer_halo}
\subsubsection{region}

One of the state variables in the shallow water code is H, effectively the
height of the water.  This variable has a halo (ghost-cell region) surrounding
the interior of the grid to handle boundary conditions.  When using
MPI, the halo regions contains values from adjacent processors that must be
updated with new values at each time step.  This is accomplished with the
{\tt transfer\_halo} function.  The shallow water code assumes a five point stencil
so the state variables are extended by 2 in each dimension (e.g., by one to the
left, right, up, and down).

\begin{verbatim}
halo = [1,1,1,1]
iH => region(H, halo, transfer=false)
\end{verbatim}

\section{Static Analysis}

1. Dependence analysis to insert thread group barriers

\begin{verbatim}
   barrier(CLK_LOCAL_MEM_FENCE)
\end{verbatim}

2. subsection variables associated with array variables

\subsection{Analysis not required}

1. loop fusion
2. removal of array temporaries (shifts)

\section{Performance}

\section{Conclusions}

%% Explain what we have accomplished (copied from proposal)

%% NEW - CER
The sheer complexity of programming for clusters of many or multi-core
processors with tens of millions threads of execution make the simplicity of
the data parallel model attractive.  Furthermore, the increasing complexity of
todays applications (especially when convolved with the increasing complexity
of the hardware) and the need for portability across hardware architectures
make a higher-level and simpler programming model like data parallel
attractive.

%% NEW - CER
The goal of this work has been to exploit source-to-source transformations that
allow programmers to develop and maintain programs at a high-level of
abstraction, without coding to a specific hardware architecture.
Furthermore these transformations allow multiple hardware architectures
to be targeted without changing the high-level source.  It also removes the
necessity for application programmers to understand details of the accelerator
architecture or to know OpenCL.

% Summarize results


\section{Junk}

\cite{chamberlain04zpl, roth97stencils}

\bibliographystyle{IEEEtran}
\bibliography{IEEEabrv,foropencl}

\end{document}
