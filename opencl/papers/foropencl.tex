\documentclass[10pt, conference, compsocconf]{IEEEtran}

\usepackage{cite}
\usepackage{graphicx}
\usepackage[cmex10]{amsmath}
\usepackage{array}
\usepackage{url}


\title{Exploiting First-Class Arrays in Fortran for Accelerator Programming}
\author{Awesome Fortran Programmers}

\begin{document}

\maketitle

\begin{abstract}
I am abstract.
\end{abstract}



\section{Introduction}

This paper presents a compiler-level approach for targeting a single
program to multiple fundamentally different low level execution and
programming models while allowing the application programmer to adopt
a single high level programming model.

1. Goal is to write once, transform many. Otherwise potentially reprogram for every architecture.
2. Goal is to code for readability and maintainability, not performance
3. This goal requires expression in a high-level language.
4. But language must be simple enough for compiler to analyze.
5. Thus ideal if language maps well to accelerator architectures.
6. Data parallel constructs in Fortran 90 are chosen.
7. Up to 65 times speed up measured on automatically transformed code.

At Los Alamos we have a wide variety of computer platforms from which
to choose.  The common platform is made up of cluster of compute nodes
with standard multi-core processors.  Some nodes have accelerators
that range from the IBM Cell processor (Roadrunner) to a variety of
GPUs from NVIDIA and AMD.  Some nodes have have hardware with vector
instructions and others do not.  An important question is what programming
model and what languages are best suited for these platforms.

A data parallel programming model is suited for accelerators....

One possibility is to adopt a new language like CUDA or Chapel...
Another possibility is to use a mature language with OpenMP directives
or write OpenCL kernels.

The problem is that many of the language choices expose too much detail
about the machine architecture.  This is particularly true of CUDA and OpenCL.

At this point in time it is not really possible to write once and
run efficiently on the wide variety of computer platforms we have
available.

This papers addresses this issue by examining features in Fortran that
allow programmers to express algorithms at a very high level but yet
can be transformed by a compiler to run efficiently on a wide variety
of platforms.  In particular we consider nodes of GPUs.

We demonstrate that the array syntax of Fortran maps surprisingly well
onto GPUs when transformed to OpenCL kernels.  These Fortran language
features include pure and elemental functions and array constructs
like where and shift.  In addition we add a few functions that enable
a program to be deployed on machines with a hierarchy of processing
elements, such as nodes employing GPU acceleration, \emph{without
  requiring explicit declaration of parallelism within the program.}
In addition the program uses entirely standard Fortran so it can be on
a single core without concurrency.

We study automatic transformations of Fortran for two applications examples.
The first is a simple shallow water model in two dimensions using
finite volume methods with .... for time update.  The second is based
on the primary compute kernels from PetaVision, a high-performance, 
neural science framework that is used in models of visual cortex.

\subsection{Why Fortran?}

A likely question that one may pose is ``\emph{Why Fortran and not a
  more modern language like X?}''  The recent rise in interest in
concurrency and parallelism at the language level driven by multicore
CPUs and manycore accelerators has driven a number of new language
developments, both as novel languages and extensions on existing ones.

For scientific users, these new languages and language extensions
present a challenge: how do developers effectively use them while
avoiding rewriting their code and potentially growing dependent on a
transient technology that will vanish tomorrow?

Fortran is unique in that it has contained language features that are
well suited to modern architectures for a number of years.  This
should be unsurprising --- Fortran was a primary language used to
target systems such as the vector supercomputers and massively
parallel systems of the 1970s and 1980s.  These are the systems in
which architectural features were developed that have led to single
chip high performance architectures of interest today.  Given that
these new systems have features very similar to their predecessors, it
is clear that the language features within Fortran for them are still
relevant.

\section{The programming model and Fortran-language subset}

The static analysis and source-to-source transformations used are very
basic and simply require the programmer to use a subset of Fortran
that employs a data-parallel programming model.  In particular, it
encourages use of language features that were introduced and
standardized in the Fortran 90 language specification.  In this
section we describe the set of Fortran 90 features that our analysis
and transformation method are based on.

Fortran 90 introduced a rich array syntax that allows programmers to
write code that is in terms of whole arrays or subarrays, with data
parallel operators to compute on the arrays.  This has the benefit of
avoiding explicit looping in the code and maintaining a high-level
style that is closer to the original mathematical specification of the
problem being solved.  More importantly from a compilation
perspective, this defers decisions about how to implement these
whole-array operations to a compilation tool.  When faced with a novel
architecture such as modern GPUs that are ideally suited to data parallel
programming models, the fit between Fortran arrays and these systems is
quite clean.

This paper uses four simple features of Fortran that form a language subset
that does not require any language extensions and can be easily transformed
to a lower level implementation in either CUDA or OpenCL.

\begin{enumerate}
\item Elemental functions
\item Pure procedures
\item Shift operators
\item Array notation
\end{enumerate}

\subsection{Relationship to other languages}

Similar to ZPL, CM5 stencil compilers, CMFortran stuff.

%% cites: ZPL, CM5 stencil compilers

% put something here on:
%
% shifts
% array notation
% 

\subsubsection{elemental functions}

\subsubsection{pure procedures}

Mention that don't require pure procedures as region functions are not pure.
But programmers should otherwise think in terms of writing pure procedures.

\subsection{Limitations}

Only Fortran procedures are transformed into OpenCL kernels.  The programmer
must currently explicitly call these kernels from Fortran using the ForOpenCL
library described below.  It is also possible using ROSE to modify the calling
site so that the entire program can be transformed but this functionality is
outside the scope of this paper.  Here we specifically examine transforming
Fortran procedures to OpenCL kernels.

Only elemental functions may be called from kernel functions.  These include
fortran functions that have an OpenCL analog and user-defined elemental functions.

Array sizes must be multiples of the local kernel size,
{\tt get\_local\_size(0)*get\_local\_size(1)}.  This may be relaxed in the future.

\subsection{ForOpenCL}

ForOpenCL is library of Fortran modules.  It contains Fortran 2003 interface
descriptions that allow language interoperability with the C OpenCL runtime.

\section{Transformation examples}

This sections show short Fortran code examples and OpenCL equivalent.
The notation uses uppercase for arrays and lowercase for scalar quantities.
Interior array subsections are denoted by an i preceeding the array, e.g. iA
is the interior subset of array A.

\subsubsection{array syntax}

PetaVision updates the action potential with the statement,

\begin{verbatim}
   V = V - iA*(V - v_rest)                  !! Fortran

   V[l] = V[l] - iA[lex]*(V[l] - v_rest);   // OpenCL equivalent
\end{verbatim}

\subsubsection{where construct}

A neuron in PetaVision fires (activity is set to 1)
whenever the action potential is greater than some threshold.
This is easily expressed with a where construct.

\begin{verbatim}
   where (V > Vth)                        !! Fortran
      iA = 1
   elsewhere
      iA = 0
   end where

   iA[lex] = (V[l] > Vth[l]) ? 1 : 0;     // OpenCL equivalent
\end{verbatim}

\subsection{New functions}

%\subsubsection{transfer_halo}
\subsubsection{region}

One of the state variables in the shallow water code is H, effectively the
height of the water.  This variable has a halo (ghost-cell region) surrounding
the interior of the grid to handle boundary conditions.  When using
MPI, the halo regions contains values from adjacent processors that must be
updated with new values at each time step.  This is accomplished with the
{\tt transfer\_halo} function.  The shallow water code assumes a five point stencil
so the state variables are extended by 2 in each dimension (e.g., by one to the
left, right, up, and down).

\begin{verbatim}
halo = [1,1,1,1]
iH => region(H, halo, transfer=false)
\end{verbatim}

\section{Static Analysis}

1. Dependence analysis to insert thread group barriers

\begin{verbatim}
   barrier(CLK_LOCAL_MEM_FENCE)
\end{verbatim}

2. subsection variables associated with array variables

\subsection{Analysis not required}

1. loop fusion
2. removal of array temporaries (shifts)

\section{Performance}


\end{document}
