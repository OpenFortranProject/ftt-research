\section{Source-To-Source Transformations}

% TODO - craig - I need to go through this section and replace PetaVision
% with the shallow water code

\subsection{ForOpenCL}

ForOpenCL is library of Fortran modules.  It contains Fortran 2003 interface
descriptions that allow language interoperability with the C OpenCL runtime.

\section{Transformation examples}

This sections show short Fortran code examples and OpenCL equivalent.
The notation uses uppercase for arrays and lowercase for scalar quantities.
Interior array subsections are denoted by an i succeeding the array, e.g. Ai
is the interior region of array A.

\subsubsection{array syntax}

PetaVision updates the action potential with the statement,

\begin{verbatim}
   V = V - iA*(V - v_rest)                  !! Fortran

   V[l] = V[l] - iA[lex]*(V[l] - v_rest);   // OpenCL equivalent
\end{verbatim}

\subsubsection{where construct}

A neuron in PetaVision fires (activity is set to 1)
whenever the action potential is greater than some threshold.
This is easily expressed with a where construct.

\begin{verbatim}
   where (V > Vth)                        !! Fortran
      iA = 1
   elsewhere
      iA = 0
   end where

   iA[lex] = (V[l] > Vth[l]) ? 1 : 0;     // OpenCL equivalent
\end{verbatim}

\subsection{New functions}

%\subsubsection{transfer_halo}
\subsubsection{region}

One of the state variables in the shallow water code is H, effectively the
height of the water.  This variable has a halo (ghost-cell region) surrounding
the interior of the grid to handle boundary conditions.  When using
MPI, the halo regions contains values from adjacent processors that must be
updated with new values at each time step.  This is accomplished with the
{\tt transfer\_halo} function.  The shallow water code assumes a five point stencil
so the state variables are extended by 2 in each dimension (e.g., by one to the
left, right, up, and down).

\begin{verbatim}
halo = [1,1,1,1]
iH => region(H, halo, transfer=false)
\end{verbatim}
