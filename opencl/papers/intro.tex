\section{Introduction}
\label{sec:intro}

This paper presents a compiler-level approach for targeting a single
program to multiple, fundamentally different low-level execution
models.  This technique allows the application programmer to adopt
a single high-level programming model without sacrificing performance.
We show that features of
Fortran 90 for data-parallel programming are well suited to automatic
transformation to generate code specifically tuned for different
hardware architectures using low-level programming models such as
OpenCL and CUDA.  For algorithms that can be easily expressed in terms
of whole array, data-parallel operations, writing code in Fortran and
transforming it automatically to specific low-level implementations
removes the burden from the programmer of working with tedious, error
prone, low-level tools.

In the ideal situation, application programmers would like to adopt a
programming model in which they write their application once and use
automated tools to retarget it to many architectures.  This has proven
to be very challenging historically due to the subtle balance between
high-level expressiveness of code and the performance of the
lower-level code that is emitted by a compiler.  This ideal high-level
model that programmers work with should emphasize readability,
maintainability, and close proximity in abstraction to the problem
being solved --- in this instance, the abstraction that we care about
are mathematical formulae.  It should not be corrupted with details of
specific target architectures solely for the purpose of single-system
performance.  For certain classes of applications, specifically those
that map onto a data-parallel programming model, we show that
Fortran 90 contains language features that encourage high-level
programming abstractions without sacrificing performance during
low-level code generation.

%1. Goal is to write once, transform many. Otherwise potentially reprogram for every architecture.
%2. Goal is to code for readability and maintainability, not performance
%3. This goal requires expression in a high-level language.
%4. But language must be simple enough for compiler to analyze.
%5. Thus ideal if language maps well to accelerator architectures.
%6. Data parallel constructs in Fortran 90 are chosen.
%7. Up to 65 times speed up measured on automatically transformed code.

%At the Los Alamos National Laboratory (LANL), as with many supercomputing
%facilities today, users have a wide variety of computer platforms from
%which to choose.  The most common platform is made up of clusters of
%compute nodes with standard multi-core processors.  An increasingly
%common feature is that some nodes also have accelerators that range
%from the IBM Cell processor (such as the LANL Roadrunner system) to a
%variety of GPUs from NVIDIA and AMD.  Some nodes have hardware with
%vector instructions and others do not.  The peak performance of these
%accelerated nodes often resides in the hundreds of gigaflops.

%The peak performance of the
%accelerated nodes range from 200+ GFlops for the IBM Cell/B.E.
%processor to XXX for NVIDIA Fermi. %% Fermi has another name

The performance that new accelerator architectures offer comes at a
cost, as processor architectures are trending toward multiple cores
with instances of integrated accelerator units (with user managed
memory) and less of a reliance on superscalar instruction level
parallelism and hardware managed memory hierarchies (such as
traditional caches).  These changes place a heavy burden on
application programmers as they are forced to adapt to the new
systems.  An especially challenging problem faced by application
programmers is not only how to program to these new architectures
(considering the massive scale of concurrency available), but also how
to design programs that are portable across the changing landscape of
computer architectures with unique memory systems and programming
models.  Can a programmer easily write one program that can run on a
conventional multicore CPU, graphics processing unit, Cell processor,
and one of many emerging many core architectures?  The fundamental
question we address is what programming model and language constructs
are best suited to span this set of new hardware designs.

%% NEW - CER
%%%We will examine how the existing data-parallel constructs in Fortran can be combined with coarrays or MPI to provide effectively a new parallel programming language, one that is evolutionary in nature and provides complete compatibility with existing applications and libraries.  Data parallelism is a high-level abstraction that is, at the same time, both easier to program and gives the compiler more leeway (if fully exploited) in retargeting a program to different computer architectures.

A common theme amongst new processors is the emphasis on data-parallel
programming.  This model is well suited to emerging architectures that
are based on either vector processing or massively parallel
collections of simple cores.  The recent CUDA and OpenCL programming
languages are are intended to support this programming model, as are
directive-based methods such as OpenMP or the Accelerator programming
model from the Portland Group~\cite{pgi10accelerator}.

%% Deleted by CER
%%A proposed approach for programming that is well suited to legacy applications and languages are language extensions or libraries that allow programmers to avoid adopting entirely new languages.  

The problem with many of these choices is that they expose too much
detail about the machine architecture to the programmer.  This is
particularly true of CUDA and OpenCL.  In CUDA, programmers must adapt
their codes to fit the threading model used by NVIDIA GPUs, while
OpenCL requires programmers to provide specially tuned versions of
their code for different classes of machine.  In both cases, the
programmer is responsible for explicitly managing memory,
including staging of data back and forth from the host CPU and the
accelerator device memory.  While these models have been attractive as
a method for early adopters to utilize these new architectures, they
are less attractive to programmers who do not have the time or
resources to manually port their code to every new architecture and
programming model that emerges.

%At this point in time it is not really possible to write once and run
%efficiently on the wide variety of computer platforms we have
%available.  For some classes of applications, we believe that this
%goal is possible using language constructs already present in a
%popular mainstream scientific programming language -- Fortran.  A
%common, long-standing tongue-in-cheek response to new language
%developments in the scientific and high performance computing
%community is that a new language will arise to answer the needs of new
%systems, and it will be called Fortran.  We believe that the work
%presented in this paper validates that notion -- we need a new
%language to work with, and that language is Fortran.

\subsection{Approach}

%This paper addresses the accelerator programming problem by examining
%features in Fortran that allow programmers to express algorithms at a
%very high level that can be easily transformed by a compiler to run
%efficiently on a wide variety of platforms.  In particular we consider
%computers based on GPUs and related accelerator processors.

We demonstrate that the array syntax of Fortran maps surprisingly well
onto GPUs when transformed to OpenCL kernels.  These Fortran language
features include pure and elemental functions and array constructs
like {\tt where} and {\tt cshift}.  In addition we add a few functions
that enable a program to be deployed on machines with a hierarchy of
processing elements, such as nodes employing GPU acceleration,
\emph{without requiring explicit declaration of parallelism within the
  program.}  In addition the program uses entirely standard Fortran so
it can be compiled for and executed on a single core without concurrency.
This work also is applicable to vendor-specific languages similar to
OpenCL such as the NVIDIA CUDA language.

%We provide (via Fortran interfaces in the ForOpenCL library) a
%mechanism to call the C OpenCL runtime and enable Fortran programmers
%to access OpenCL kernels.  
Transformations are supplied that provide a mechanism for converting
Fortran procedures written in the Fortran subset described in this
paper to OpenCL kernels.  We use the ROSE compiler
infrastructure\footnote{\url{http://www.rosecompiler.org/}} to
develop these transformations.  ROSE uses the Open Fortran
Parser\footnote{\url{http://fortran-parser.sf.net/}} to parse
Fortran 2008 syntax and can generate C-based OpenCL.  Since ROSE's
intermediate representation (IR) was constructed to represent multiple
languages, it is relatively straightforward to transform high-level
Fortran IR nodes to C OpenCL nodes.

%  Furthermore, the Fortran array syntax maps directly to one,
%two, and three-dimensional thread groups in OpenCL.

Transformations for arbitrary Fortran procedures are not attempted.
Furthermore, a mechanism to transform the calling site to
automatically invoke OpenCL kernels is not provided at this time.
While it is possible to accomplish this task within ROSE, it is
considered outside the scope of this paper.

We examine the performance of the Fortran data-parallel abstraction
when transformed to OpenCL to run on GPU architectures.  Since single
node performance is often given as a reason for not using
data-parallel constructs within Fortran, we consider the performance
of serial data-parallel codes compared with the usage of explicit loop
constructs.

We study automatic transformations and the performance for an
application example that is typical of many applications that are
based on finite-difference or finite-volume methods in computational
fluid dynamics (CFD).  The example described later in this paper is a
simple shallow water model in two dimensions using finite volume
methods.

An initial study was made for an important procedure in PAGOSA, a
non-research, production-grade code at LANL completely written in
data-parallel Fortran.  We investigated automatically transforming
this code to run on LANL's Petaflop Roadrunner computer (a hybrid
mixture of AMD Opterons and IBM Cell processors).  We demonstrated that a
source-to-source compiler can automatically vectorize and parallelize
a small section of this code for the Cell processor.  Preliminary results
showed a 9 times performance gain of the transformed code when compared
with the original serial version on a traditional single-core processor.

\subsection{Why Fortran?}

Fortran is the oldest high-level programming language in continuous
use since its introduction, and was developed to facilitate the
translation of math formulae into machine code. Fortran was the first
major language to use a compiler to translate from a high-level
program representation to assembly language. Due to its age, it
carries certain arcane baggage.  However with the introduction of
Fortran 90 (and later revisions to the standard), Fortran became a
truly modern programming language.  It is now modular and has many
object-oriented features.  Most importantly for this work, it now
includes a type system in which rich first-class array data types and
corresponding syntax are part of the language, something that
languages like C continue to lack\footnote{The recent Intel 12.0 C/C++
  compiler supports extensions that provide array notation for C/C++
  code, as detailed here: http://intel.ly/g454zp}.  Furthermore,
Fortran should be of interest to those studying parallel programming
because of its functional and data-parallel constructs and because of
the coarray notation introduced in Fortran 2008. Unlike languages like
C and C++, Fortran has become a truly parallel language with features
added to recent language standards.

%% - deleted by CER
%% that later programming languages have evolved away from. As a result, Fortran has fallen into disfavor in certain programming circles.  Modern versions of the language standardized in 1990, 1995, 2003, and 2008 have removed much of this legacy baggage, but these changes are not widely known.  Modern Fortran exhibits features that are similar to other modern programming languages, and does not mandate the use of legacy features from decades old, deprecated versions of the language.  Compilers for Fortran (like those for other languages that have removed features over time) can prohibit the use of archaic features such as fixed format code or constructs that have been removed from the language.

Yet, a likely question that one may pose is ``\emph{Why Fortran and
  not a more modern language like X?}''  The recent rise in interest
in concurrency and parallelism at the language level driven by
multicore CPUs and manycore accelerators has driven a number of new
language developments, both as novel languages and extensions on
existing ones.  For scientific users, new languages and language
extensions to use novel new architectures present a challenge: how do
developers effectively use them while avoiding rewriting code
and potentially growing dependent on a transient technology that will
vanish tomorrow?


%% NEW - CER
%So perhaps it is time to replace Fortran with yet another computer language.  The
%problems with replacing Fortran with an entirely new language are two fold:
%the economics of replacing the existing application base and the difficulty in
%obtaining programmer acceptance.  It is estimated that replacing a major
%production application at Los Alamos National Laboratory would cost between 50
%and 150 million dollars.  In terms of programmer acceptance, there is always
%the "chicken and egg problem": programmers won't use a new language until they
%can expect good performance across a variety of platforms, and compiler
%vendors can't afford to produce quality compilers until there is a reasonable
%expectation of a market.


%% NEW - CER
History has also shown that an investment in rewriting code does not
guarantee success either, as seen in an effort at LANL to modernize a
legacy Fortran code with the newer C++ POOMA framework.  This
massive overhaul effort led to a code that was both slower and less
flexible than the original Fortran \cite{basili08hpc}.  

%Similar
%experiences have occurred in the past, notably during the development
%of the functional SISAL language in the early 1990s.

Fortran is unique in that it has contained language features that are
well suited to modern architectures for a number of years.  This
should be unsurprising --- Fortran was a primary language used to
target systems such as the vector supercomputers and massively
parallel systems of the 1970s and 1980s.  These are the systems in
which architectural features were developed that have led to single
chip high performance architectures of interest today.  Given that
these new systems have features very similar to their predecessors, it
is clear that the language features within Fortran for them are still
relevant.

\subsection{Comparison to Other Languages}

A number of previous efforts have exploited data-parallel programming
at the language level to utilize novel architectures, particularly in
previous decades during the reign of vector and massively parallel
computers in the high performance computing world.  The origin of the
array syntax that was adopted in Fortran 90 can be found in the APL
language.  Fortran 90 differed from previous % \cite{iverson79apl}
extensions of Fortran in that parallelism within whole-array
operations was expressed at the expression level instead of via
parallelism within explicit DO-loops (such as within IVTRAN for the
Illiac IV).

The High Performance Fortran (HPF) extension of Fortran 90 was
proposed to add features to the language that would enhance the
ability of compilers to emit fast parallel code for distributed and
shared memory parallel computers\cite{koelbel94hpf}.  One of the
notable additions to the language in HPF was syntax to specify the
distribution of data structures amongst a set of parallel processors.
HPF also introduced an alternative looping construct to the
traditional DO-loop called {\tt FORALL} that was better suited for
parallel compilation.  An additional keyword, {\tt INDEPENDENT}, was
added to allow the programmer to indicate when the order of execution
of the program (such as a sequence of loop iterations) can be flexible
in order to allow parallel execution.  Interestingly, the parallelism
features introduced in HPF did not exploit the new array features
introduced in 1990 in any significant way, relying instead on explicit
loop-based parallelism.  This was likely in order to support parallel
programming that wasn't easily mapped onto a pure data-parallel model.

In some instances though, a purely data-parallel model is appropriate
for part or all of the major computations within a program.  One of
the systems where programmers relied heavily on higher level
operations instead of explicit looping constructs was the Thinking
Machines Connection Machine 5 (CM-5).  A common programming pattern
used on the CM-5 that we exploit in this paper was to write
whole-array operations from a global perspective in which computations
are expressed in terms of operations over the entire array instead of
a single local index.  The use of the array shift intrinsic functions
(like {\tt CSHIFT}) were used to build computations in which arrays
were combined by shifting the entire arrays instead of working based
on local offsets from single indices.  A simple 1D example is one in
which an element is replaced with the average of its own value with
that of its two direct neighbors.  Ignoring boundary indices that wrap
around, explicit indexing will result in a loop such as:

{\small
\begin{verbatim}
  do i = 2,(n-1)
    X(i) = (X(i-1) + X(i) + X(i+1)) / 3
  end do
\end{verbatim}
}

\noindent When shifts are employed, this can be expressed as:

{\small
\begin{verbatim}
  X = (cshift(X,-1) + X + cshift(X,1)) / 3
\end{verbatim}
}

Similar whole array shifting was used in higher dimensions for finite
different codes within the computational physics community, especially
at Los Alamos for codes targeting the CM-5 system that resided there until the
late 1990s.  A body of research in compilation of stencil-based codes
that use shift operators targeting these systems is related to the
work we present here~\cite{stencil-compiler}.

The whole-array model was attractive because it deferred
responsibility for optimally implementing the computations to the
compiler.  Instead of relying on a compiler to infer parallelism from
a set of explicit loops, the choice for how to implement loops was
left entirely up to the tool.  Unfortunately, this had two side
effects that have limited broad acceptance of the whole-array
programming model in Fortran.  First, programmers must translate their
algorithms into a set of global operations.  Finite difference
stencils and similar computations are traditionally defined in terms
of offsets from some central index.  Shifting, while conceptually
analogous, can be awkward to think about for high dimensional stencils
with many points.  Second, the semantics of these operations are such
that all elements of an array operation are updated as if they were
updated simultaneously.  In a program where the programmer explicitly
manages arrays and loops, double buffering techniques and user managed
temporaries are used to maintain these semantics.  When the compiler
is responsible for managing this intermediate storage, it has
historically proven that they are inefficient and generate code that
requires far more temporary storage than really necessary.  This is
not a flaw of the language constructs, but a sign of the lack of
sophistication of the compilers with respect to their internal
analysis to determine how to optimally generate this intermediate
storage.

An interesting line of language research that grew out of the early work
with HPF was that associated with the ZPL language work at the University
of Washington~\cite{chamberlain04zpl}.  In ZPL, programmers adopt a similar
global view of computation over arrays, but define their computations based
on the local view of indices that participate in the update of each element of
an array.  For example, in the 1D averaging case stated above, a programmer would
define the computation in terms of the update that occurs at each index in terms
of relative offsets from that index:
